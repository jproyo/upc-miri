\documentclass[12pt, a4paper]{article}
\usepackage[utf8]{inputenc}
\usepackage{amsmath}
\usepackage{amsthm}
\usepackage{amssymb}
\usepackage{graphicx}
\usepackage{parskip}
\usepackage{hyperref}
\usepackage{fancyhdr}
\usepackage{lastpage}
\usepackage[vlined,ruled]{algorithm2e}
\usepackage[acronym]{glossaries}
\usepackage{caption}
\usepackage{titlesec}
\usepackage{multirow}


\title{%
  Algorithmic Game Theory \\
  Homework 4
}
\author{%
  Juan Pablo Royo Sales\\
  \small{Universitat Politècnica de Catalunya}
}
\date\today

\pagestyle{fancy}
\fancyhf{}
\fancyhead[C]{}
\fancyhead[R]{Juan Pablo Royo Sales - UPC MIRI}
\fancyhead[L]{AGT - Homework 4}
\fancyfoot[L,C]{}
\fancyfoot[R]{Page \thepage{} of \pageref{LastPage}}
\setlength{\headheight}{15pt}
\renewcommand{\headrulewidth}{0.4pt}
\renewcommand{\footrulewidth}{0.4pt}

\renewcommand{\qedsymbol}{$\blacksquare$}

\begin{document}

\maketitle

\section{Problem 4}
\subsection{(a) Valuation Function}
Lets define a variable $x_i$ in which 
\[
x_{ij} = 
\begin{cases}
  1 \text{ if a copy of album } j \text{ is provided by player } i\\
  0 \text{ otherwise } 
\end{cases}
\]

Therefore, 

\begin{equation*}
  v(C) = \Biggr|\biggr\{j \mid \sum_{i \in C} x_{ij} \geq k \biggl\}\Biggl|\\
\end{equation*}

\subsection{(b) Is the game convex?}
\textbf{No, it is not convex because it is not super modular}.

Lets probe it by assuming that the game is super modular.
A game is \textbf{super modular} if $v(C \cup D) + v(C \cap D) \geq v(C) + v(D)$.

Lets take $2$ partitions $C$ and $D$ and lets assume that both partitions by itself has $ \geq k$ copies of some album $j$ but not each player by itself, therefore:

\begin{enumerate}
  \item $v(C) \geq 1 \land v(D) \geq 1$ because at least both partition can re-recording at least the album $j$.
  \item It can be seen by previous that $v(C \cup D) \geq 1$ because we are going to have at least 1 album recording of $j$ because one of the 2 partition group can provide at least $k$ copies and more if we join both.
  \item Lets suppose that the combination of both partition $C \cap D$ leads to a partition in which some players don't contribute copy of album $j$ because some of them contribute with players of partition $C$ and some other move to form coalition with players of partition $D$. In this case coalition $C \cap D$ is not reaching to at least $k$ of album $j$.
  \item By previous statement the only thing we can assure is that $v(C \cap D) \geq 0$ by non-negative condition on Cooperative Games.
  \item So we have that $v(C \cup D) + v(C \cap D) \geq 1$ and $v(C) + v(D) \geq 2$
\end{enumerate}

Therefore $v(C \cup D) + v(C \cap D) \ngeq v(C) + v(D)$

Then, \textbf{it is not convex.}

\subsection{(c) Shapley values}
Knowing that Shapley value is additive we can think on splitting up the game as the sum of several
Shaply values if they were different games. 

Lets define that a game $\Gamma_j$ as the recording of $j$ album type. So,
\[
  v(C) = \begin{cases}
    1 \text{ if } \sum_{i \in C} x_i \geq k\\
    0 \text{ otherwise}
  \end{cases}
  \]

where $x_i$ is the amount of copies of album $j$ that player $i$ contributes.

The Shapley value for this game should be:

\begin{subequations}
  \begin{align}    
    \Phi_i(\Gamma_j) &= \frac{1}{N!} \sum_{C \subseteq N \setminus \{i\}} v(C \cup \{i\}) - v(C)\\
                     &= \frac{1}{N!}\label{eq:1}
  \end{align}
\end{subequations}

This~\ref{eq:1} because once we have $1$ we are not summing up anything.

Using additivity property of Shapley values if we telescope this to the sum of all games of different albums

\begin{subequations}
  \begin{align}    
  \Phi_i(\Gamma) &= \Phi_i\bigr(\sum_{j=1}^m \Gamma_j\bigl)\\
                 &= \sum_{j=1}^n \Phi_i(\Gamma_j)\\
                 &= \frac{m}{N!}
\end{align}
\end{subequations}

It can be easily calculate in \textit{Poly-time} because of the simplicity of the final expression.

\end{document}

