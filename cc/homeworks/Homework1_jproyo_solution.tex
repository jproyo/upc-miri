\documentclass[12pt, a4paper]{article}
\usepackage[utf8]{inputenc}
\usepackage{amsmath}
\usepackage{amsthm}
\usepackage{graphicx}
\usepackage{parskip}
\usepackage{hyperref}
\usepackage{fancyhdr}
\usepackage{lastpage}

\title{%
      Homework 1 \\
      Solutions
}
\author{Juan Pablo Royo Sales}
\date\today

\pagestyle{fancy}
\fancyhf{}
\fancyhead[C]{}
\fancyhead[R]{Juan Pablo Royo Sales - UPC MIRI}
\fancyhead[L]{CC - Homework 1}
\fancyfoot[L,C]{}
\fancyfoot[R]{Page \thepage{} of \pageref{LastPage}}
\setlength{\headheight}{15pt}
\renewcommand{\headrulewidth}{0.4pt}
\renewcommand{\footrulewidth}{0.4pt}

\begin{document}

\maketitle

\section{Exercise 1}
\newtheorem{moderate}{Moderate}
\begin{moderate}
Lets first prove that if $f$ and $g$ are moderate therefore $f \circ g$ is moderate.
\end{moderate}

\begin{proof}
  If $g$ is moderate by definition of the problem, then $\exists\ p(n): |g(x)| \leq p(|x|)\ \forall x \in \{0,1\}^*$.

  Lets define a $TM\ G$ which executes $g$ in polinomial time in $n$.

  Lets call $y = p(|x|)$ to the result of executing $G$ by definition of moderation on $g$.

  And then applying,

  \begin{subequations}
    \begin{align}
      |f(g(x))| &\leq p(|g(x)|) \label{eq:1} \\
                &\leq p(|p(|x|)|) \label{eq:2}\\
                &= p(|n^c|) \label{eq:3} \\
                &= n^{c^{c'}} \label{eq:4} \\
                &= n^{cc'} \label{eq:5} \\
                &= n^d
    \end{align}
  \end{subequations}

  \ref{eq:1} and \ref{eq:2} by definition of the problem because $f$ is moderate.
  Replacing \ref{eq:3} by definition of polynomial. We can replace again in \ref{eq:4} by definition of polynomial. We can derived \ref{eq:5} by exponential of exponential law which is the multiplication of both constant replacing $c{c'} = d$.

  We can state that the upper bound of the composition is polynomial in their input $\exists\ p(n): |f(g(x))| \leq p(|g(x)|)\ \forall x \in \{0,1\}^*$.

  Therefore $f \circ g$ is also moderate.
\end{proof}

\newtheorem{logspace}{Logspace computable}

\begin{logspace}
  If $f$ and $g$ are \textbf{logspace commputable} then $f \circ g$ is \textbf{logspace computable}.
\end{logspace}

\begin{proof} \label{proof:logspace}
    In order to prove that \textbf{logspace computable} functions are closed under composition, we need to first run $g(x)$ and to the result of that applied $f$ using \textbf{logspace computation}.

    Taking the $\text{BIT}_g$ we know by that and because $g$ is moderate that the output of $g$ is of some polynomial length at least $p(|x|)$. This is going to be the input of $f$ in the composition.

    Although both machines of $f$ and $g$ use space $O(log(n))$ by definition of the problem, we cannot use that for proving the composition in logspace because the output of $g$ is going to be use as the input of machine $f$ and we need to store that in a work tape. If we do that as we know that the output of $g$ is polynomial we are not going to have logspace. The way to do this is to transform the output of $g$ in something that use $O(log(n))$ space.

    In order to do so, we are going to use the \textbf{bit-graph} of $g$ $\text{BIT}_g$. Using the bit-graph we know what is the output of $g(x)$ in the $i\text{-th}$ bit by definition of the problem, without needing to store its output.

    When we are running $f$ on $g(x)$ we only need 1 bit at time of the output and we are going to obtain from the bit-graph, using two counters with $O(log(n))$ space on a work-tape indicating the number of the next bit needed by $f$ from $g(x)$.

    The algorithm should look as the following:

    \begin{enumerate}

      \item For each bit on $f$ that we need to compute

      \item Increment counter according to the bit number $f$ needs from the $g$ output. For example if we need the $n\text{-th}$ input bit for $f$, we are going to increment counter by $n$ \label{algo:1}

      \item Reset $g$ input $x$ to beginig and use that counter to count down moving the posision until counter is 0, to the output bit $i$ we need to find. So we found bit-graph $g(x)_{i}$ output bit. \label{algo:2}

      \item Process bit output calculated in the previous step as the input bit of $f$.

      \item After processing $\text{BIT}_f$ on the previous output bit on $f$ we need to check if it is 1 or not to decide the language. We count in another count the amount of 1. \label{algo:3} \\

      \item At the end we decrement the counter in the size of $|f(x)|$. If the result is $0$ we halt and we output 1, we output 0 otherwise.

    \end{enumerate}

    Here we can see that counters \ref{algo:1}, \ref{algo:2} and \ref{algo:3} only need $O(log(n))$ space on $2^n$ bits of the $i\text{-th}$ positions, so both machines uses $O(log(n))$ space.

    Therefore $f \circ g$ is \textbf{logspace computable}.
  \end{proof}

\section{Exercise 2}
\newtheorem{verifier}{Logspace Verifier}

\begin{verifier}
  Assuming that $L = \text{SPACE}(log(n))$ we want to prove that if we replace $P$ by $L$ we still can have a characterization of $NP$ like the one defined in the problem:
  \begin{align*}
    x \in A \iff \exists\ y \in \{0,1\}^* : |y| \leq p(|x|)\ \land \langle x,y \rangle \in B
  \end{align*}
 \end{verifier}

  But since $B$ is in $P$ and not in $L$, we need something that transform $B$ to $L$ maintaining $NP$ characteristics.

 Having in mind the following:

\newtheorem{cook-levin}{Cook-Levin Theorem} \label{th:cook}
\begin{cook-levin}
    $SAT$ is \textbf{NP}-complete.
    $3SAT$ is \textbf{NP}-complete.
\end{cook-levin}

 \begin{proof}
   We know by \ref{th:cook} that because $SAT$ is \textbf{NP}-complete, it is \textbf{NP}-hard and polynomial-time reducible. Because of that we know that \textit{every} \textbf{NP} language $L$ can be reduced to $SAT$.

   If we know that by the theorem, by the definition of the problem we can reduce the language $B$ which is in $NP$ to a $SAT$.

   We define $\langle x,y \rangle \in B \iff y=f(x);\omega$, where $f$ is the reduction from $L$ to $SAT$, and $\omega$ is a satifiable truth assignment of $f(x)$.

   The machine computing $f(x)$ compares 1 by 1 the output with $y$ in logspace, telling when $\omega$ satifies $f(x)$ and does this in logspace.

   Logspace usage can be trivially proved using a counter that compares the position of bit $i$-th with the output of $f(x)_i$. Basically we count 1 each time we advance in output bit $i$ on $f(x)_i$. We use that counter to find the position on $y$ to be compare with.

   Therefore we can replace $P$ with $L$ and still have $NP$ characterization.
 \end{proof}


\end{document}

