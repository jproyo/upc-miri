\documentclass[12pt, a4paper]{article}
\usepackage[utf8]{inputenc}
\usepackage{amsmath}
\usepackage{amsthm}
\usepackage{amssymb}
\usepackage{graphicx}
\usepackage{parskip}
\usepackage{hyperref}
\usepackage{fancyhdr}
\usepackage{lastpage}
\usepackage[vlined,ruled]{algorithm2e}
\usepackage[acronym]{glossaries}
\usepackage{caption}
\usepackage{titlesec}
\usepackage{tikz}
\usetikzlibrary{arrows,automata}

\titleformat{\section}
  {\normalfont\bfseries}{Problem 7.\thesection}
  {0em}{}

\titleformat{\subsection}
  {\normalfont\bfseries}{7.\thesubsection}
  {0em}{}

\titleformat{\subsubsection}
  {\normalfont\bfseries}{7.\thesubsection}
  {0em}{}

\title{%
  Stochastic Network Modeling \\
  Homework 7 - Solutions
}
\author{%
  Juan Pablo Royo Sales\\
  \small{Universitat Politècnica de Catalunya}
}
\date\today

\pagestyle{fancy}
\fancyhf{}
\fancyhead[C]{}
\fancyhead[R]{Juan Pablo Royo Sales - UPC MIRI}
\fancyhead[L]{SNM - Homework 7}
\fancyfoot[L,C]{}
\fancyfoot[R]{Page \thepage{} of \pageref{LastPage}}
\setlength{\headheight}{15pt}
\renewcommand{\headrulewidth}{0.4pt}
\renewcommand{\footrulewidth}{0.4pt}

\renewcommand{\qedsymbol}{$\blacksquare$}

\begin{document}

\maketitle

\section{}
\subsection{}
The period of the chain is $d = 3$ and the cyclic classes are: $C_0 = \{a,d,e,f\}, C_1 = \{b,c\}, C_2=\{g,h\}$

\subsection{}
\begin{align*}
  P = \begin{bmatrix}
       & a & b & c & d & e & f & g & h \\
     a & 0 & 0.2 & 0.8 & 0 & 0 & 0 & 0 & 0 \\
     b & 0 & 0 & 0 & 0.3 & 0.7 & 0 & 0 & 0 \\
     c & 0 & 0 & 0 & 0 & 0.3 & 0.7 & 0 & 0 \\
     d & 0 & 0 & 0 & 0 & 0 & 0 & 1 & 0 \\
     e & 0 & 0 & 0 & 0 & 0 & 0 & 0.5 & 0.5 \\
     f & 0 & 0 & 0 & 0 & 0 & 0 & 0 & 1 \\
     g & 0.8 & 0 & 0 & 0.2 & 0 & 0 & 0 & 0 \\
     h & 0.8 & 0 & 0 & 0 & 0 & 0.2 & 0 & 0 \\
  \end{bmatrix}
\end{align*}

\begin{align*}
  P^7 = \begin{bmatrix}
   0.00000 & 0.03200 & 0.12800 &  0.00000 &  0.00000 &  0.00000 &  0.21000 &  0.63000\\
   0.67200 & 0.00000 & 0.00000 &  0.05480 &  0.06080 &  0.21240 &  0.00000 &  0.00000\\
   0.67200 & 0.00000 & 0.00000 &  0.05080 &  0.06080 &  0.21640 &  0.00000 &  0.00000\\
   0.00000 & 0.13440 & 0.53760 &  0.00000 &  0.00000 &  0.00000 &  0.08800 &  0.24000\\
   0.00000 & 0.13440 & 0.53760 &  0.00000 &  0.00000 &  0.00000 &  0.08400 &  0.24400\\
   0.00000 & 0.13440 & 0.53760 &  0.00000 &  0.00000 &  0.00000 &  0.08000 &  0.24800\\
   0.26240 & 0.00000 & 0.00000 &  0.05792 &  0.25536 &  0.42432 &  0.00000 &  0.00000\\
   0.26240 & 0.00000 & 0.00000 &  0.05632 &  0.25536 &  0.42592 &  0.00000 &  0.00000\\
 \end{bmatrix}
\end{align*}

\begin{align*}
  P^{15} = \begin{bmatrix}
   0.00000 &  0.06559 &  0.26235 &  0.00000 &  0.00000 &  0.00000 &  0.16802 &  0.50405\\
   0.53765 &  0.00000 &  0.00000 &  0.05328 &  0.12462 &  0.28445 &  0.00000 &  0.00000\\
   0.53765 &  0.00000 &  0.00000 &  0.05328 &  0.12462 &  0.28446 &  0.00000 &  0.00000\\
   0.00000 &  0.10753 &  0.43012 &  0.00000 &  0.00000 &  0.00000 &  0.11560 &  0.34675\\
   0.00000 &  0.10753 &  0.43012 &  0.00000 &  0.00000 &  0.00000 &  0.11559 &  0.34676\\
   0.00000 &  0.10753 &  0.43012 &  0.00000 &  0.00000 &  0.00000 &  0.11558 &  0.34676\\
   0.36988 &  0.00000 &  0.00000 &  0.05538 &  0.20431 &  0.37044 &  0.00000 &  0.00000\\
   0.36988 &  0.00000 &  0.00000 &  0.05538 &  0.20431 &  0.37044 &  0.00000 &  0.00000\\
\end{bmatrix}
\end{align*}

\section{}
\subsection{}
Period is $d=2$ and the cycles are 2 $C_0 = \{a,c\}, C_1=\{b\}$

\subsection{}
\begin{align*}
  P = \begin{bmatrix}
       & a & b & c\\
     a & 0 & 1 & 0 \\
     b & \alpha & 0 & 1-\alpha \\
     c & 0 & 1 & 0 \\
  \end{bmatrix}
\end{align*}


\begin{align*}
  P^2 = \begin{bmatrix}
       & a & b & c\\
     a & 0 & 1 & 0 \\
     b & \alpha^2 & 0 & (1-\alpha)^2 \\
     c & 0 & 1 & 0 \\
  \end{bmatrix}
\end{align*}


\begin{align*}
  P^{20} = \begin{bmatrix}
       & a & b & c\\
     a & 0 & 1 & 0 \\
     b & \alpha^{20} & 0 & (1-\alpha)^{20} \\
     c & 0 & 1 & 0 \\
  \end{bmatrix}
\end{align*}


\begin{align*}
  P^{\infty} = \begin{bmatrix}
       & a & b & c\\
     a & 0 & 1 & 0 \\
     b & 0 & 0 & 0 \\
     c & 0 & 1 & 0 \\
  \end{bmatrix}
\end{align*}

\subsection{}
Given the chain the eigenvalues are $\lambda_1 = 1, \lambda_2 = -1, \lambda_3 = 0$

Therefore for $\pi_a(n)$ we have:

\begin{subequations}
  \begin{align}
    \pi_a(n) &= a \lambda_1^n + b \lambda_2^n \\
    \pi_a(0) &= a + b = (\pi(0)P^0)_a = 1\\
    \pi_a(1) &= a - b = (\pi(0)P^1)_a = 0\\
    \pi_a(2) &= a + b = (\pi(0)P^2)_a = \alpha\\
  \end{align}
\end{subequations}
Solving the equation system we have that $a = \frac{2-\alpha}{2}, b = \frac{\alpha}{2}$.\\
Therefore, $\pi_a(n) = \frac{2-\alpha}{2} + (-1)^n \frac{\alpha}{2}$

Therefore for $\pi_b(n)$ we have:

\begin{subequations}
  \begin{align}
    \pi_b(n) &= a \lambda_1^n + b \lambda_2^n \\
    \pi_b(0) &= a + b = (\pi(0)P^0)_b = 0\\
    \pi_b(1) &= a - b = (\pi(0)P^1)_b = 1\\
    \pi_b(2) &= a + b = (\pi(0)P^2)_b = 0\\
  \end{align}
\end{subequations}
Solving the equation system we have that $a = \frac{1}{2}, b = -\frac{1}{2}$.\\
Therefore, $\pi_b(n) = \frac{1}{2} + (-1)^n -\frac{1}{2}$

Therefore for $\pi_c(n)$ we have:

\begin{subequations}
  \begin{align}
    \pi_c(n) &= a \lambda_1^n + b \lambda_2^n \\
    \pi_c(0) &= a + b = (\pi(0)P^0)_c = 0\\
    \pi_c(1) &= a - b = (\pi(0)P^1)_c = 0\\
    \pi_c(2) &= a + b = (\pi(0)P^2)_c = 1-\alpha\\
  \end{align}
\end{subequations}
Solving the equation system we have that $a = \frac{1-\alpha}{2}, b = \frac{1-\alpha}{2}$.\\
Therefore, $\pi_c(n) = \frac{1-\alpha}{2} + (-1)^n \frac{1-\alpha}{2}$

\section{}
I havent follow this problem. How to compute the stationary distribution without values? On the other hand i havent followed the problem properly.
 \end{document}

