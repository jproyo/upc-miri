\documentclass[12pt, a4paper]{article}
\usepackage[utf8]{inputenc}
\usepackage{amsmath}
\usepackage{amsfonts}
\usepackage{amsthm}
\usepackage{array}
\usepackage{graphicx}
\usepackage{parskip}
\usepackage[pdfencoding=auto]{hyperref}
\usepackage{fancyhdr}
\usepackage{lastpage}
\usepackage{tikz}
\usepackage{float}
\usepackage{listings}
\usepackage{color}
\usepackage{caption}
\usepackage{authblk}
\usepackage[acronym]{glossaries}
\usepackage[nottoc]{tocbibind}
\usepackage[cache=false]{minted}
\usemintedstyle{default}
\newminted{haskell}{frame=lines,framerule=2pt}
\newminted{R}{frame=lines,framerule=2pt}
\newminted{cpp}{frame=lines,framerule=2pt}
\graphicspath{{./images/}}
\DeclareMathOperator*{\argmax}{arg\,max}
\DeclareMathOperator*{\argmin}{arg\,min}

\tikzstyle{bag} = [align=center]

\title{%
      Homework 3\\
      Significance of Metrics\\
}
\author{Juan Pablo Royo Sales}
\affil{Universitat Politècnica de Catalunya}
\date\today

\pagestyle{fancy}
\fancyhf{}
\fancyhead[C]{}
\fancyhead[R]{UPC MIRI}
\fancyhead[L]{CSN - Homework 3}
\fancyfoot[L,C]{}
\fancyfoot[R]{Page \thepage{} of \pageref{LastPage}}
\setlength{\headheight}{15pt}
\renewcommand{\headrulewidth}{0.4pt}
\renewcommand{\footrulewidth}{0.4pt}

\newacronym{lang}{Language Set}{Provided Language Set}
\newacronym{switching}{Switching Model}{Null Model based on Switching}
\newacronym{binomial}{Binomial Model}{Null Model based on Binomial Erdos Renyi}
\newacronym{cc}{Closeness Centratlity}{Closeness Centratlity Measure}
\newacronym{nh}{Null Hypothesis}{Null Hypothesis}
\newacronym{mc}{Montecarlo}{Montecarlo Approximation Algorithm}

\begin{document}

\maketitle

\tableofcontents

\section{Introduction}
In this work I have been analyzing the significance of \acrfull{cc} metric over a \acrfull{lang} provided for such case.
Basically the idea of the work is the following: Giving this set of Languages with some set of connected terms, determine 
the \acrshort{cc} accurately measured by:

\begin{equation}
    C = \frac{1}{N} \sum_{i}^N C_i
\end{equation}

where,

\begin{equation}
    C_i = \frac{1}{N-1} \sum_{j \neq i}^N \frac{1}{d_{ij}}
\end{equation}


After that I am going to measure the \textbf{Significance} of that metric against different \acrshort{nh}, in particular against \acrfull{switching} and \acrfull{binomial}.

Using a \acrfull{mc} I am going to approximate the \textit{p-value} of both \textbf{Null} Hypothesis against the $C$ for each language.

The present work is divided in the following sections:

\begin{itemize}
    \item \textbf{Results} section: Where I show the results obtained for each language according to the previous statements.
    \item \textbf{Discussion} section: In this section I will provide some discussion about the results and my personal impression.
    \item \textbf{Methods} section: Where i am going to describe each of the method and technique use to do the work.
    \item \textbf{Conclusions} section: Finally I am going to give my opinion about the results obtained and what I have learnt.
\end{itemize}

\section{Results}
Lets divide the results for each language, but first lets see the Summary of the data that I am analyzing.

\subsection{Languages Main Characteristics}

\begin{table}[H]
    \centering
    \begin{tabular}{l l l l l}
    Language & N & E & $<k>$ & $\delta$ \\
     \hline
     Arabic & 21532 & 68743 & 6.3851941 & 0.0002966 \\ 
     Basque & 12207 & 25541 & 4.1846482 & 0.0003428 \\ 
     Catalan & 36865 & 197075 & 10.6917130 & 0.0002900 \\ 
     Chinese & 40298 & 180925 & 8.9793538 & 0.0002228 \\ 
     Czech & 69303 & 257254 & 7.4240365 & 0.0001071 \\ 
     English & 29634 & 193078 & 13.0308430 & 0.0004397 \\ 
     Greek & 13283 & 43961 & 6.6191372 & 0.0004984 \\ 
     Hungarian & 36126 & 106681 & 5.9060510 & 0.0001635 \\ 
     Italian & 14726 & 55954 & 7.5993481 & 0.0005161 \\ 
     Turkish & 20409 & 45625 & 4.4710667 & 0.0002191
    \end{tabular}
   \caption{Language List - Main Characteristics}
   \label{table:1}
\end{table}

\subsection{Closeness Centrality Measures}

\begin{table}[H]
  \centering
  \begin{tabular}{l c c c}
  Language & Metric & p-value (switching) & p-value (binomial)\\
   \hline
   Arabic & 0.3264629  & 1.00 & 0.00 \\
   Basque & 0.2697357  & 0.00 & 0.00 \\
   Catalan & 0.3410137 & 1.00 & 0.00 \\
   Chinese & 0.3264533 & 0.00 & 0.00 \\
   Czech & 0.3059504 & 0.392 & 0.00 \\
   English & 0.3435140  & 0.00 & 0.00 \\
   Greek & 0.3147265  & 1.00 & 0.00 \\
   Hungarian & 0.2883451 & 1.00 & 0.00 \\
   Italian & 0.3278253 & 1.00 & 0.00 \\
   Turkish & 0.3603387  & 0.00 & 0.00 \\
     \end{tabular}
 \caption{Language List - Closeness Centrality}
 \label{table:2}
\end{table}


\section{Discussion}
\subsection{Switching}
Accordingly to the results obtained as it can be appreciate in~\ref{table:2} the majority of the \textbf{p-value}, it means $\frac{6}{10}$ for the 
\acrshort{switching} are greater than $0.05$. This means that I can \textbf{accept the \acrfull{nh}} for the \acrshort{switching} in the case of those languages. 
As we can see for almost all iterations \acrshort{cc} value is greater than the real model $C$.
For detailed iterations and values of each \acrshort{cc} of each iteration and for each language, please see results on \mintinline{bash}{results/results_montecarlo_XXX_1.txt} 
where $XXX$ is the raw results for language $XXX$ of the \acrshort{mc}, where it is describe the result for each iteration.

Lets analyze what happen for those cases that I \textbf{cannot accept} \acrshort{nh} because the \textit{p-value} is $0.00$. For doing that I will purpose to analyze each of those 
rejected language against some language whose \textit{p-value} \textbf{has been accepted} and it has similar characteristics. So lets define Similar. 

Given the set of languages $L = \{Arabic, \dots, Turkish\}$ A language $S_i$ is similar to other language if the difference in number of vertices are minimum with respect to $L$:

\begin{equation}
    S_i = \argmin_{j \in L, j \neq i} \{ |N_j - N_i| \land p_{value}(j) > 0.05 \}
\end{equation}

\begin{itemize}
    \item \textbf{Basque}: $S_{Basque} = Greek$. In this case we can appreciate that $<k>$ and $\delta$ in \textbf{Greek} is higher which indicate that the probability of being connected with central nodes are higher. 
    This can explained why \acrshort{switching} preserves the structure and centrality in this case, which is more difficult to achieve with \textbf{Basque}.
    \item \textbf{Chinese}: $S_{Chinese} = Catalan$. In this case we observe the same as in the previous case of \textbf{Basque}. This language \textbf{Chinese} has higher $<k>$ and $\delta$ rather than \textbf{Catalan}
    which is the most Similar with acceptance of \acrshort{nh}. 
    \item \textbf{English}: $S_{English} = Hungarian$. In this case since the $<k>$ is extremely high in comparison to the density this indicates that although \textbf{English} has a higher degree on average, that does not
    mean are dense, which in random permutations those start to be lost for the number of terms and edges.
    \item \textbf{Turkish}: $S_{Chinese} = Catalan$. In the case of \textbf{Chinese} we can appreciate that the $\delta$ is lower than \textbf{Catalan} which could indicate the same as \textbf{English}.
\end{itemize}
 

\subsection{Binomial}
Regarding the \acrshort{binomial} none of the random generated graphs has been close to the real \acrshort{cc} of the given models,
and it can be seen by the results output file \mintinline{bash}{docs/results_montecarlo_XXX_2.txt} where $XXX$ is the raw results for language $XXX$ of the \acrshort{mc}, 
that all of the them behaves in a stable number according to the number of \textbf{Edges and Vertices} that were provided for generating the Random Graph.

This could be explain by the fact that \acrshort{binomial} are fully random graph and \acrshort{cc} is an intrinsic property of \textbf{non random graph}, because it indicates how concentrated 
are the elements around some of them. In the previous \acrshort{switching} this behavior was different and it gives more accurate \acrshort{cc} values, because it is a permutation
over the same graph, and in that permutation I tried to preserve the structure of the original graph, although the selection of the \textbf{vertices} that are selected for switching itself is random.

Therefore, I need to \textbf{reject} the \acrshort{nh} in the case of \acrshort{binomial}.

\section{Methods}
In this section i am going to describe the method used for arriving to this conclusions, starting from analyzing the source code, following up to the parameters selected
for running the analysis.

\subsection{Source Code Implementation}
When we talk about the methodology that it is used for conducting this kind of experiments it is a very important to remark all those decisions regarding the implementation in the source
code.
As it can be seen in the folder of this distribution, the language use was \textbf{C++}.

\subsubsection{Graph Representation}

Basically the Graph of the language is represented by an Adjacency matrix of the \textbf{Vertex} and \textbf{Edges} that are connected to those edges.

\begin{listing}[H]
    \inputminted[firstline=40, lastline=46, breaklines]{cpp}{../src/domain/graph.cc}
    \caption{Extracted from source code graph.cc}
    \label{source:code:1}
\end{listing}  

All the computations are done over the array of edges \mintinline{cpp}{vector<int> *adj} which is the adjacency list.

\subsubsection{Order of Edges}
After several and different experimentation I have finally decide to order the edges by \textbf{Degree sequence in descendent order}.
This order is done just before calculating the \acrshort{cc} to speed and optimize the process. In fact \textbf{Adjacency list of nodes are not sorted at all},
the sorting process is done \textit{on-the-fly} when we need to iterate over the vertices for calculate \acrshort{cc}.

\begin{listing}[H]
    \inputminted[firstline=272, lastline=281, highlightlines={277-279}, breaklines]{cpp}{../src/domain/graph.cc}
    \caption{Extracted from source code graph.cc}
    \label{source:code:2}
\end{listing}  

\subsubsection{Closeness Centrality}
The calculation of \acrshort{cc} is as it was proposed in the work, a regular \textbf{BFS} Bread First Search algorithm 
that can be seen here:

\begin{listing}[H]
    \inputminted[firstline=167, lastline=167, breaklines]{cpp}{../src/domain/graph.cc}
    \caption{Extracted from source code graph.cc}
    \label{source:code:3}
\end{listing} 

In order to \textbf{speed the process} I have also implemented the proposal of the work to calculate for those nodes whose 
$k < 2$ taking the distance of the parent $ + 1$. 

\begin{listing}[H]
    \inputminted[firstline=126, lastline=156, highlightlines={127-132,145-153}, breaklines]{cpp}{../src/domain/graph.cc}
    \caption{Extracted from source code graph.cc}
    \label{source:code:3}
\end{listing} 

Also I have implemented the \textbf{Optimization} regarding to not calculate the whole \acrshort{cc} for the whole graph, and 
only a portion of them taking and $M = \frac{N}{1000}$.

\begin{listing}[H]
    \inputminted[firstline=284, lastline=296, breaklines]{cpp}{../src/domain/graph.cc}
    \caption{Extracted from source code graph.cc}
    \label{source:code:4}
\end{listing} 

\subsubsection{Montecarlo}
The main algorithm can be found here:

\begin{listing}[H]
    \inputminted[firstline=37, lastline=59, breaklines]{cpp}{../src/graph/aprox.cc}
    \caption{Extracted from source code aprox.cc}
    \label{source:code:8}
\end{listing} 


In the case of \acrshort{mc} I have implemented the generation of the \acrshort{binomial} which can be found here:

\begin{listing}[H]
    \inputminted[firstline=74, lastline=98, breaklines]{cpp}{../src/graph/aprox.cc}
    \caption{Extracted from source code aprox.cc}
    \label{source:code:5}
\end{listing} 

There are more lines that it is not appearing here~\ref{source:code:5} because to fit in the page.

And in the case of \acrshort{switching} it is a slightly more complicated because i needed to take care of no breaking the 
structure of the graph when i toss the coin to switch 2 terminals of 2 different edges.

Basically the main implementation of the \acrshort{switching} can be seen here:

\begin{listing}[H]
    \inputminted[firstline=61, lastline=71, breaklines]{cpp}{../src/graph/aprox.cc}
    \caption{Extracted from source code aprox.cc}
    \label{source:code:6}
\end{listing} 

But the \mintinline{cpp}{TrySwitch} method which is doing the whole work can be found here:

\begin{listing}[H]
    \inputminted[firstline=299, lastline=299, breaklines]{cpp}{../src/domain/graph.cc}
    \caption{Extracted from source code graph.cc}
    \label{source:code:7}
\end{listing}

There are more lines that it is not appearing here~\ref{source:code:7} because to fit in the page.

\subsection{Optimization and Improvements}\label{sub:section:opt}
I have done both optimizations proposed by the work:

\begin{itemize}
    \item Not calculating exhaustively \acrshort{cc} for nodes whose degree is less than 2 $k < 2$. This has been
    shown here~\ref{source:code:3}
    \item In the case of \acrshort{mc} Calculate a portion $M$ for both the real model and the \acrshort{nh}. This has 
    been explained here~\ref{source:code:4}
\end{itemize}

\subsection{Experiment Parameters}
Implementing all the optimizations described before~\ref{sub:section:opt}, I was able to produce experiments with higher value of $T$
for all languages in the case of \acrshort{switching}.

\begin{itemize}
    \item $T = 1000$ 
    \item $Q = 15$
\end{itemize}

For the case of \acrshort{binomial} since i detected early that because of the fully randomization i will never going to achieve a close result, i decided
to setup in the minimum value which is $100$.
\begin{itemize}
    \item $T = 100$ 
\end{itemize}

All the outputs of the experiment with each value generated can be found as i pointed out at the beginning of this document 
\mintinline{bash}{results/results_montecarlo_XXX_1.txt}, suffix $1$ for the case of \acrshort{switching} and $2$ for \acrshort{binomial}.

\section{Conclusions}
One of the first conclusions i can arrive after doing these experiments is that \acrshort{switching} is a more accurate
tool in order to measure the significance of \acrshort{cc}. On the other hand \acrshort{binomial} is something that in the case
of \acrshort{cc} doesn't apply because of the fully randomization of the generated model, which cannot represent properly a strong
relationship regarding a set of important nodes. 

Moreover, it can be appreciated that perhaps regardless the value of $T$ \acrshort{binomial} is not suitable as measuring tool for this
metric.

As a future work, it would be important to explore other mechanisms to tunning better the \acrshort{nh} model generators in order
to have the same level of randomization but preserving better \acrshort{cc} characteristics of the original model.

In conclusion although it is quite hard to iterate over a large number of $T$ in order to measure properly the \acrshort{cc} based on the \acrshort{nh},
it has been seen that \acrshort{switching} gives us an accurate approximation.

\printglossary[type=\acronymtype]

\appendix\label{apx:org}
\section{Source Code and Assets}
The source code and different assets that are contained in these folders are classified in the following way.

\begin{itemize}
    \item \textbf{bin} This folder contains the compiled binary files for the code that implements the requirements suggested by the lab work.
    \item \textbf{data} This folder contains the input \acrshort{lang} provided.
    \item \textbf{docs} This folder contains this report in Latex and PDF compiled format.
    \item \textbf{out} This folder contains the standard output results after executing each program over the \acrshort{lang}
    \item \textbf{results} This folder contains the standard output of the raw results obtained in the experiments conducted for each \acrshort{lang}
    \item \textbf{src} This folder contains the \mintinline{bash}{cpp} files with the source code implementation.
    \item \textbf{\mintinline{bash}{./Makefile}} Makefile to compile programs
    \item \textbf{\mintinline{bash}{./README.md}} Markdown file with the instructions to run the program.
    \item \textbf{\mintinline{bash}{./run_closeness.sh}} Script file for running the algorithm over the \acrshort{lang} in order to calculate \textit{Closeness Centrality} and 
    main features of each language.
    \item \textbf{\mintinline{bash}{./run_montecarlo.sh}} Script file for running the approximation algorithms for each language with null models.
\end{itemize}

\section{Running and Compiling}
\subsection{Compiling}

\begin{listing}[H]
    \begin{minted}{bash}
        > make clean
        > make all
    \end{minted}
\end{listing}

All the binaries are going to be place in \mintinline{bash}{bin/} directory.

\subsection{Help}

\begin{listing}[H]
    \begin{minted}{bash}
        > bin/closeness -h
        > bin/montecarlo -h
    \end{minted}
\end{listing}

\subsection{Running for a Single Language}
\subsubsection{Closeness Centrality for real Model}

\begin{listing}[H]
    \begin{minted}{bash}
        > bin/closeness "data/Basque_syntactic_dependency_network.txt"
    \end{minted}
\end{listing}

\subsubsection{Montecarlo Approximation for Switching Model}

\begin{listing}[H]
    \begin{minted}{bash}
        > bin/montecarlo "data/Basque_syntactic_dependency_network.txt" T Q 1
    \end{minted}
\end{listing}

Where $T$ is the number of iterations of Montecarlo Method and $Q$ the number of Switching that is going to be multiplied by the number of Edges to get the total number of switching.

\subsubsection{Montecarlo Approximation for Binomial Model}

\begin{listing}[H]
    \begin{minted}{bash}
        > bin/montecarlo "data/Basque_syntactic_dependency_network.txt" T Q 2
    \end{minted}
\end{listing}

\subsection{Running for ALL Language}

\begin{listing}[H]
    \begin{minted}{bash}
        > sh run_closeness.sh
    \end{minted}
\end{listing}

\begin{listing}[H]
    \begin{minted}{bash}
        > sh run_montecarlo.sh
    \end{minted}
\end{listing}

\textbf{WARNING: This script might take more than 6 HOURS TO FINISH.}



\end{document}