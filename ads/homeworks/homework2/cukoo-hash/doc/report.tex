\documentclass[12pt, a4paper]{article}
\usepackage[utf8]{inputenc}
\usepackage{amsmath}
\usepackage{amsfonts}
\usepackage{amsthm}
\usepackage{graphicx}
\usepackage{parskip}
\usepackage{hyperref}
\usepackage{fancyhdr}
\usepackage{lastpage}
\usepackage{tikz}
\usepackage{float}
\usepackage{listings}
\usepackage{color}
\usepackage{caption}
\usepackage[acronym]{glossaries}
\usepackage[nottoc]{tocbibind}
\usepackage[cache=false]{minted}
\usemintedstyle{default}
\newminted{haskell}{frame=lines,framerule=2pt}
\graphicspath{{./images/}}

\tikzstyle{bag} = [align=center]

\lstset{frame=tb,
  language=Haskell,
  aboveskip=3mm,
  belowskip=3mm,
  showstringspaces=false,
  columns=flexible,
  basicstyle={\small\ttfamily},
  numbers=left,
  numberstyle=\tiny\color{gray},
  keywordstyle=\color{blue},
  commentstyle=\color{dkgreen},
  stringstyle=\color{mauve},
  breaklines=true,
  breakatwhitespace=true,
  tabsize=2,
  stepnumber=1,
  escapechar=`
}

\definecolor{dkgreen}{rgb}{0,0.6,0}
\definecolor{gray}{rgb}{0.5,0.5,0.5}
\definecolor{mauve}{rgb}{0.58,0,0.82}

\title{%
      Homework 2 \\
      Experimental Prove of Cukoo Hashing features
}
\author{%
  Juan Pablo Royo Sales \\
  \small{Universitat Politècnica de Catalunya}
}
\date\today

\pagestyle{fancy}
\fancyhf{}
\fancyhead[C]{}
\fancyhead[R]{Juan Pablo Royo Sales - UPC MIRI}
\fancyhead[L]{ADS - Homework 2}
\fancyfoot[L,C]{}
\fancyfoot[R]{Page \thepage{} of \pageref{LastPage}}
\setlength{\headheight}{15pt}
\renewcommand{\headrulewidth}{0.4pt}
\renewcommand{\footrulewidth}{0.4pt}

\makeglossaries

\newacronym{cuk}{CUKOO}{Cukoo Hashing}
\newacronym{ccc}{CCC}{Complex Connected Component}
\newacronym{bpgraph}{BGRAPH}{Bipartite Graph}
\newacronym{haskell}{Haskell}{Haskell Programming Language}

\begin{document}

\maketitle

\section{Introduction}
In this homework I am going to show through experimentation analysis two of the most important features of \acrfull{cuk} that are describe in detail here \cite{cukoo}:

\begin{itemize}
  \item Insertion amortized expected $\theta(1)$
  \item The probability that \acrshort{cuk} contains \acrfull{ccc} in terms of the Hastables that forms a \acrfull{bpgraph} when $m = (1+\epsilon)n$ is $m =O(1/n)$ \cite{kutze}
\end{itemize}

I am going to show through experimental analysis, that empirically we can implement a \acrshort{cuk} that fulfill these properties.

In order to see how the code and resources are organized and where to find all the material need, please check~\ref{appendix} section.

\section{Implementation}
The \acrshort{cuk} implementation that has been done for running this experiments is in \acrfull{haskell}.

I have implemented a full feature \acrshort{cuk} with some limitations:

\begin{haskellcode*}{}
module Data.Hash.Cukoo
  ( create
  , insert
  , delete
  , lookup
  , toList
  , elements
  , rehashed
  , rehashesCount
  , length
  ) where
\end{haskellcode*}

The first limitation that is important to pointed out is the \mintinline{haskell}{create} function which creates a new \acrshort{cuk} empty data structure. My implementation does not support multiple Hashtables indirections, and I have worked with a \acrshort{cuk} implementation of \textbf{2} Hashtables.

Another important limitation is that I only allow \mintinline{haskell}{Integer} Positive values $x | x \in \mathbb{N}$. This restriction is allowing me to used \textbf{Unboxed Values} which increase the performance of my Data Type in big order of magnitude. You can check \textbf{Unboxed Values} explanation on \acrshort{haskell} here \cite{unboxed}.

As we can see in the \textbf{export} module list of the \acrshort{cuk} implementation we have the following functions exported and implemented:

\begin{itemize}
  \item Complete \mintinline{haskell}{insert}, \mintinline{haskell}{delete} and \mintinline{haskell}{lookup} functions
  \item \mintinline{haskell}{toList} to convert the \acrshort{cuk} Data Type to a regular list with all the keys in the Hashtable.
  \item \mintinline{haskell}{elements} Count the number of elements in the Hashtable
  \item \mintinline{haskell}{rehashed} Return \mintinline{haskell}{True} if the hash was rehashed on the last \mintinline{haskell}{insert}
  \item \mintinline{rehashesCount} Return the number of rehashes so far.
  \item \mintinline{haskell}{length} Return the length of both internal Hashtables. In fact this is going to return $2M$ where $M$ is the size of 1 of the internal hashtable.
\end{itemize}

\section{Experiments}
In this section I am going to describe what is the setup of the experiment I have prepared and run and what assumptions I am doing.

\subsection{Insert without Rehashing - Average time}
In order to calculate the average time of insertion without rehashing, I am measuring the time of the successful inserts, and only those that after insert the \mintinline{haskell}{rehashed} function returns \mintinline{haskell}{False}.

\begin{listing}[H]
\begin{haskellcode*}{}
insertWithoutRehash' :: Int -> IO [Integer]
insertWithoutRehash' n = do
  idx <- getPositive <$> generate (arbitrary @(Positive Int))
  list <- generate $ shuffle [idx `.`. idx + n]
  let (test, train) = splitAt (n '`div'` 2) list
  fmap catMaybes <$> return $
    runST $ do
      hash <- create
      mapM_ (C.insert hash) $ test
      forM train $ \e -> do
        start <- toNano <$> return (unsafePerformIO getCurrentTime)
        _ <- C.insert hash e
        rehash <- C.rehashed hash
        end <- toNano <$> return (unsafePerformIO getCurrentTime)
        return $ unsafePerformIO $ print start
        let diff = (end - start)
        return $
          if rehash
            then Nothing
            else Just diff
\end{haskellcode*}
\caption{Insert without Rehashing Function}
\label{lst:insertWithoutRehash}
\end{listing}

\appendix\label{appendix}
\section{Haskell Code}
\subsection{Source Code}
In the source code there are 2 folders with code:

\begin{itemize}
  \item \textbf{app}: \mintinline{haskell}{Main.hs} which contains the main entry point of the program that run all the experiment
  \item \textbf{src}: \mintinline{haskell}{Experiments.hs} contains the experiments and \mintinline{haskell}{Data/Hash/Cukoo.hs} that contains the implementation of \acrshort{cuk}
\end{itemize}

\subsection{Run the Code}
All the solution has been coded with \textbf{Stack} \cite{stack} version 2.1.3 or higher. It is a prerequisite to install \textit{stack} for running this code.

\subsubsection{Running Experiments}

In order to run the experiments just do the following:

\begin{lstlisting}[language=Haskell,title={Running Experiments}]
shell> stack build
shell> stack exec cukoo
\end{lstlisting}

\section{Generated Reports}
All the Generated Reports that has been used to build this document are under \mintinline{haskell}{output} folder.


\section{Report PDF Document}
This report document is under \mintinline{haskell}{doc} folder alongside images that are embedded in this report.


\bibliographystyle{alpha}
\bibliography{report}

\printglossary[type=\acronymtype]

\end{document}

