\documentclass[12pt, a4paper]{article}
\usepackage[utf8]{inputenc}
\usepackage{amsmath}
\usepackage{amsthm}
\usepackage{amssymb}
\usepackage{graphicx}
\usepackage{parskip}
\usepackage{hyperref}
\usepackage{fancyhdr}
\usepackage{lastpage}
\usepackage[vlined,ruled]{algorithm2e}
\usepackage[acronym]{glossaries}
\usepackage{caption}
\usepackage{titlesec}
\usepackage{multirow}


\title{%
  Algorithmic Game Theory \\
  Homework 4
}
\author{%
  Juan Pablo Royo Sales\\
  \small{Universitat Politècnica de Catalunya}
}
\date\today

\pagestyle{fancy}
\fancyhf{}
\fancyhead[C]{}
\fancyhead[R]{Juan Pablo Royo Sales - UPC MIRI}
\fancyhead[L]{AGT - Homework 4}
\fancyfoot[L,C]{}
\fancyfoot[R]{Page \thepage{} of \pageref{LastPage}}
\setlength{\headheight}{15pt}
\renewcommand{\headrulewidth}{0.4pt}
\renewcommand{\footrulewidth}{0.4pt}

\renewcommand{\qedsymbol}{$\blacksquare$}

\begin{document}

\maketitle

\section{Problem 4}
\subsection{(a) Valuation Function}
Lets define a variable $x_i$ in which 
\[
x_{ij} = 
\begin{cases}
  1 \text{ if a copy of album } j \text{ is provided by player } i\\
  0 \text{ otherwise } 
\end{cases}
\]

Therefore, 

\begin{equation*}
  v(C) = \Biggr|\biggr\{j \mid \sum_{i \in C} x_{ij} \geq k \biggl\}\Biggl|\\
\end{equation*}

\subsection{(b) Is the game convex?}
\textbf{Yes, it is convex}.

A way to probe this is this is establishing is the game is \textbf{super modular}, because 
we know that if the game is \textbf{super modular} then it is \textbf{convex}.

A game is \textbf{super modular} if $v(C \cup D) + v(C \cap D) \geq v(C) + v(D)$.

Lets take $2$ partitions $C$ and $D$ and analyze if this is true:

\begin{itemize}
  \item By the valuation function we know that $v(C) \geq 0 \land v(D) \geq 0$
  \item Lets suppose that partition $v(C) = 0$ and $v(D) > 0$ then $v(C \cup D) > 0$ and $V(C \cap D) = 0$, therefore $v(C \cup D) + v(C \cap D) \geq v(C) + v(D)$
  \item Lets suppose that both partitions are greater than 0 $v(C) > 0$ and $v(D) > 0$ then $v(C \cup D) > 0$ and $V(C \cap D) \geq 0$, therefore $v(C \cup D) + v(C \cap D) \geq v(C) + v(D)$ 
\end{itemize}

\subsection{(c) Shapley values}
An accurate representation could be:

\begin{equation}
  \Phi(\Gamma) = \frac{1}{n!}\Biggr|\biggr\{j \mid \sum_{\pi \in \Pi(n)} x_{(\pi)j} \geq k \biggl\}\Biggl|
\end{equation}

It \textbf{cannot be computed in Polynomial} time because there are $n!$ number of ways to combine the configurations for which
we can have different Shapley values depending on the order on how the players want to contribute to the game.

\end{document}

