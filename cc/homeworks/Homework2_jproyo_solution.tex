\documentclass[12pt, a4paper]{article}
\usepackage[utf8]{inputenc}
\usepackage{amsmath}
\usepackage{amsthm}
\usepackage{graphicx}
\usepackage{parskip}
\usepackage{hyperref}
\usepackage{fancyhdr}
\usepackage{lastpage}
\usepackage[vlined,ruled]{algorithm2e}
\usepackage[acronym]{glossaries}
\usepackage{caption}

\title{%
  Computational Complexity \\
  Homework 2 - Solutions
}
\author{%
  Juan Pablo Royo Sales\\
  \small{Universitat Politècnica de Catalunya}
}
\date\today

\pagestyle{fancy}
\fancyhf{}
\fancyhead[C]{}
\fancyhead[R]{Juan Pablo Royo Sales - UPC MIRI}
\fancyhead[L]{CC - Homework 2}
\fancyfoot[L,C]{}
\fancyfoot[R]{Page \thepage{} of \pageref{LastPage}}
\setlength{\headheight}{15pt}
\renewcommand{\headrulewidth}{0.4pt}
\renewcommand{\footrulewidth}{0.4pt}

\makeglossaries

\newacronym{kcol}{$k$-COL}{$k$-coloring problem}
\newacronym{scol}{SEARCH-$k$-COL}{Search $k$-coloring problem}
\newacronym{2col}{$2$-COL}{$2$-coloring problem}
\newacronym{bipart}{BIPARTITE}{Bipartite Graph}
\newacronym{dreach}{DREACH}{Direct Reachability problem}
\newacronym{immsze}{IMME-SZE}{Immerman-Szelepcsényi Theorem}
\newacronym{nbipart}{$\overline{BIPARTITE}$}{Not Bipartite Graph}

\begin{document}

\maketitle

\section{Self-reducibility}\label{ex:1}

In order to prove that if \acrfull{kcol} decision problem can be solved in polynomial time, then it search version \acrfull{scol} can be also solved in polynomial time, we are going to be based on the following theorem:

\newtheorem{searchpnp}{Theorem}
\begin{searchpnp}
  If $P = NP$, then for every decision problem in $A \in NP$ the search version of $A$ can be solved in polynomial time.
\end{searchpnp}

For that purpose we need to find a Prefix version of \acrshort{kcol} that can at the end certificate \acrshort{kcol}\@. In other words, given a $x$ and $z$, is there a $y$, such that $|zy| \leq |x|^c\ \land \langle x,zy \rangle \in V$?
By the following lemma we know that:

\newtheorem{lemmasearch}{Lemma}
\begin{lemmasearch}
  If the Prefix version of $A$ is in $P$ then the search version can be solved in polynomial time.
\end{lemmasearch}

Given the \acrshort{kcol} problem we can construct the following algorithm as the Prefix version of \acrshort{kcol}.

\begin{algorithm}[H]
  \SetKwInOut{Input}{Input}
  \SetKwInOut{Output}{Output}
  \Input{Given $G = (V, E)$}
  \Output{$a_1,\dots,a_{|V|}$}
  \nlset{$|V|$ times}\For(\tcp*[f]{For each vertex}){$i \leftarrow 1$ \KwTo $|V|$}
  {\nlset{$k$ times}\For(\tcp*[f]{For each color}){$j \leftarrow 1$ \KwTo $k$}
    {$a_i \leftarrow j$; \\
      \nlset{polytime}Check if $v_1 = a_1, \dots, v_{i-1} = a_{i-1}, v_i = a_i$ is valid assignment coloring applying the decision \acrshort{kcol};\label{polytime}
    }\eIf{None $j$ was valid}
    {\emph{Halt and return NOT-COLORING}
    }
    {$a_i \leftarrow j$; \tcp{With the $j$ that was valid}}
  }
  return $a_1,\dots,a_{|V|}$;
  \caption{Prefix version of \acrshort{kcol}}
\end{algorithm}

As we can see in the algorithm for the Prefix \acrshort{kcol} this part~\ref{polytime} is polynomial time by definition of the problem, so that part cannot take more than $O(n^c)$. Having that the total cost of the algorithm is $O(|V| \times k \times n^c)$. Therefore it is polynomial time.

\section{Zero-error}
In order to show that given a \acrshort{kcol} in $BPP$ we can construct a $NTM\ M'$ which decide that \acrshort{kcol} in $ZPP$ we are going to assume that $P = NP$, because we know that if $NP = P \implies BPP = P$.

Having this assumption stated, we are going to work with the definition of the Excercise 1~\ref{ex:1}.

\begin{itemize}
  \item Lets $M$ be the $BPP$ that decide \acrshort{kcol}.
  \item Lets $M'$ be new machine that is going decide \acrshort{kcol} using \acrshort{scol} from~\ref{ex:1}
\end{itemize}

What we are going to do is to run $M(x)$ which we know it is a $BPP$. Since we cannot have errors, we are going to run \acrshort{scol} to find a valid assignment for that \acrshort{kcol}. We can do this because we are assuming that $NP = P$ and under that assumption $BPP = P$. So our $BPP$ for \acrshort{kcol} applies to problem in~\ref{ex:1}. So we have a \acrshort{scol} in polynomial time that can find a valid assignment for that \acrshort{kcol}. If the valid assignment coincides with the same output as $M$ we are sure if $x \in L \lor x \notin L$.

Having that, if the output of $M(x) = 0$ and $x \in L$, and we can know that because of running $M'$ \acrshort{scol}, since we cannot be sure about the error so with mark as \textbf{?}. By symmetry if $x \notin L \land M(x) = 1$, we are going to mark as \textbf{?}. So the other 2 cases we are going to have a $x \in L \implies Pr[M(x) = 1] \geq 3/4$ because of $BPP$ on $M$ and if $x \notin L \implies Pr[M(x) = 0] \geq 3/4$ also because of $BPP$

Therefore our $M'$ is going to have the following probabilities:

\begin{subequations}
  \begin{align}
  x \in L \implies Pr[M'(x)=0] = 0 \land Pr[M'(x) = 1] \geq 1/2\\
  x \in L \implies Pr[M'(x)=1] = 0 \land Pr[M'(x) = 0] \geq 1/2\\
\end{align}
\end{subequations}

There is no way to put an error because those output that we dont know we have marked as \textbf{?}, therefore $M' \in ZPP$.


\section{Logarithmic Space}
In order to show that \acrfull{2col} is in $P$ and even in $NL$ we are going to use the fact that any \acrshort{2col} is a \acrfull{bipart} with not contains \textbf{cycles} with an \textbf{odd number} of nodes. If we can give a \textbf{polynomial time} algorithm for constructing a \acrshort{bipart} which also use \textbf{logarithmic space}, we can have an $TM\ M$ that decides \acrshort{2col} in $P$ and $NL$.

In order to do so, we are going to give an algorithm that shows if a Graph $G=(V,E)$ is \acrfull{nbipart}, using \acrfull{dreach} algorithm that we know it is in $NL$.

\begin{algorithm}[H]
  \SetKwInOut{Input}{Input}
  \SetKwInOut{Output}{Output}
  \Input{Given $\langle G,s,t \rangle$}
  \Output{$1$ if there is a connected path from $s$ to $t$ with odd number of edges, $0$ otherwise}
  \Begin{$n \leftarrow |V|$\;
  \nlset{$|V|$ times}\For{$v \in V$}
  {$c \leftarrow v$\;
   $i \leftarrow 1$\;
   \While{$c \neq t \land i \leq n$}
   {\nlset{$\log(n)$ space}\emph{Guess $u \in V$}\;
     \nlset{$\log(n)$ space}\If{$(c,u) \in E$}
     {\nlset{$\log(n)$ space}$c \leftarrow u$\;
       \nlset{$\log(n)$ space}$i \leftarrow i + 1$\;
      \If{$u = t \land i \text{ is odd}$}
      {return 1\;}
     }
   }
  }
  return 0\;
  }
  \caption{Algorithm \acrshort{dreach} for counting cycle with odd number of edges}\label{algo:2}
\end{algorithm}



\begin{algorithm}[H]
  \SetKwInOut{Input}{Input}
  \SetKwInOut{Output}{Output}
  \Input{Given $G = (V, E)$}
  \Output{$1$ if $G \notin$ \acrshort{2col}, $0$ otherwise}
  \Begin
  {\nlset{$|V|$ times}\For(\tcp*[f]{For each vertex}){$v \in V$}
  {$r \leftarrow \text{Call algo~\ref{algo:2} with} \langle G,v,v \rangle$\;
   \If{$r = 1$}
   {\emph{Halt and Output 1}\tcp*[f]{\acrshort{nbipart}}}
  }
  return 0\tcp*[f]{\acrshort{bipart}}
  }
  \caption{Algorithm for checking if $G \notin$ \acrshort{2col}}\label{algo:3}
\end{algorithm}

As we can see our algorithm is in $P$ because it use at most $O(|V|^3)$ between the 2 for loops plus the Guessing for the next vertex in the algo~\ref{algo:2}. As we can also it is in $NL$ since it use $\log(n)$ space.

But this algorithm tell us if our $G$ is not in \acrshort{2col}, and for doing this we are using a \acrshort{nbipart} algorithm, so we can say that this algorithm is in $co$-NL\@. By \acrfull{immsze} we now that $co$-NL $= NL$, therefore to determine if $G$ is in \acrshort{2col} algorithm is in $NL$.



\end{document}

