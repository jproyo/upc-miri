\documentclass{beamer}
\usetheme{Boadilla}
\usepackage[T1]{fontenc}
\usepackage[utf8]{inputenc}
\usepackage{amsmath}
\usepackage{amsfonts}
\usepackage{amsthm}
\usepackage{array}
\usepackage{graphicx}
\usepackage{listings}
\usepackage{color}
\usepackage{fancyhdr}
\usepackage{tikz}
\usetikzlibrary{calc,shapes.multipart,chains,arrows}
\usepackage{float}
\usepackage[vlined,ruled]{algorithm2e}
\usepackage{caption}
\usepackage[linguistics]{forest}
\usepackage[cache=false]{minted}
\usemintedstyle{default}
\newminted{haskell}{}
\graphicspath{{./images/}}
\setbeamertemplate{frametitle}{%
  \usebeamerfont{frametitle}\insertframetitle%
  \vphantom{g}% To avoid fluctuations per frame
  % \hrule% Uncomment to see desired effect, without a full-width hrule
  \par\hspace*{-\dimexpr0.5\paperwidth-0.5\textwidth}\rule[0.5\baselineskip]{\paperwidth}{0.4pt}
  \par\vspace*{-\baselineskip}% <-- reduce vertical space after rule
}
\setbeamertemplate{footline}
{
  \leavevmode%
  \hbox{%
  \begin{beamercolorbox}[wd=.333333\paperwidth,ht=2.25ex,dp=1ex,center]{author in head/foot}%
    \usebeamerfont{author in head/foot}\insertauthor
  \end{beamercolorbox}%
  \begin{beamercolorbox}[wd=.333333\paperwidth,ht=2.25ex,dp=1ex,center]{title in head/foot}%
    \usebeamerfont{title in head/foot}\insertshortsubtitle
  \end{beamercolorbox}%
  \begin{beamercolorbox}[wd=.333333\paperwidth,ht=2.25ex,dp=1ex,right]{date in head/foot}%
    \usebeamerfont{date in head/foot}\insertshortdate{}\hspace*{2em}
    \insertframenumber{} / \inserttotalframenumber\hspace*{2ex} 
  \end{beamercolorbox}}%
  \vskip0pt%
}

\DeclareMathOperator*{\st}{st}

\title{SAT Based Scheduling in High Level Synthesis}
\subtitle{RTL PB-SAT Optimizer}

\author{Juan Pablo Royo Sales}
\institute{Universitat Politècnica de Catalunya}
\date{January, 2021}
\begin{document}

\begin{frame}
\titlepage
\end{frame}

\begin{frame}{Agenda}
  \tableofcontents
\end{frame}

\begin{frame}{Agenda}
  \section{Introduction}
  \tableofcontents[currentsection]
\end{frame}

\begin{frame}[fragile]{Introduction}
  \begin{block}{Paper}
      \begin{itemize}
        \item Generation of \textbf{RTL (Register Transfer Level)}: Optimization Process of Generating \textbf{High Level Synthesis} 
    \end{itemize}
  \end{block}
\end{frame}

\begin{frame}[fragile]{Introduction}
  \begin{block}{Paper}
      \begin{itemize}
        \item Generation of \textbf{RTL (Register Transfer Level)}: Optimization Process of Generating \textbf{High Level Synthesis} 
        \item Paper: \textit{SAT-Based High Level Synthesis}\footnote{Kundu S., Chandrakar K., Roy S. (2014) SAT Based Scheduling in High Level Synthesis. In: Kumar Kundu M., Mohapatra D., Konar A., Chakraborty A. (eds) Advanced Computing, Networking and Informatics- Volume 2. Smart Innovation, Systems and Technologies, vol 28. Springer, Cham.}
    \end{itemize}
  \end{block}
\end{frame}


\begin{frame}{Agenda}
  \section{Paper Goal: Optimize RTL}
  \tableofcontents[currentsection]
\end{frame}


\begin{frame}[fragile]{Paper Goal: Optimize RTL}
  \begin{block}{2 Scheduling Optimization}
    \begin{itemize}
      \item \textbf{Time-Constraint Scheduling}: \textit{Given constraints on the maximum number
      of control steps, find the cost-efficient schedule which satisfies the constraints}
    \end{itemize}
  \end{block}
\end{frame}

\begin{frame}[fragile]{Paper Goal: Optimize RTL}
  \begin{block}{2 Scheduling Optimization}
    \begin{itemize}
      \item \textbf{Time-Constraint Scheduling}: \textit{Given constraints on the maximum number
      of control steps, find the cost-efficient schedule which satisfies the constraints}
      \item \textbf{Resource-Constraint Scheduling}: \textit{Given constraints on the resources,
      find the fastest scheduling which satisfies the constraints.}
    \end{itemize}
  \end{block}
\end{frame}

\begin{frame}[fragile]{Paper Goal: Optimize RTL}
  \begin{block}{2 Scheduling Optimization}
    \begin{itemize}
      \item \textbf{Time-Constraint Scheduling}: \textit{Given constraints on the maximum number
      of control steps, find the cost-efficient schedule which satisfies the constraints}
      \item \textbf{Resource-Constraint Scheduling}: \textit{Given constraints on the resources,
      find the fastest scheduling which satisfies the constraints.}
    \end{itemize}
  \end{block}

  \begin{block}{Proposed Approach}
    \centering\textbf{Pseudo-Boolean SAT Optimizer}
  \end{block}
\end{frame}

\begin{frame}{Agenda}
  \section{PB-SAT Encoding}
  \tableofcontents[currentsection]
\end{frame}

\begin{frame}[fragile]{PB-SAT Encoding}
 \begin{block}{Time-Constraint Scheduling}
  \begin{subequations}
    \begin{align}
      \min \quad & \sum_{k=1}^m \biggl(c_k \times \biggl(\sum_{b=0}^{r-1}2^b \times M_k\biggr)\biggr)\\
      \st  \quad & \\
      \quad & \sum_{j=S_i}^{L_i} x_{i,j} = 1 & \quad & \forall 1 \leq i \leq N \\\label{c:1}
      \quad & \sum_{j=S_i}^{L_i} (j \times x_{i,j}) - \sum_{j=S_k}^{L_k} (j \times x_{k,j}) \leq -1 \\\label{c:2}
      \quad & \sum_{j=o_i \in FU_k} x_{i,j} - \sum_{b=0}^{r-1}(2^b \times M_k) \leq 0 & \quad & \forall 1 \leq j \leq s\\\label{c:3}
    \end{align}
    \end{subequations}
  \end{block}
\end{frame}

\begin{frame}[fragile]{PB-SAT Encoding}
  \begin{block}{Resource-Constraint Scheduling}
   \begin{subequations}
     \begin{align}
       \min \quad & \sum_{b=0}^{r-1}2^b \times C_{\text{step},b}\\
       \st  \quad & \\
       \quad & \sum_{j=S_i}^{L_i} x_{i,j} = 1 & \quad & \forall 1 \leq i \leq N \\\label{c:4}
       \quad & \sum_{j=S_i}^{L_i} (j \times x_{i,j}) - \sum_{j=S_k}^{L_k} (j \times x_{k,j}) \leq -1 \\\label{c:5}
       \quad & \sum_{j=o_i \in FU_k} x_{i,j} - M_k \leq 0 & \quad & \forall 1 \leq j \leq s\\\label{c:6}
       \quad & \sum_{j=S_i}^{L_i} x_{i,j} - \sum_{b=0}^{r-1}2^b \times C_{\text{step},b} \leq 0 & \quad & \forall o_i \text{ without successor} \\\label{c:7}
     \end{align}
     \end{subequations}
   \end{block}
 \end{frame}

 \begin{frame}{Agenda}
  \section{Proposed Algorithm}
  \tableofcontents[currentsection]
\end{frame}

\begin{frame}[fragile]{Proposed Algorithm}
  \begin{algorithm}[H]
    \SetKwInOut{Input}{Input}
    \SetKwInOut{Output}{Output}
    \Input{$G = (V,E)$ with ASAP and ALAP Schedule}
    \Output{$T \land R$ New Graphs with Time and Resource Schdule Optimized}
    $G' \leftarrow \textbf{ Determine mobility of each node of the Graph}$\\
    $T_M \leftarrow \text{ Build PB-SAT Time schedule on } G'$\\
    $R_M \leftarrow \text{ Build PB-SAT Resource schedule on } G'$\\
    $T_{SAT} \leftarrow PB-SAT(T_M)$\\
    $R_{SAT} \leftarrow PB-SAT(T_M)$\\
    \eIf{$T_{SAT} \neq UNSAT$}
    {$T \leftarrow \text{Build result on } T_{SAT}$}
    {Output $UNSAT$ for Time}
    \eIf{$R_{SAT} \neq UNSAT$}
    {$R \leftarrow \text{Build result on } R_{SAT}$}
    {Output $UNSAT$ for Resource}
    return $T$ and $R$
   \caption{RTL PB-SAT Optimizer}
  \end{algorithm}
\end{frame}
 
\begin{frame}{Agenda}
  \section{My Contribution}
  \tableofcontents[currentsection]
\end{frame}

\begin{frame}[fragile]{My Contribution}
  \begin{block}{}
    \begin{itemize}
      \item \textbf{Define an input format}: Input wasn't well defined in Author's word
    \end{itemize}
  \end{block}
\end{frame}

\begin{frame}[fragile]{My Contribution}
  \begin{block}{}
    \begin{itemize}
      \item \textbf{Define an input format}: Input wasn't well defined in Author's word
      \item \textbf{Haskell}: Using Haskell based PB-SAT Solver with an \textbf{FP} Encoding approach
    \end{itemize}
  \end{block}
\end{frame}

\begin{frame}[fragile]{My Contribution}
  \begin{block}{}
    \begin{itemize}
      \item \textbf{Define an input format}: Input wasn't well defined in Author's word
      \item \textbf{Haskell}: Using Haskell based PB-SAT Solver with an \textbf{FP} Encoding approach
      \item \textbf{Output}: Graphical Output Representation of the Results in PNG format
    \end{itemize}
  \end{block}
\end{frame}

\begin{frame}{Agenda}
  \section{Input Format - SCH Format}
  \tableofcontents[currentsection]
\end{frame}

\begin{frame}[fragile]{Input Format - SCH Format}
    \begin{minted}[fontsize=\tiny,highlightlines={3-5,15-17,27-29}]{text}
      # This file has the following format
      # TYPE_RESOURCE NODE_ID FROM_STEP TO_STEP CONNECTED_TO_NODE_ID
      ASAP
      M 1 1 1 [2]
      M 2 2 2 [3]
      S 3 3 3 [4]
      S 4 4 4
      M 5 1 1 [2]
      M 6 1 1 [7]
      M 7 2 3 [4]
      M 8 1 1 [9]
      A 9 2 2
      A 10 1 1 [11]
      C 11 2 2
      ALAP
      M 1 1 1 [2]
      M 2 2 2 [3]
      S 3 3 3 [4]
      S 4 4 4
      M 5 1 1 [2]
      M 6 2 2 [7]
      M 7 3 3 [4]
      M 8 3 3 [9]
      A 9 4 4
      A 10 3 3 [11]
      C 11 4 4
      RESOURCES
      M 2 3
      A 1 3
      S 1 3
      C 1 3
     \end{minted}  
\end{frame}

\begin{frame}
  \frametitle{Agenda}
  \section{Haskell Main Algorithm Implementation}
  \tableofcontents[currentsection]
\end{frame}

\begin{frame}[fragile]{Haskell Main Algorithm Implementation}

  \begin{minted}[fontsize=\tiny,highlightlines={3-3,9-11,17}]{haskell}
    solve :: RTLConf -> IO ()
    solve RTLConf{..} = do
      parse <- parseFromFileEx parseSchedule _rtlcFileInput
      case parse of
        Failure ex -> putText $ show ex
        Success sc -> do
          createDirectoryIfMissing True _rtlcFolderOutput
          dumpInputSchedules _rtlcFolderOutput sc
          forM_ 
            [encodeTimeSchedule, encodeResourceSchedule]
            (solve' sc _rtlcFolderOutput)
    
    solve' :: Schedule -> FilePath -> (Schedule -> ResultEncoder) -> IO ()
    solve' sc output fSolve = do
      solver <- SAT.newSolver
      let rEncoder = fSolve sc
      result <- fmap (toSchedule sc (rEncoder^.reEncode)) <$> S.solvePB solver (rEncoder^.reFormula)
      maybe 
        (putText $ "NO SOLUTION FOUND FOR: " <> show (rEncoder^.reType) <> " Schedule") 
        (\r -> putText (show r) >> dumpResult output (rEncoder^.reType) r)
        result
  \end{minted}

\end{frame}

\begin{frame}
  \frametitle{Agenda}
  \section{Caveats and Analysis}
  \tableofcontents[currentsection]
\end{frame}

\begin{frame}[fragile]{Caveats and Analysis}
  \begin{block}{About Author's work}
    \begin{itemize}
      \item Although the work the author's presented are promising they are only focusing in the speed up of solver and not in doing complex examples
      \item I have tried to reproduce their results even with \textbf{Minisat+} which is the solver they use and results are different on what they presented, but not in the Optimum just on the order.
      \item Author's don't provide big examples and all their experiments are based on tiny examples.
  \end{itemize}
  \end{block}
\end{frame}

\begin{frame}
  \frametitle{Agenda}
  \section{Conclusion}
  \tableofcontents[currentsection]
\end{frame}

\begin{frame}[fragile]{Conclusion}
  \begin{block}{About Author's work}
    \begin{itemize}
      \item Very Fast Optimization technique
      \item Simple to Encode in PB-SAT 
  \end{itemize}
  \end{block}
\end{frame}

\begin{frame}
  \begin{center}
    \textbf{\huge{Demo Time}}
    \end{center}
\end{frame}


\begin{frame}
  \begin{center}
    \textbf{\huge{Thank you!!}}
    \end{center}
\end{frame}

\end{document}

