\documentclass[12pt, a4paper]{article}
\usepackage[utf8]{inputenc}
\usepackage{amsmath}
\usepackage{amsthm}
\usepackage{amssymb}
\usepackage{graphicx}
\usepackage{parskip}
\usepackage{hyperref}
\usepackage{fancyhdr}
\usepackage{lastpage}
\usepackage[vlined,ruled]{algorithm2e}
\usepackage[acronym]{glossaries}
\usepackage{caption}
\usepackage{titlesec}
  

\title{%
  Algorithmic Game Theory \\
  Homework 5 - Cooperative Games
}
\author{%
  Juan Pablo Royo Sales\\
  \small{Universitat Politècnica de Catalunya}
}
\date\today

\pagestyle{fancy}
\fancyhf{}
\fancyhead[C]{}
\fancyhead[R]{Juan Pablo Royo Sales - UPC MIRI}
\fancyhead[L]{AGT - Homework 5}
\fancyfoot[L,C]{}
\fancyfoot[R]{Page \thepage{} of \pageref{LastPage}}
\setlength{\headheight}{15pt}
\renewcommand{\headrulewidth}{0.4pt}
\renewcommand{\footrulewidth}{0.4pt}

\renewcommand{\qedsymbol}{$\blacksquare$}
\newacronym{bpp}{BPP}{BPP}

\begin{document}

\maketitle

\section{Problem 11}

\subsection{Simple Game Analysis}
Lets analyze case by case to see if some of the option $(a)$, $(b)$ or $(c)$ is a $\Gamma_s$ \textbf{Simple Game}.

\newtheorem{case_a}{$(a)$}
\newtheorem{proofpart}{Part}
\begin{case_a}
  A Coalition $X$ is winning in $\Gamma_s \iff X$ wins in $\Gamma$ and $H[X]$ has not isolated vertices.
\end{case_a}

\begin{proof}
\begin{proofpart}
$\implies$\\
This part is trivial by \textbf{monotonicity} of Simple Games since if it is wining on $\Gamma_s$ is winning on $\Gamma$ and
there should be not isolated vertices since all the players should communicate.
\end{proofpart}

\begin{proofpart}
$\impliedby$\\
Lets assume there is no isolated vertices. But if that occurs it doesn't mean that $H[X]$ is a connected graph, because
for example if $(u,v) \in E$ and $(v,z) \in E$ but $(u,z) \notin E$, the graph has not isolated vertices but it is not connected.
\end{proofpart}

Therefore, $(a)$ \textbf{is not a simple game} on $\Gamma_s$.
\end{proof}

\newtheorem{case_b}{$(b)$}
\newtheorem{proofpart_b}{Part}
\begin{case_b}
  A Coalition $X$ is winning in $\Gamma_s \iff X$ wins in $\Gamma$ and $H[X]$ is connected.
\end{case_b}

\begin{proof}
\begin{proofpart}
$\implies$\\
This part is trivial by the same reason of previous analysis.
\end{proofpart}

\begin{proofpart}
$\impliedby$\\
If $H[X]$ is connected means that $\forall \{u,v\} \in N, (u,v) \in E$. So every member communicate each other.
At the same time $X$ is winning in $\Gamma$, therefore $X$ is also winning in $\Gamma_s$ because both condition holds.
\end{proofpart}

Therefore, $(b)$ \textbf{is a simple game} on $\Gamma_s$.
\end{proof}

\newtheorem{case_c}{$(c)$}
\newtheorem{proofpart_c}{Part}
\begin{case_c}
  A Coalition $X$ is winning in $\Gamma_s \iff$ there is an $Y \subseteq X$, so that $Y$ wins in $\Gamma$ and $H[Y]$ is connected.
\end{case_c}

\begin{proof}
\begin{proofpart}
$\implies$\\
Suppose that $X$ belongs to a minimal wining coalition $W^m$. If that happens there cannot be any other winning coalition that is included in $X$, 
because $\forall Y \in W, Y \subsetneq X$ according to the definition of minimal winning coalition.
\end{proofpart}

Therefore, $(c)$ \textbf{is a not simple game} on $\Gamma_s$.
\end{proof}

\subsection{Complexity Analysis}
For the complexity analysis and taking into consideration that $X \in W$ can be decide in $Poly-time$, i could provide the following Polynomial time algorithm for deciding \textit{empty-core} in $\Gamma_s$. In this case for $(b)$,
based on the Theorem in which we know that \textit{A Simple Game has a non-empty core if it has a veto player}.
The idea of the algorithm is trying to find that \textbf{veto player}

\begin{algorithm}[H]
  \SetKwInOut{Input}{Input}
  \SetKwInOut{Output}{Output}
  \Input{Given $H = (N, E)$}
  \Output{If $\Gamma_s$ has an empty core}
  \nlset{$O(|N|+|E|)$ times}{$SCC \leftarrow $ Calculate \textbf{SCC} of $H$}\\
  \nlset{$n$ times}\For{each $X \in SCC$}{
    \nlset{$k$ times}\For{each $p \in X$}{
      $X \leftarrow X \setminus \{p\}$;\\
      \nlset{$Poly-time$}\lIf{$X \notin W$}
      {return \emph{NON EMPTY CORE};
      }
    }
  }
  return \emph{EMPTY CORE};
  \caption{Decide $\Gamma_s$ has an empty core}
\end{algorithm}

\end{document}

