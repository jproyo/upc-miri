\documentclass[12pt, a4paper]{article}
\usepackage[utf8]{inputenc}
\usepackage{amsmath}
\usepackage{amsfonts}
\usepackage{amsthm}
\usepackage{array}
\usepackage{graphicx}
\usepackage{parskip}
\usepackage[pdfencoding=auto]{hyperref}
\usepackage{fancyhdr}
\usepackage{lastpage}
\usepackage{tikz}
\usepackage{float}
\usepackage{listings}
\usepackage{color}
\usepackage{caption}
\usepackage{authblk}
\usepackage[acronym]{glossaries}
\usepackage[nottoc]{tocbibind}
\usepackage[cache=false]{minted}
\usemintedstyle{default}
\newminted{haskell}{frame=lines,framerule=2pt}
\newminted{R}{frame=lines,framerule=2pt}
\newminted{cpp}{frame=lines,framerule=2pt}
\graphicspath{{./images/}}

\tikzstyle{bag} = [align=center]

\title{%
      Homework 3\\
      Significance of Metrics\\
}
\author{Juan Pablo Royo Sales}
\affil{Universitat Politècnica de Catalunya}
\date\today

\pagestyle{fancy}
\fancyhf{}
\fancyhead[C]{}
\fancyhead[R]{UPC MIRI}
\fancyhead[L]{CSN - Homework 3}
\fancyfoot[L,C]{}
\fancyfoot[R]{Page \thepage{} of \pageref{LastPage}}
\setlength{\headheight}{15pt}
\renewcommand{\headrulewidth}{0.4pt}
\renewcommand{\footrulewidth}{0.4pt}

\newacronym{lang}{Language Set}{Provided Language Set}
\newacronym{switching}{Switching Model}{Null Model based on Switching}
\newacronym{binomial}{Binomial Model}{Null Model based on Binomial Erdos Renyi}
\newacronym{cc}{Closeness Centratlity}{Closeness Centratlity Measure}
\newacronym{nh}{Null Hypothesis}{Null Hypothesis}
\newacronym{mc}{Montecarlo}{Montecarlo Approximation Algorithm}

\begin{document}

\maketitle

\tableofcontents

\section{Introduction}
In this work I have been analyzing the significance of \acrfull{cc} metric over a \acrfull{lang} provided for such case.
Basically the idea of the work is the following: Giving this set of Languages with some set of connected terms, determine 
the \acrshort{cc} accurately measured. Hypothesis

After that we want to measure the Significance of that metric against different \acrshort{nh}, in particular against \acrfull{switching} and \acrfull{binomial}.

Using \acrfull{mc} 

\begin{itemize}
    \item \textbf{Results} section: We are going to show the results obtained for each language according to the previous statements.
    \item \textbf{Discussion} section: In this section we are going to conclude and give some explanaition based on the results.
    \item \textbf{Methods}
    \item \textbf{Conclusions} section: Finally we are going to give our opinion about the results obtained and what we have learnt.
\end{itemize}

\section{Results}
Lets divide the results for each language, but first lets see the Summary of the data that we are analysing.

\subsection{Distributions}

\begin{table}[H]
    \centering
    \begin{tabular}{l l l l l}
    Language & N & E & $<k>$ & $\delta$ \\
     \hline
    Arabic & 21532 & 68743 & 6.3851941 & 0.0002966 \\
    Basque & 12207 & 25541 & 4.1846482 & 0.0003428 \\
    Catalan & 36865 & 197075 & 10.6917130 & 0.0002900 \\
    Chinese & 40298 & 180925 & 8.9793538 & 0.0002228 \\
    Czech & 69303 & 257254 & 7.4240365 & 0.0001071 \\
    English & 29634 & 193078 & 13.0308430 & 0.0004397 \\
    Greek & 13283 & 43961 & 6.6191372 & 0.0004984 \\
    Hungarian & 36126 & 106681 & 5.9060510 & 0.0001635 \\
    Italian & 14726 & 55954 & 7.5993481 & 0.0005161 \\
    Turkish & 20409 & 45625 & 4.4710667 & 0.0002191 \\
    \end{tabular}
   \caption{Language List - Paramters}
   \label{table:1}
\end{table}

\begin{table}[H]
  \centering
  \begin{tabular}{l l l l}
  Language & Metric & p-value (binomial) & p-value (switching)\\
   \hline
   Arabic & 0.3264580  & & \\
   Basque & 0.2697194  & & \\
   Catalan & 0.3410102 & & \\
   Chinese & 0.3264540 & & \\
   Czech & 0.3059482 & & \\
   English & 0.3435094  & & \\
   Greek & 0.3147174  & & \\
   Hungarian & 0.2883464 & & \\
   Italian & 0.3278201 & & \\
   Turkish & 0.3603335  & & \\
     \end{tabular}
 \caption{Language List - Paramters}
 \label{table:2}
\end{table}


\section{Discussion}
\section{Methods}

\subsubsection{Displaced Poisson and Geometric}
On those cases we have done the regular setup but taking into consideration that we are calculating $C$ value from the sample.
\begin{listing}[H]
    \inputminted[firstline=11, lastline=19, breaklines]{R}{./Solution.R}
    \caption{Extracted from source Solution.R}
    \label{apx:src:4}
\end{listing}  
\begin{listing}[H]
    \inputminted[firstline=86, lastline=95, breaklines]{R}{./Solution.R}
    \caption{Extracted from source Solution.R}
    \label{apx:src:5}
\end{listing}  


\section{Conclusions}

\printglossary[type=\acronymtype]

\appendix\label{apx:org}
\section{Source Code and Assets}
The source code and different assets that are contained in these folders are classified in the following way.

\begin{itemize}
    \item \textbf{bin} This folder contains the compiled binary files for the code that implements the requirements suggested by the lab work.
    \item \textbf{data} This folder contains the input \acrshort{lang} provided.
    \item \textbf{docs} This folder contains this report in Latex and PDF compiled format.
    \item \textbf{out} This folder contains the standar output results after executing each program over the \acrshort{lang}
    \item \textbf{src} This folder contains the \mintinline{bash}{cpp} files with the source code implementation.
    \item \textbf{\mintinline{bash}{./Makefile}} Makefile to compile programs
    \item \textbf{\mintinline{bash}{./README.md}} Markdown file with the instructions to run the program.
    \item \textbf{\mintinline{bash}{./run_closeness.sh}} Script file for running the algorithm over the \acrshort{lang} in order to calculate \textit{Closeness Centrality} and 
    main features of each language.
    \item \textbf{\mintinline{bash}{./run_montecarlo.sh}} Script file for running the aproximation algorithms for each language with null models.
\end{itemize}

\end{document}