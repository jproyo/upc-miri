\documentclass[12pt, a4paper]{article}
\usepackage[utf8]{inputenc}
\usepackage{amsmath}
\usepackage{amsthm}
\usepackage{amssymb}
\usepackage{graphicx}
\usepackage{parskip}
\usepackage{hyperref}
\usepackage{fancyhdr}
\usepackage{lastpage}
\usepackage[vlined,ruled]{algorithm2e}
\usepackage[acronym]{glossaries}
\usepackage{caption}
\usepackage{titlesec}

\titleformat{\section}
  {\normalfont\bfseries}{Problem 2.\thesection}
  {0em}{}

\titleformat{\subsection}
  {\normalfont\bfseries}{2.\thesubsection}
  {0em}{}

\title{%
  Stochastic Network Modeling \\
  Homework 2 - Solutions
}
\author{%
  Juan Pablo Royo Sales\\
  \small{Universitat Politècnica de Catalunya}
}
\date\today

\pagestyle{fancy}
\fancyhf{}
\fancyhead[C]{}
\fancyhead[R]{Juan Pablo Royo Sales - UPC MIRI}
\fancyhead[L]{SNM - Homework 2}
\fancyfoot[L,C]{}
\fancyfoot[R]{Page \thepage{} of \pageref{LastPage}}
\setlength{\headheight}{15pt}
\renewcommand{\headrulewidth}{0.4pt}
\renewcommand{\footrulewidth}{0.4pt}

\renewcommand{\qedsymbol}{$\blacksquare$}
\newacronym{bpp}{BPP}{BPP}

\begin{document}

\maketitle

\section{}
\subsection{}

\begin{subequations}
  \begin{align}
    1 &= \int_0^G \alpha x^2 dx + \int_G^{\infty} \alpha \frac{G^3}{x^2} dx\\
      &= \alpha \frac{x^3}{3}\Bigm|_0^G + \alpha G^3 \frac{1}{x}\Bigm|_G^{\infty}\\
      &= \alpha \frac{G^3}{3} + \left(\alpha \frac{G^3}{\infty} - \alpha \frac{G^3}{G}\right)\\
      &= \alpha \left(\frac{G^3}{3} - G^2\right)\\
      &= \alpha \frac{G^3 - 3G^2}{3}\\
  \end{align}
\end{subequations}

Therefore, 
\begin{equation*}
  \alpha = \frac{3}{G^2(G - 3)}
\end{equation*}

\subsection{}

\[
  f(x) = \begin{cases}
          \frac{3x^2}{G^2(G - 3)} & 0 \leq x \leq G\\
          \frac{3G^2}{x^2(G - 3)} & x \geq G
          \end{cases}
\]

\begin{subequations}
  \begin{align}
    F(x) &= \int_0^x \frac{3t^2}{G^2(G - 3)} dt + \int_x^{+\infty} \frac{3G^2}{t^2(G - 3)} dt\\
         &= \frac{3}{G^2(G - 3)}\frac{t^3}{3}\Bigm|_0^x\\
         &= \frac{t^3}{G^2(G - 3)}\Bigm|_0^x\\
         &= \frac{x^3}{G^2(G - 3)}
  \end{align}
\end{subequations}

\subsection{}
I am not sure but i think the idea is to put $F(x) = 0.95$. If that the case it would be 

\begin{subequations}
  \begin{align}
    \frac{x^3}{G^2(G - 3)}  &= 0.95\\
    \frac{0.95}{G^2(G - 3)} &= x^3\\
    \sqrt[3]{\frac{0.95}{G^2(G - 3)}} &= x\\
  \end{align}
\end{subequations}

\section{}

Not sure about this...

\begin{itemize}
  \item $f(x_1, x_2)$
  \begin{subequations}
    \begin{align}
      f(x_1,x_2) &= \sum_{x_1}f(x_2 | x_1)P(x_1)\\
                 &= \frac{1}{6}\sum_{x_1}x_1e^{-x_1x_2}
    \end{align}
  \end{subequations}
  \item $f(x_2)$
  \begin{subequations}
    \begin{align}
      f(x_2) &= \int_{x_1}^{+\infty}f(x_1, x_2) dx_1 \\
                 &= 1-e^{-x_1x_2}\Bigm|_1^6\\
                 &= -e^{-7x_2}
    \end{align}
  \end{subequations}
  \item $E[X_2]$ does it mean the conditional $E[X_2]$?
\end{itemize}

\section{}
\subsection{}
\begin{subequations}
  \begin{align}
    F(X > Y) = P(X > Y | X < 1) &= \int_y^1 3(1 - x) dx \\
               &= 3(x - \frac{x^2}{2})\Bigm|_y^1\\
               &= \frac{3}{2} - 3(y - \frac{y^2}{2})
  \end{align}
\end{subequations}

\subsection{}
\begin{subequations}
  \begin{align}
    F(Y) &= \int_x 3(1 - x) dx \\
         &= 3(x - \frac{x^2}{2})\Bigm|_x^1\\
         &= \frac{3}{2} - 3(x - \frac{x^2}{2})
  \end{align}
\end{subequations}

\subsection{}
\begin{subequations}
  \begin{align}
    E(Y) &= \int_0^1 (1-3(x - \frac{x^2}{2})) dx \\
         &= \frac{x(x^2 - 3x + 2))}{2}\Bigm|_0^1\\
         &= 0
  \end{align}
\end{subequations}

\section{}
\begin{subequations}
  \begin{align}
    P[X < Y] + P[Y < X] = \int_0^1 f(x)dx \int_y^x f(y)dy + \int_0^1 f(y)dy \int_x^y f(x)dx
  \end{align}
\end{subequations}

\section{}
\begin{subequations}
  \begin{align}
    P[X > Y] + P[Y > X] = \int_0^1 f(x)dx \int_x^y f(y)dy + \int_0^1 f(y)dy \int_y^x f(x)dx
  \end{align}
\end{subequations}

\section{}
Since the previous 2 i dont think there are ok, and it seems intuitively that this is a composition of previous 2.

\section{}
Let $P[X=7] = \frac{6}{36}$ be the probability of wining a 7 in first shot.

Let $P[X=11] = \frac{2}{36}$ be the probability of wining a 11 in first shot.

So the $P[\text{Win in first shot}] = \frac{8}{36}$.

Let $P[X=2] = \frac{1}{36}$ be the probability of lossing with 2 in first shot.

Let $P[X=3] = \frac{2}{36}$ be the probability of lossing with 3 in first shot.

Let $P[X=12] = \frac{1}{36}$ be the probability of lossing with 12 in first shot.

So the $P[\text{Lose in first shot}] = \frac{4}{36}$.

Probability of getting some number is $P(n) = \frac{n-1}{36}$

\begin{subequations}
  \begin{align}
    P[W] &= \frac{8}{36} + 2 \sum_3^5 P[W|n]P(n)\\
         &= \frac{8}{36} + 2 \sum_3^5 \frac{i}{36}\frac{i}{i+6}\\
         &= \frac{2}{36} (4+\sum_3^5\frac{i^2}{i+6})\\
         &= \frac{2}{36} (4+\frac{9}{9}+\frac{16}{10}+\frac{25}{11})\\
         &= 0.492929293
  \end{align}
\end{subequations}

\end{document}

