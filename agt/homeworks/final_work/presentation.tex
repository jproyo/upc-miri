\documentclass{beamer}
\usetheme{Madrid}
\usepackage{bookmark}
\usepackage[T1]{fontenc}
\usepackage[utf8]{inputenc}
\usepackage{amsmath}
\usepackage{amsfonts}
\usepackage{amsthm}
\usepackage{array}
\usepackage{graphicx}
\usepackage{listings}
\usepackage{color}
\usepackage{fancyhdr}
\usepackage{tikz}
\usetikzlibrary{calc,shapes.multipart,chains,arrows}
\usepackage{float}
\usepackage{caption}
\usepackage[linguistics]{forest}
\setbeamertemplate{frametitle}{%
  \usebeamerfont{frametitle}\insertframetitle%
  \vphantom{g}
  \par\hspace*{-\dimexpr0.5\paperwidth-0.5\textwidth}\rule[0.5\baselineskip]{\paperwidth}{0.4pt}
  \par\vspace*{-\baselineskip}
}
\setbeamertemplate{footline}
{
  \leavevmode%
  \hbox{%
  \begin{beamercolorbox}[wd=.333333\paperwidth,ht=2.25ex,dp=1ex,center]{author in head/foot}%
    \usebeamerfont{author in head/foot}\insertauthor
  \end{beamercolorbox}%
  \begin{beamercolorbox}[wd=.333333\paperwidth,ht=2.25ex,dp=1ex,center]{title in head/foot}%
    \usebeamerfont{title in head/foot}\insertshortinstitute
  \end{beamercolorbox}%
  \begin{beamercolorbox}[wd=.333333\paperwidth,ht=2.25ex,dp=1ex,right]{date in head/foot}%
    \usebeamerfont{date in head/foot}\insertshortdate{}\hspace*{2em}
    \insertframenumber{} / \inserttotalframenumber\hspace*{2ex} 
  \end{beamercolorbox}}%
  \vskip0pt%
}

\usetheme{Boadilla}

\title{Boolean Combinations of Weighted Voting Games}

\author{Juan Pablo Royo Sales}
\institute{Universitat Politècnica de Catalunya}
\date{January 18th, 2021}
\begin{document}

\begin{frame}
\titlepage
\end{frame}
  
\begin{frame}[fragile]{Agenda}
  \tableofcontents
\end{frame}

\begin{frame}[fragile]{Agenda}
  \section{Introduction}
  \tableofcontents[currentsection]
\end{frame}

  
\begin{frame}[fragile]{Introduction}
  \begin{block}{Basic Notions}
    \begin{itemize}
      \item Based on \textit{Boolean Combinations of Weighted Voting Games paper} \textit{\textbf{BWVG}}\footnotemark
      \item It is a natural Generalization over \textbf{Weighted Voting Games}
      \item Intuitively is a decision making process via multiple committees
      \item Each committee has the authority to decide the outcome \textit{"yes" or "no"} about an issue.
      \item And each committee is a WVG 
      \item Individuals can appear in multiple committees
      \item Different committees can have different Threshold values
    \end{itemize}
  \end{block}
  \footnotetext{\tiny{Piotr Faliszewski, Edith Elkind, and Michael Wooldridge. 2009. Boolean combinations of weighted voting games. In Proceedings of The 8th International Conference on Autonomous Agents and Multiagent Systems - Volume 1 (AAMAS '09). International Foundation for Autonomous Agents and Multiagent Systems, Richland, SC, 185–192.}}
\end{frame}

\begin{frame}[fragile]{Introduction}
  \begin{block}{Questions to be answered?}
    \begin{itemize}
      \item Which coalitions might be able to bring the goal about?
      \item How important is a particular individual with respect to the achievement of the goal?
    \end{itemize}
  \end{block}
\end{frame}

\begin{frame}[fragile]{Introduction}
  \begin{block}{Goals of the Paper}
    \begin{itemize}
      \item Formal Definition of \textbf{BWVG}
      \item Investigate Computational Properties of \textbf{BWVG}
    \end{itemize}
  \end{block}
\end{frame}

\begin{frame}[fragile]{Agenda}
  \section{Preliminary Definitions}
  \tableofcontents[currentsection]
\end{frame}

\begin{frame}[fragile]{Preliminary Definitions}
  \begin{block}{Propositional Logic}
    \begin{itemize}
      \item Let $\Phi = \{p,q,\dots\}$ be a fixed non-empty vocabulary of Boolean variables    
      \item Let $\mathcal{L}$ denote the set of formulas of propositional logic over $\Phi$, constructed using conventional Boolean operators: 
      $\land, \lor, \implies, \iff, \neg$
    \end{itemize}
  \end{block}
\end{frame}


\begin{frame}[fragile]{Preliminary Definitions}
  \begin{block}{Propositional Logic}
    \begin{itemize}
      \item Let $\Phi = \{p,q,\dots\}$ be a fixed non-empty vocabulary of Boolean variables    
      \item Let $\mathcal{L}$ denote the set of formulas of propositional logic over $\Phi$, constructed using conventional Boolean operators: 
      $\land, \lor, \implies, \iff, \neg$
      \item If "$\lor$" and "$\land$" are the only operators appearing in formula $\varphi$, se say that $\varphi$ is \textbf{monotone}
    \end{itemize}
  \end{block}
\end{frame}


\begin{frame}[fragile]{Preliminary Definitions}
  \begin{block}{Propositional Logic}
    \begin{itemize}
      \item Let $\Phi = \{p,q,\dots\}$ be a fixed non-empty vocabulary of Boolean variables    
      \item Let $\mathcal{L}$ denote the set of formulas of propositional logic over $\Phi$, constructed using conventional Boolean operators: 
      $\land, \lor, \implies, \iff, \neg$
      \item If "$\lor$" and "$\land$" are the only operators appearing in formula $\varphi$, se say that $\varphi$ is \textbf{monotone}
      \item If $\xi \subseteq \Phi$, we write $\xi \models \varphi$ mean that $\varphi$ is true satisfied by valuation $\xi$
    \end{itemize}
  \end{block}
\end{frame}

\begin{frame}[fragile]{Preliminary Definitions}
  \begin{block}{Simple Games}
    \begin{itemize}
      \item A coalitional game is Simple if $v(C) \in \{0,1\} \forall C \subseteq N$
      \item $C$ wins if $v(C) = 1$ and $C$ losses otherwise.  
      \item A Simple Game is \textbf{monotone} if $v(C) = 1 \implies v(C') = 1$ for any $C \subseteq C'$.
      \item In this paper authors consider both \textit{monotone} and \textit{non-monotone} Simple Games.
      \item They assume games with finite numbers of players $|N| = n$, $N = \{1,\dots,n\}$
    \end{itemize}
  \end{block}
\end{frame}

\begin{frame}[fragile]{Preliminary Definitions}
  \begin{block}{Weighted Voting Games}
    \begin{itemize}
      \item Given $N = \{1,\dots,n\}$ players
      \item A list of $n$ weights $w = (w_1, \dots, w_n) \in \mathbb{R}^n$
      \item A threshold $T \in \mathbb{R}$
      \item When $N$ is clear from the context $q = (T; w_1, \dots, w_n)$ to denote a $WVG$ $g$
      \item $w(C)$ total weight of coalition $C$, $w(C) = \sum_{i \in C} w_i$ 
      \item Characteristic function given by $v(C) = 1$ if $w(C) \geq T$ and $v(C) = 0$ otherwise.
      \item If all Weights are non-negative the game is monotone.
    \end{itemize}
  \end{block}
\end{frame}

\begin{frame}[fragile]{Preliminary Definitions}
  \begin{block}{Computational Complexity}
    \begin{itemize}
      \item $P$, $NP$, $coNP$, $\Sigma_2^p$, $\Pi_2^p$
    \end{itemize}
  \end{block}
\end{frame}

\begin{frame}[fragile]{Preliminary Definitions}
  \begin{block}{Computational Complexity}
    \begin{itemize}
      \item $P$, $NP$, $coNP$, $\Sigma_2^p$, $\Pi_2^p$
      \item $D^p$: A Language $L \in D^p$ if $L = L_1 \cap L_2$, for some language $L_1 \in NP$ and $L_2 \in coNP$
    \end{itemize}
  \end{block}
\end{frame}


\begin{frame}[fragile]{Preliminary Definitions}
  \begin{block}{Computational Complexity}
    \begin{itemize}
      \item $P$, $NP$, $coNP$, $\Sigma_2^p$, $\Pi_2^p$
      \item $D^p$: A Language $L \in D^p$ if $L = L_1 \cap L_2$, for some language $L_1 \in NP$ and $L_2 \in coNP$
      \item $D_2^p$: A Language $L \in D_2^p$ if $L = L_1 \cap L_2$, for some language $L_1 \in \Sigma_2^p$ and $L_2 \in \Pi_2^p$
    \end{itemize}
  \end{block}
\end{frame}

\begin{frame}[fragile]{Agenda}
  \section{Formal Definition BWVG}
  \tableofcontents[currentsection]
\end{frame}

\begin{frame}[fragile]{Boolean Weighted Voting Games}
  \begin{block}{Definition}
    A \textbf{BWVG} is a tuple $G = \langle N, \mathcal{G}, \Phi, \varphi \rangle$, where:
    \begin{itemize}
      \item $N = \{1,\dots,n\}$ is a set of players;
      \item $\mathcal{G} = \{g^1, \dots, g^m\}$ is a Set of \textbf{WVG} over $N$, where $j$th game, $g^j$, is
      given by a vector of weights $w^j = (w_1^j, \dots, w_n^j)$ and a Threshold $T^j$. $\mathcal{G}$ is called the \textbf{component games} of $G$;
      \item $\Phi = \{p^1, \dots, p^m\}$ Set of Propositional Variables, in which each $p^j$ correspond with the \textbf{component} $g^j$;
      \item $\varphi$ is a propositional formula over $\Phi$. 
    \end{itemize}
  \end{block}
  \begin{block}{Shorthand Definition}
    Example:
    \begin{itemize}
      \item $g^1 \land g^2 \equiv \langle N, \{g^1,g^2\}, \{p^1, p^2\}, p^1 \land p^2 \rangle$
    \end{itemize}
  \end{block}
\end{frame}

\begin{frame}[fragile]{Boolean Weighted Voting Games}
  \begin{block}{Winning Coalition}
    We say that $C$ is a \textit{wins} $G$ if:
    \begin{equation}
      \exists \xi_1 \subseteq \Phi_C\ :\ \forall \xi_2 \subseteq (\Phi \setminus \Phi_C)\ :\ \xi_1 \cup \xi_2 \models \varphi \label{eq:1}
    \end{equation}
  \end{block}
\end{frame}

\begin{frame}[fragile]{Boolean Weighted Voting Games}
  \begin{block}{Winning Coalition}
    We say that $C$ is a \textit{wins} $G$ if:
    \begin{equation}
      \exists \xi_1 \subseteq \Phi_C\ :\ \forall \xi_2 \subseteq (\Phi \setminus \Phi_C)\ :\ \xi_1 \cup \xi_2 \models \varphi \label{eq:1}
    \end{equation}
  \end{block}
  \begin{block}{Intuitively~\ref{eq:1}}
    A coalition $C$ wins if it is able to fix variables under its control in such a way that the goal formula $\varphi$ is guaranteed to be \textbf{True}.
  \end{block}
\end{frame}

\begin{frame}[fragile]{Boolean Weighted Voting Games}
  \begin{block}{Winning Coalition}
    We say that $C$ is a \textit{wins} $G$ if:
    \begin{equation}
      \exists \xi_1 \subseteq \Phi_C\ :\ \forall \xi_2 \subseteq (\Phi \setminus \Phi_C)\ :\ \xi_1 \cup \xi_2 \models \varphi \label{eq:1}
    \end{equation}
  \end{block}
  \begin{block}{Intuitively~\ref{eq:1}}
    A coalition $C$ wins if it is able to fix variables under its control in such a way that the goal formula $\varphi$ is guaranteed to be \textbf{True}.
  \end{block}
  \begin{block}{Notes}
    It is allowed \textbf{\textit{BWVG}} to contain \textit{negative} weights
  \end{block}
\end{frame}

\begin{frame}[fragile]{Agenda}
  \section{Representational Complexity}
  \tableofcontents[currentsection]
\end{frame}

\begin{frame}[fragile]{Representational Complexity}
  \begin{block}{Preliminaries}
    \begin{itemize}
      \item Any Simple Game with $n$ players can be represented as a \textit{K-Vector Weighted Voting Game}
      for $k = O(2^n)$, and therefore as a \textbf{BWVG} with $O(2^n)$ \textbf{component games} $\mathcal{G}$.
    \end{itemize}
  \end{block}
\end{frame}

\begin{frame}[fragile]{Representational Complexity}
  \begin{block}{Preliminaries}
    \begin{itemize}
      \item Any Simple Game with $n$ players can be represented as a \textit{K-Vector Weighted Voting Game}
      for $k = O(2^n)$, and therefore as a \textbf{BWVG} with $O(2^n)$ \textbf{component games} $\mathcal{G}$.
      \item That worst-case unfortunately cannot be improved in \textbf{BWVG}
    \end{itemize}
  \end{block}
\end{frame}

\begin{frame}[fragile]{Representational Complexity}
  \newtheorem{prop1}{Proposition}
  \begin{prop1}
    The total number of Boolean weighted voting games with $|N| = n$ and $|\varphi| = s$ is
    most $2^{O(sn^2 \log(sn))}$
  \end{prop1}
\end{frame}

\begin{frame}[fragile]{Representational Complexity}
  \begin{prop1}
    The total number of Boolean weighted voting games with $|N| = n$ and $|\varphi| = s$ is
    most $2^{O(sn^2 \log(sn))}$
  \end{prop1}
  \begin{proof}
    \begin{itemize}
      \item Any weighted voting game\footnotemark\ can be represented using Integer weights whose absolute values do not exceed $2^{O(n \log{n})}$
    \end{itemize}
  \end{proof}
  \footnotetext{\tiny{S. Muroga. Threshold Logic and its Applications. Wiley, 1971.}}
\end{frame}

\begin{frame}[fragile]{Representational Complexity}
  \begin{prop1}
    The total number of Boolean weighted voting games with $|N| = n$ and $|\varphi| = s$ is
    most $2^{O(sn^2 \log(sn))}$
  \end{prop1}
  \begin{proof}
    \begin{itemize}
      \item Any weighted voting game\footnotemark\ can be represented using Integer weights whose absolute values do not exceed $2^{O(n \log{n})}$
      \item \textit{w.l.g.} we assumed that $|\mathcal{G}| = |\Phi|$ and $|\Phi| \leq |\varphi| = s$
    \end{itemize}
  \end{proof}
  \footnotetext{\tiny{S. Muroga. Threshold Logic and its Applications. Wiley, 1971.}}
\end{frame}

\begin{frame}[fragile]{Representational Complexity}
  \begin{prop1}
    The total number of Boolean weighted voting games with $|N| = n$ and $|\varphi| = s$ is
    most $2^{O(sn^2 \log(sn))}$
  \end{prop1}
  \begin{proof}
    \begin{itemize}
      \item Any weighted voting game\footnotemark\ can be represented using Integer weights whose absolute values do not exceed $2^{O(n \log{n})}$
      \item \textit{w.l.g.} we assumed that $|\mathcal{G}| = |\Phi|$ and $|\Phi| \leq |\varphi| = s$
      \item Given a \textbf{BWVG} $G$ with $n$ players and $|\varphi| = s$, we can find a equivalent representation using $O(sn^2 \log{n})$
      bits to represent all weights in ALL components, plus another $O(s \log{s})$ bits to represent $\mathcal{G}, \Phi$ and $\varphi$.
    \end{itemize}
  \end{proof}
  \footnotetext{\tiny{S. Muroga. Threshold Logic and its Applications. Wiley, 1971.}}
\end{frame}

\begin{frame}[fragile]{Representational Complexity}
  \begin{prop1}
    The total number of Boolean weighted voting games with $|N| = n$ and $|\varphi| = s$ is
    most $2^{O(sn^2 \log(sn))}$
  \end{prop1}
  \begin{proof}
    \begin{itemize}
      \item Any weighted voting game\footnotemark\ can be represented using Integer weights whose absolute values do not exceed $2^{O(n \log{n})}$
      \item \textit{w.l.g.} we assumed that $|\mathcal{G}| = |\Phi|$ and $|\Phi| \leq |\varphi| = s$
      \item Given a \textbf{BWVG} $G$ with $n$ players and $|\varphi| = s$, we can find a equivalent representation using $O(sn^2 \log{n})$
      bits to represent all weights in ALL components, plus another $O(s \log{s})$ bits to represent $\mathcal{G}, \Phi$ and $\varphi$.
      \item Therefore, the total number of \textbf{distinct games} can be represented as \textbf{BWVG} with $|N| = n$ and $|\varphi| = s$ is
      $2^{O(sn^2 \log(sn))}$
    \end{itemize}
  \end{proof}
  \footnotetext{\tiny{S. Muroga. Threshold Logic and its Applications. Wiley, 1971.}}
\end{frame}

\begin{frame}[fragile]{Representational Complexity}
  \begin{block}{Linear Representation - Specific Case}    
    \begin{itemize}
      \item We are going to show that for \textbf{some specific} instance that captures \textbf{realistic voting scenarios} that can be improve with linear representation.
    \end{itemize}
  \end{block}
\end{frame}

\begin{frame}[fragile]{Representational Complexity}
  \begin{block}{Linear Representation - Specific Case}    
    \begin{itemize}
      \item We are going to show that for \textbf{some specific} instance that captures \textbf{realistic voting scenarios} that can be improve with linear representation.
    \end{itemize}
  \end{block}
  \newtheorem{theorem1}{Theorem}
  \begin{theorem1}
   Consider a \textbf{BWVG} $G = \langle N, \mathcal{G}, \Phi, \varphi \rangle$ where
   $\mathcal{G} = \{g^1, g^2\}, g^1 = (k; 1,0,\dots,1,0), g^2 = (k;0,1,\dots,0,1), |N| = 2k$
   and $\varphi = p^1 \lor p^2$. To represent $G$ as a conjunction of $m$ weighted voting games
   requires $m \geq k/2$ \textbf{component games} $\mathcal{G}$ 
  \end{theorem1}
\end{frame}

\begin{frame}[fragile]{Representational Complexity}
  \begin{proof}
    \begin{itemize}
      \item Poof by contradiction
    \end{itemize}
  \end{proof}
\end{frame}

\begin{frame}[fragile]{Representational Complexity}
  \begin{proof}
    \begin{itemize}
      \item Poof by contradiction
      \item A coalition $C$ to win in $G$ has to contain either \textit{even players} or \textit{odd players}
    \end{itemize}
  \end{proof}
\end{frame}

\begin{frame}[fragile]{Representational Complexity}
  \begin{proof}
    \begin{itemize}
      \item Poof by contradiction
      \item A coalition $C$ to win in $G$ has to contain either \textit{even players} or \textit{odd players}
      \item Any \textit{maximal losing coalition} (MLC) in $G$ is of the form $N \setminus \{2i, 2j - 1\}$ where
      $i,j \in \{1,\dots,k\}$, denote as $C_{i,j}$
    \end{itemize}
  \end{proof}
\end{frame}

\begin{frame}[fragile]{Representational Complexity}
  \begin{proof}
    \begin{itemize}
      \item Poof by contradiction
      \item A coalition $C$ to win in $G$ has to contain either \textit{even players} or \textit{odd players}
      \item Any \textit{maximal losing coalition} (MLC) in $G$ is of the form $N \setminus \{2i, 2j - 1\}$ where
      $i,j \in \{1,\dots,k\}$, denote as $C_{i,j}$
      \item There are exactly $k^2$ MLC
    \end{itemize}
  \end{proof}
\end{frame}

\begin{frame}[fragile]{Representational Complexity}
  \begin{proof}
    \begin{itemize}
      \item Poof by contradiction
      \item A coalition $C$ to win in $G$ has to contain either \textit{even players} or \textit{odd players}
      \item Any \textit{maximal losing coalition} (MLC) in $G$ is of the form $N \setminus \{2i, 2j - 1\}$ where
      $i,j \in \{1,\dots,k\}$, denote as $C_{i,j}$
      \item There are exactly $k^2$ MLC
      \item 2 MLC $C_{i,j}$ and $C_{i',j'}$ clashes if $i=i'$ or $j= j'$, if $C_{i,j} \cup C_{i',j'} \neq N$
    \end{itemize}
  \end{proof}
\end{frame}

\begin{frame}[fragile]{Representational Complexity}
  \begin{proof}
    \begin{itemize}
      \item Poof by contradiction
      \item A coalition $C$ to win in $G$ has to contain either \textit{even players} or \textit{odd players}
      \item Any \textit{maximal losing coalition} (MLC) in $G$ is of the form $N \setminus \{2i, 2j - 1\}$ where
      $i,j \in \{1,\dots,k\}$, denote as $C_{i,j}$
      \item There are exactly $k^2$ MLC
      \item 2 MLC $C_{i,j}$ and $C_{i',j'}$ clashes if $i=i'$ or $j= j'$, if $C_{i,j} \cup C_{i',j'} \neq N$
      \item \textbf{Suppose} that $G$ can be represented as $\langle N, \{h^1, \dots, h^m\}, \{q^1, \dots, q^m\}, q^1 \land \dots \land q^m \rangle$ with $m < k/2$
    \end{itemize}
  \end{proof}
\end{frame}

\begin{frame}[fragile]{Representational Complexity}
  \begin{proof}
    \begin{itemize}
      \item Poof by contradiction
      \item A coalition $C$ to win in $G$ has to contain either \textit{even players} or \textit{odd players}
      \item Any \textit{maximal losing coalition} (MLC) in $G$ is of the form $N \setminus \{2i, 2j - 1\}$ where
      $i,j \in \{1,\dots,k\}$, denote as $C_{i,j}$
      \item There are exactly $k^2$ MLC
      \item 2 MLC $C_{i,j}$ and $C_{i',j'}$ clashes if $i=i'$ or $j= j'$, if $C_{i,j} \cup C_{i',j'} \neq N$
      \item \textbf{Suppose} that $G$ can be represented as $\langle N, \{h^1, \dots, h^m\}, \{q^1, \dots, q^m\}, q^1 \land \dots \land q^m \rangle$ with $m < k/2$
      \item Each component has to lose in at least one game $h^1, \dots, h^m$. By \textbf{pigeonhole principle}, there must be at least $1$
      component game (w.l.g.) that is lost by at least $2k$ distinct MLC.
    \end{itemize}
  \end{proof}
\end{frame}

\begin{frame}[fragile]{Representational Complexity}
  \begin{proof}[Proof Cont.]
    \begin{itemize}
      \item Fix an arbitrary MLC $C_{i,j}$ that loses in $h^1$
    \end{itemize}
  \end{proof}
\end{frame}


\begin{frame}[fragile]{Representational Complexity}
  \begin{proof}[Proof Cont.]
    \begin{itemize}
      \item Fix an arbitrary MLC $C_{i,j}$ that loses in $h^1$
      \item Among $2k$ MLCs that loses in $h^1$ there can be at most $k - 1$ MLCs of the form
      $C_{i,j'}, j' \neq j$ and $C_{i',j}, i' \neq i$
    \end{itemize}
\end{proof}
\end{frame}

\begin{frame}[fragile]{Representational Complexity}
  \begin{proof}[Proof Cont.]
    \begin{itemize}
      \item Fix an arbitrary MLC $C_{i,j}$ that loses in $h^1$
      \item Among $2k$ MLCs that loses in $h^1$ there can be at most $k - 1$ MLCs of the form
      $C_{i,j'}, j' \neq j$ and $C_{i',j}, i' \neq i$
      \item There must be a $C_{x,y}$ that loses in $h^1$ and don't clashes with $C_{i,j}$.
    \end{itemize}
\end{proof}
\end{frame}


\begin{frame}[fragile]{Representational Complexity}
  \begin{proof}[Proof Cont.]
    \begin{itemize}
      \item Fix an arbitrary MLC $C_{i,j}$ that loses in $h^1$
      \item Among $2k$ MLCs that loses in $h^1$ there can be at most $k - 1$ MLCs of the form
      $C_{i,j'}, j' \neq j$ and $C_{i',j}, i' \neq i$
      \item There must be a $C_{x,y}$ that loses in $h^1$ and don't clashes with $C_{i,j}$.
      \item Let $h^1 = (T; w_1, \dots, w_n)$, we have
    \end{itemize}
    \begin{equation}
      w(N) - w_{2i} - w_{2j - 1} < T; w(N) - w_{2x} - w_{2y - 1} < T\label{eq:2}
    \end{equation}
\end{proof}
\end{frame}

\begin{frame}[fragile]{Representational Complexity}
  \begin{proof}[Proof Cont.]
    \begin{itemize}
      \item Fix an arbitrary MLC $C_{i,j}$ that loses in $h^1$
      \item Among $2k$ MLCs that loses in $h^1$ there can be at most $k - 1$ MLCs of the form
      $C_{i,j'}, j' \neq j$ and $C_{i',j}, i' \neq i$
      \item There must be a $C_{x,y}$ that loses in $h^1$ and don't clashes with $C_{i,j}$.
      \item Let $h^1 = (T; w_1, \dots, w_n)$, we have
    \end{itemize}
    \begin{equation}
      w(N) - w_{2i} - w_{2j - 1} < T; w(N) - w_{2x} - w_{2y - 1} < T\label{eq:2}
    \end{equation}
    \begin{itemize}
      \item Also, $C_{i,j} \setminus \{2y -1\} \cup \{2i\}$ and $C_{x,y} \setminus \{2y -1\} \cup \{2i\}$ are wining in $G$ and hence in $h^1$
    \end{itemize}
    \begin{equation}
      w(N) - w_{2j - 1} - w_{2y - 1} \geq T; w(N) - w_{2i} - w_{2x} \geq T\label{eq:3}
    \end{equation}
    \textbf{Equation~\ref{eq:2} and~\ref{eq:3} give a contradiction}
    Therefore $m \geq k/2$.
\end{proof}
\end{frame}


\begin{frame}[fragile]{Agenda}
  \section{Decision Problems in BWVG}
  \tableofcontents[currentsection]
\end{frame}

\begin{frame}[fragile]{Decision Problems in BWVG}
  \newtheorem{theorem3}{Winning Coalitions}
  \begin{theorem3}
    Given a game $G = \langle N, \mathcal{G}, \Phi, \varphi \rangle$ and a coalition $C \subseteq N$, 
    deciding whether $C$ wins in $G$ is $\Sigma_2^p$-complete. This results holds even if there are 2 players
    and the weights of all players in all components are in $\{0,1\}$. However, the problem is in $P$ if the underlying
    formula is monotone.
  \end{theorem3}
\end{frame}

\begin{frame}[fragile]{Decision Problems in BWVG}
  \begin{theorem3}
    Given a game $G = \langle N, \mathcal{G}, \Phi, \varphi \rangle$ and a coalition $C \subseteq N$, 
    deciding whether $C$ wins in $G$ is $\Sigma_2^p$-complete. This results holds even if there are 2 players
    and the weights of all players in all components are in $\{0,1\}$. However, the problem is in $P$ if the underlying
    formula is monotone.
  \end{theorem3}
  \begin{proof}[Proof Sketch]
    \begin{itemize}
      \item By definition of Winning coalition of \textbf{BWVG} is easy to see that is in $\Sigma_2^p$ for the general case
    \end{itemize}
  \end{proof}
\end{frame}

\begin{frame}[fragile]{Decision Problems in BWVG}
  \begin{theorem3}
    Given a game $G = \langle N, \mathcal{G}, \Phi, \varphi \rangle$ and a coalition $C \subseteq N$, 
    deciding whether $C$ wins in $G$ is $\Sigma_2^p$-complete. This results holds even if there are 2 players
    and the weights of all players in all components are in $\{0,1\}$. However, the problem is in $P$ if the underlying
    formula is monotone.
  \end{theorem3}
  \begin{proof}[Proof Sketch]
    \begin{itemize}
      \item By definition of Winning coalition of \textbf{BWVG} is easy to see that is in $\Sigma_2^p$ for the general case
      \item In the case of monotonicity of propositional formula testing whether a Coalition $C$ is winning we need to set all 
      all the controlled variables by $C$ in \textbf{True}, while All others in \textbf{$\bot$}.
    \end{itemize}
  \end{proof}
\end{frame}

\begin{frame}[fragile]{Decision Problems in BWVG}
  \begin{theorem3}
    Given a game $G = \langle N, \mathcal{G}, \Phi, \varphi \rangle$ and a coalition $C \subseteq N$, 
    deciding whether $C$ wins in $G$ is $\Sigma_2^p$-complete. This results holds even if there are 2 players
    and the weights of all players in all components are in $\{0,1\}$. However, the problem is in $P$ if the underlying
    formula is monotone.
  \end{theorem3}
  \begin{proof}[Proof Sketch]
    \begin{itemize}
      \item By definition of Winning coalition of \textbf{BWVG} is easy to see that is in $\Sigma_2^p$ for the general case
      \item In the case of monotonicity of propositional formula testing whether a Coalition $C$ is winning we need to set all 
      all the controlled variables by $C$ in \textbf{True}, while All others in \textbf{$\bot$}.
      \item With formulas with few variables we can enumerate all possible truth assignments.
    \end{itemize}
  \end{proof}
\end{frame}


\begin{frame}[fragile]{Decision Problems in BWVG}
  \begin{theorem3}
    Given a game $G = \langle N, \mathcal{G}, \Phi, \varphi \rangle$ and a coalition $C \subseteq N$, 
    deciding whether $C$ wins in $G$ is $\Sigma_2^p$-complete. This results holds even if there are 2 players
    and the weights of all players in all components are in $\{0,1\}$. However, the problem is in $P$ if the underlying
    formula is monotone.
  \end{theorem3}
  \begin{proof}[Proof Sketch]
    \begin{itemize}
      \item By definition of Winning coalition of \textbf{BWVG} is easy to see that is in $\Sigma_2^p$ for the general case
      \item In the case of monotonicity of propositional formula testing whether a Coalition $C$ is winning we need to set all 
      all the controlled variables by $C$ in \textbf{True}, while All others in \textbf{$\bot$}.
      \item With formulas with few variables we can enumerate all possible truth assignments.
      \item For the case of unrestricted formulas we do a reduction from $QSAT_2$
    \end{itemize}
  \end{proof}
\end{frame}

\begin{frame}[fragile]{Decision Problems in BWVG}
  \newtheorem{theorem4}{Swing Player}
  \begin{block}{Swing Player: Definition}
    $i$ is a \textbf{swing player} for $C$ in game $G$ if $C$ loses in $G$ but $C \cup \{i\}$ wins in $G$.\newline
    The problem of deciding if $i$ is Swing Player or not, is easy if $\varphi$ is monotone or its size is bounded by a constant, but in general is
    Computationally hard.
  \end{block}
\end{frame}


\begin{frame}[fragile]{Decision Problems in BWVG}
  \begin{block}{Swing Player: Definition}
    $i$ is a \textbf{swing player} for $C$ in game $G$ if $C$ loses in $G$ but $C \cup \{i\}$ wins in $G$.\newline
    The problem of deciding if $i$ is Swing Player or not, is easy if $\varphi$ is monotone or its size is bounded by a constant, but in general is
    Computationally hard.
  \end{block}
  \begin{theorem4}
    SWINGPLAYER is $D_2^p$-complete. This holds even for 3 players and all components are of the form $\{0,1\}$. However, the problem is in $P$ if the underlying
    formula is monotone.
  \end{theorem4}
\end{frame}

\begin{frame}[fragile]{Decision Problems in BWVG}
  \begin{proof}[Proof Sketch]
    \begin{itemize}
      \item The case of monotone formulas follows directly from previous theorem of Winning Coalitions.
    \end{itemize}
  \end{proof}
\end{frame}

\begin{frame}[fragile]{Decision Problems in BWVG}
  \begin{proof}[Proof Sketch]
    \begin{itemize}
      \item The case of monotone formulas follows directly from previous theorem of Winning Coalitions.
      \item We must exhibit 2 languages $L_1$ and $L_2$, such that $L_1 \in \Sigma_2^p$, $L_2 \in \Pi_2^p$ and $SWINGPLAYER = L_1 \cap L_2$.
      \begin{subequations}
        \begin{align}
          L_1 &= \{ \langle G, C, i \rangle : C \cup \{i\} \text{ wins in } G \}\\
          L_2 &= \{ \langle G, C, i \rangle : C \text{ does not win in } G \}
        \end{align}
      \end{subequations}
    \end{itemize}
  \end{proof}
\end{frame}

\begin{frame}[fragile]{Decision Problems in BWVG}
  \begin{proof}[Proof Sketch]
    \begin{itemize}
      \item The case of monotone formulas follows directly from previous theorem of Winning Coalitions.
      \item We must exhibit 2 languages $L_1$ and $L_2$, such that $L_1 \in \Sigma_2^p$, $L_2 \in \Pi_2^p$ and $SWINGPLAYER = L_1 \cap L_2$.
      \begin{subequations}
        \begin{align}
          L_1 &= \{ \langle G, C, i \rangle : C \cup \{i\} \text{ wins in } G \}\\
          L_2 &= \{ \langle G, C, i \rangle : C \text{ does not win in } G \}
        \end{align}
      \end{subequations}
      \item Clearly $L_1 \in \Sigma_2^p$ and $L_2 \in \Pi_2^p$
    \end{itemize}
  \end{proof}
\end{frame}

\begin{frame}[fragile]{Decision Problems in BWVG}
  \begin{proof}[Proof Sketch]
    \begin{itemize}
      \item The case of monotone formulas follows directly from previous theorem of Winning Coalitions.
      \item We must exhibit 2 languages $L_1$ and $L_2$, such that $L_1 \in \Sigma_2^p$, $L_2 \in \Pi_2^p$ and $SWINGPLAYER = L_1 \cap L_2$.
      \begin{subequations}
        \begin{align}
          L_1 &= \{ \langle G, C, i \rangle : C \cup \{i\} \text{ wins in } G \}\\
          L_2 &= \{ \langle G, C, i \rangle : C \text{ does not win in } G \}
        \end{align}
      \end{subequations}
      \item Clearly $L_1 \in \Sigma_2^p$ and $L_2 \in \Pi_2^p$
      \item By definition $SWINGPLAYER = L_1 \cap L_2$
    \end{itemize}
  \end{proof}
\end{frame}


\begin{frame}[fragile]{Decision Problems in BWVG}
  \begin{proof}[Proof Sketch]
    \begin{itemize}
      \item The case of monotone formulas follows directly from previous theorem of Winning Coalitions.
      \item We must exhibit 2 languages $L_1$ and $L_2$, such that $L_1 \in \Sigma_2^p$, $L_2 \in \Pi_2^p$ and $SWINGPLAYER = L_1 \cap L_2$.
      \begin{subequations}
        \begin{align}
          L_1 &= \{ \langle G, C, i \rangle : C \cup \{i\} \text{ wins in } G \}\\
          L_2 &= \{ \langle G, C, i \rangle : C \text{ does not win in } G \}
        \end{align}
      \end{subequations}
      \item Clearly $L_1 \in \Sigma_2^p$ and $L_2 \in \Pi_2^p$
      \item By definition $SWINGPLAYER = L_1 \cap L_2$
      \item To show $D_2^p$-hardness a reduction can be provided from $D_2^p$-complete problem $SAT_2^\Sigma-UNSAT_2^\Sigma$, which is a generalization of $SAT-UNSAT$ problem.
    \end{itemize}
  \end{proof}
\end{frame}

\begin{frame}[fragile]{Decision Problems in BWVG}
  \newtheorem{theorem2}{Dummy Player}
  \begin{block}{Dummy Player: Definition}
    $i$ is a \textbf{dummy player} for $C$ in game $G$ if $v(C) = v(C \cup \{i\})$ for all $C \subseteq N \setminus \{i\}$.
  \end{block}
  \begin{theorem2}
    DUMMYPLAYER is $coNP$-hard even if all weights in all component games are in $\{0,1\}$, and $G$ is in an $m$-vector weighed voting game.
  \end{theorem2}
\end{frame}

\begin{frame}[fragile]{Decision Problems in BWVG}
  \begin{proof}[Proof Sketch]
    \begin{itemize}
      \item The proof can be done following a reduction from the clasic $NP$-complete problem \textbf{X3C (Exact Cover by 3-Sets)}
      \item An instance $\mathcal{E}$ of this problem is given
    \end{itemize}
  \end{proof}
\end{frame}

\begin{frame}[fragile]{Decision Problems in BWVG}
  \begin{proof}[Proof Sketch]
    \begin{itemize}
      \item The proof can be done following a reduction from the clasic $NP$-complete problem \textbf{X3C (Exact Cover by 3-Sets)}
      \item An instance $\mathcal{E}$ of this problem is given
      \item It is an "yes"-instance if $\mathcal{E}$ can be covered by $K$, "no"-otherwise
    \end{itemize}
  \end{proof}
\end{frame}

\begin{frame}[fragile]{Decision Problems in BWVG}
  \begin{proof}[Proof Sketch]
    \begin{itemize}
      \item The proof can be done following a reduction from the clasic $NP$-complete problem \textbf{X3C (Exact Cover by 3-Sets)}
      \item An instance $\mathcal{E}$ of this problem is given
      \item It is an "yes"-instance if $\mathcal{E}$ can be covered by $K$, "no"-otherwise
      \item Basically given this $\mathcal{E}$ a \textbf{BWVG} is constructed, where the games $\mathcal{G}$ are of size $3K+1$.
    \end{itemize}
  \end{proof}
\end{frame}

\begin{frame}[fragile]{Decision Problems in BWVG}
  \begin{proof}[Proof Sketch]
    \begin{itemize}
      \item The proof can be done following a reduction from the clasic $NP$-complete problem \textbf{X3C (Exact Cover by 3-Sets)}
      \item An instance $\mathcal{E}$ of this problem is given
      \item It is an "yes"-instance if $\mathcal{E}$ can be covered by $K$, "no"-otherwise
      \item Basically given this $\mathcal{E}$ a \textbf{BWVG} is constructed, where the games $\mathcal{G}$ are of size $3K+1$.
      \item A Coalition $C$ wins the first $3K$ games if and only if corresponds to a valid cover of $\mathcal{E}$
    \end{itemize}
  \end{proof}
\end{frame}

\begin{frame}[fragile]{Decision Problems in BWVG}
  \begin{proof}[Proof Sketch]
    \begin{itemize}
      \item The proof can be done following a reduction from the clasic $NP$-complete problem \textbf{X3C (Exact Cover by 3-Sets)}
      \item An instance $\mathcal{E}$ of this problem is given
      \item It is an "yes"-instance if $\mathcal{E}$ can be covered by $K$, "no"-otherwise
      \item Basically given this $\mathcal{E}$ a \textbf{BWVG} is constructed, where the games $\mathcal{G}$ are of size $3K+1$.
      \item A Coalition $C$ wins the first $3K$ games if and only if corresponds to a valid cover of $\mathcal{E}$
      \item Analyzing the "no"-instance and the reminding players $+1$ establish the Dummy player or not. 
    \end{itemize}
  \end{proof}
\end{frame}

\begin{frame}[fragile]{Decision Problems in BWVG}
  \begin{proof}[Proof Sketch]
    \begin{itemize}
      \item The proof can be done following a reduction from the clasic $NP$-complete problem \textbf{X3C (Exact Cover by 3-Sets)}
      \item An instance $\mathcal{E}$ of this problem is given
      \item It is an "yes"-instance if $\mathcal{E}$ can be covered by $K$, "no"-otherwise
      \item Basically given this $\mathcal{E}$ a \textbf{BWVG} is constructed, where the games $\mathcal{G}$ are of size $3K+1$.
      \item A Coalition $C$ wins the first $3K$ games if and only if corresponds to a valid cover of $\mathcal{E}$
      \item Analyzing the "no"-instance and the reminding players $+1$ establish the Dummy player or not. 
      \item Therefore, a "no"-instance of $X3C$ is an "yes"-instance of DUMMYPLAYER.
    \end{itemize}
  \end{proof}
\end{frame}


\begin{frame}[fragile]{Agenda}
  \section{Shapley Value}
  \tableofcontents[currentsection]
\end{frame}

\begin{frame}[fragile]{Shapley Value}
  \begin{block}{Shapley Value in BWVG}
    \begin{itemize}
      \item In \textbf{WVG} it is known that computing Shapley Value is hard ($\# P$-complete).
    \end{itemize}
  \end{block}
\end{frame}

\begin{frame}[fragile]{Shapley Value}
  \begin{block}{Shapley Value in BWVG}
    \begin{itemize}
      \item In \textbf{WVG} it is known that computing Shapley Value is hard ($\# P$-complete).
      \item This implies that the problem is as least as hard to \textbf{BWVG}
    \end{itemize}
  \end{block}
\end{frame}

\begin{frame}[fragile]{Shapley Value}
  \begin{block}{Shapley Value in BWVG}
    \begin{itemize}
      \item In \textbf{WVG} it is known that computing Shapley Value is hard ($\# P$-complete).
      \item This implies that the problem is as least as hard to \textbf{BWVG}
      \item However there is a \textit{poly-time} algorithm for computing Shapley Value in \textbf{WVG} with unary-encoded weights.
    \end{itemize}
  \end{block}
\end{frame}

\begin{frame}[fragile]{Shapley Value}
  \begin{block}{Shapley Value in BWVG}
    \begin{itemize}
      \item In \textbf{WVG} it is known that computing Shapley Value is hard ($\# P$-complete).
      \item This implies that the problem is as least as hard to \textbf{BWVG}
      \item However there is a \textit{poly-time} algorithm for computing Shapley Value in \textbf{WVG} with unary-encoded weights.
      \item But this is not true for \textbf{BWVG}
    \end{itemize}
  \end{block}
\end{frame}

\begin{frame}[fragile]{Shapley Value}
  \newtheorem{theorem5}{Shapley Value}
  \begin{theorem5}
    Computing a player's Shapley value in a \textbf{BWVG} is $\# P$-hard even if the game in question is a \textbf{VWVG}
    and all weights in all component games are in $\{0,1\}$.
  \end{theorem5}
\end{frame}

\begin{frame}[fragile]{Shapley Value}
  \begin{theorem5}
    Computing a player's Shapley value in a \textbf{BWVG} is $\# P$-hard even if the game in question is a \textbf{VWVG}
    and all weights in all component games are in $\{0,1\}$.
  \end{theorem5}
  \begin{proof}
    \begin{itemize}
      \item For the probe it is use a reduction of $X3C$ (Exact Cover by 3-Sets), where an instance of this problem is giving and 
      a \textbf{BWVG} is constructed based on this.
      \item Given that there is a $q$ which is a swing player for exactly $N_k$ combinations, where $N_k$ is the number of exact covers of $\mathcal{E}$,
      and the size of each such coalition is exactly $K$.
      \item Hence the Shapley Value for the $q$ player is exactly $N_k \frac{K!(\ell + 1 - K)!}{(\ell + 1)!}$
      \item $N_K$ can be compute given $sh_q^G$, $\ell$, and $K$
      \item As computing $N_K$ is $\# P$-complete, it follows the statement.
    \end{itemize}
  \end{proof}
\end{frame}

\begin{frame}[fragile]{Shapley Value}
  \begin{block}{Conclusion}
    \begin{itemize}
      \item Shapley value requires to decide whether a given players is a SWINGPLAYER or not.
      \item Since SWINGPLAYER is in $D_2^p$-complete as we have seen, this hardness cannot be improved as it seems.
    \end{itemize}
  \end{block}
 \end{frame}

\begin{frame}[fragile]{Shapley Value}
  \begin{block}{Conclusion}
    \begin{itemize}
      \item Shapley value requires to decide whether a given players is a SWINGPLAYER or not.
      \item Since SWINGPLAYER is in $D_2^p$-complete as we have seen, this hardness cannot be improved as it seems.
    \end{itemize}
  \end{block}
  \begin{block}{Poly-time}
    \begin{itemize}
      \item Shapley value can be still computed in \textit{poly-time} if both the weights are given in unary and the number of component games 
      is bounded by a constant.
    \end{itemize}
  \end{block}
 \end{frame}

 \begin{frame}[fragile]{Shapley Value}
  \newtheorem{theorem6}{Theorem: Shapley Value in Poly-Time}
  \begin{theorem6}
    Given a BWVG $G = \langle N, \mathcal{G}, \Phi, \varphi \rangle$ and a player $p \in N$, Shapley value of $p$ can be computed in time $O((n^2 + s)(4nW)^m)$, where $|\Phi| = m, |\varphi| = s, |W| = max_{i,j}|w_i^j|$
  \end{theorem6}
\end{frame}

\begin{frame}[fragile]{Shapley Value}
  \begin{theorem6}
    Given a BWVG $G = \langle N, \mathcal{G}, \Phi, \varphi \rangle$ and a player $p \in N$, Shapley value of $p$ can be computed in time $O((n^2 + s)(4nW)^m)$, where $|\Phi| = m, |\varphi| = s, |W| = max_{i,j}|w_i^j|$
  \end{theorem6}
  \begin{proof}
    \begin{itemize}
      \item Proof using Dynamic Programming.
      \item Given an Integer vector $z = (z^1,\dots,z^m) \in [-nW, nW]^m$, any $k=1,\dots,n-1$, and $t = 1,\dots,k$
    \end{itemize}
  \end{proof}
\end{frame}

\begin{frame}[fragile]{Shapley Value}
  \begin{theorem6}
    Given a BWVG $G = \langle N, \mathcal{G}, \Phi, \varphi \rangle$ and a player $p \in N$, Shapley value of $p$ can be computed in time $O((n^2 + s)(4nW)^m)$, where $|\Phi| = m, |\varphi| = s, |W| = max_{i,j}|w_i^j|$
  \end{theorem6}
  \begin{proof}
    \begin{itemize}
      \item Proof using Dynamic Programming.
      \item Given an Integer vector $z = (z^1,\dots,z^m) \in [-nW, nW]^m$, any $k=1,\dots,n-1$, and $t = 1,\dots,k$
      \item Let $N(z,t,k)$ be the number of coalitions $C \subseteq N$, such that $w^j(C) = z^j$ for $j=1,\dots,m$, $|C|=t$, and $C \subseteq \{1,\dots,k\}$.
    \end{itemize}
  \end{proof}
\end{frame}


\begin{frame}[fragile]{Shapley Value}
  \begin{theorem6}
    Given a BWVG $G = \langle N, \mathcal{G}, \Phi, \varphi \rangle$ and a player $p \in N$, Shapley value of $p$ can be computed in time $O((n^2 + s)(4nW)^m)$, where $|\Phi| = m, |\varphi| = s, |W| = max_{i,j}|w_i^j|$
  \end{theorem6}
  \begin{proof}
    \begin{itemize}
      \item Proof using Dynamic Programming.
      \item Given an Integer vector $z = (z^1,\dots,z^m) \in [-nW, nW]^m$, any $k=1,\dots,n-1$, and $t = 1,\dots,k$
      \item Let $N(z,t,k)$ be the number of coalitions $C \subseteq N$, such that $w^j(C) = z^j$ for $j=1,\dots,m$, $|C|=t$, and $C \subseteq \{1,\dots,k\}$.
      \item For $k=1$, $N(z,t,1) = 1$ if $t=1$ and $w_1^j=z^j$ and $N(z,t,1) =  0$ otherwise.
    \end{itemize}
    \begin{equation}
      N(z,t,k+1) = N(z,t,k) + N(z_{k+1},t-1,k)
    \end{equation}
  \end{proof}
\end{frame}

\begin{frame}[fragile]{Shapley Value}
  \begin{proof}[Proof Cont. Poly-Time]
    \begin{itemize}
      \item First summand Coalitions without $k+1$ player, Second summand with $k+1$ player
    \end{itemize}
  \end{proof}
\end{frame}

\begin{frame}[fragile]{Shapley Value}
  \begin{proof}[Proof Cont. Poly-Time]
    \begin{itemize}
      \item First summand Coalitions without $k+1$ player, Second summand with $k+1$ player
      \item So all $N(z,t,k)$ can be compute in $O(n^2(2nW)^m)$
    \end{itemize}
  \end{proof}
\end{frame}

\begin{frame}[fragile]{Shapley Value}
  \begin{proof}[Proof Cont. Poly-Time]
    \begin{itemize}
      \item First summand Coalitions without $k+1$ player, Second summand with $k+1$ player
      \item So all $N(z,t,k)$ can be compute in $O(n^2(2nW)^m)$
      \item Let $I(z,t) = v(C \cup \{n\})-v(C)$
      \item Computing the value of $\varphi$ under a truth assignment can be done in $O(s)$, and for a fixed vector $z$ in $O(s2^m)$. Hence all $I(z,t)$ requires $O(s2^m(2nW)^m)$
    \end{itemize}
  \end{proof}
\end{frame}

\begin{frame}[fragile]{Shapley Value}
  \begin{proof}[Proof Cont. Poly-Time]
    \begin{itemize}
      \item First summand Coalitions without $k+1$ player, Second summand with $k+1$ player
      \item So all $N(z,t,k)$ can be compute in $O(n^2(2nW)^m)$
      \item Let $I(z,t) = v(C \cup \{n\})-v(C)$
      \item Computing the value of $\varphi$ under a truth assignment can be done in $O(s)$, and for a fixed vector $z$ in $O(s2^m)$. Hence all $I(z,t)$ requires $O(s2^m(2nW)^m)$
      \item Shapley value can be computed as follows:
    \end{itemize}
  \end{proof}
  \end{frame}

\begin{frame}[fragile]{Shapley Value}
\begin{proof}[Proof Cont. Poly-Time]
  \begin{itemize}
    \item First summand Coalitions without $k+1$ player, Second summand with $k+1$ player
    \item So all $N(z,t,k)$ can be compute in $O(n^2(2nW)^m)$
    \item Let $I(z,t) = v(C \cup \{n\})-v(C)$
    \item Computing the value of $\varphi$ under a truth assignment can be done in $O(s)$, and for a fixed vector $z$ in $O(s2^m)$. Hence all $I(z,t)$ requires $O(s2^m(2nW)^m)$
    \item Shapley value can be computed as follows:
  \end{itemize}
  \begin{equation}
    sh_n = \frac{1}{n!} \sum_{z \in [-nW,nW]^m} \sum_{t=1}^{n-1} N(z,t,n-1)I(z,t)t!(n-1-t)!
  \end{equation}
  \begin{itemize}
    \item Therefore, overall running time: $O(n^2(2nW)^m)+s2^m(2nW)^m) = O((n^2+s)(4nW)^m)$
  \end{itemize}
\end{proof}
\end{frame}

\begin{frame}[fragile]{Agenda}
  \section{The Core}
  \tableofcontents[currentsection]
\end{frame}

\begin{frame}[fragile]{The Core}
  \begin{block}{The Core and BWVG}
    \begin{itemize}
      \item Problem \textbf{InCore} we are given a BWVG $G$ and a payoff vector $x$ and we are asked if $x$ belongs to $G$'s core.
      \item Problem \textbf{CoreNonEmpty} we are given a BWVG $G$ and we ask if its core is nonempty
      \item Problem \textbf{Veto} we are given a BWVG $G$ and a player $i$ and we ask if $i$ is a veto player in $G$
    \end{itemize}
  \end{block}
\end{frame}

\begin{frame}[fragile]{The Core}
  \newtheorem{theorem7}{InCore, CoreNonEmpty and Veto}
  \begin{theorem7}
    \textbf{InCore, CoreNonEmpty and Veto} are $\Pi_2^p$-complete even if $|N| = 2$ and all weights in all components games are either $0$ or $1$.
    However for non-negative weights these problems are in $P$ if the underlying formulas are monotone.
  \end{theorem7}
  \begin{block}{Proof}
    Authors Do Not provide any proof due to space restrictions.
  \end{block}
\end{frame}

\begin{frame}[fragile]{Agenda}
  \section{Conclusions}
  \tableofcontents[currentsection]
\end{frame}

\begin{frame}[fragile]{Conclusions}
  \begin{block}{Conclusions}
    \begin{itemize}
      \item \textbf{BWVG} are more expressive than \textbf{VWVG}, in particular when formulas are monotone
    \end{itemize}
  \end{block}
\end{frame}

\begin{frame}[fragile]{Conclusions}
  \begin{block}{Conclusions}
    \begin{itemize}
      \item \textbf{BWVG} are more expressive than \textbf{VWVG}, in particular when formulas are monotone
      \item Although expressiveness gain, \textbf{BWVG} are worst in terms of Computational Complexity
    \end{itemize}
  \end{block}
\end{frame}

\begin{frame}[fragile]{Conclusions}
  \begin{block}{Conclusions}
    \begin{itemize}
      \item \textbf{BWVG} are more expressive than \textbf{VWVG}, in particular when formulas are monotone
      \item Although expressiveness gain, \textbf{BWVG} are worst in terms of Computational Complexity
      \item Unrestricted \textbf{BWVG} leads to increase Complexity
    \end{itemize}
  \end{block}
\end{frame}

\begin{frame}[fragile]{Conclusions}
  \begin{block}{Conclusions}
    \begin{itemize}
      \item \textbf{BWVG} are more expressive than \textbf{VWVG}, in particular when formulas are monotone
      \item Although expressiveness gain, \textbf{BWVG} are worst in terms of Computational Complexity
      \item Unrestricted \textbf{BWVG} leads to increase Complexity
      \item As we have seen there are trade-off to deal with this increase of Complexity and gain in expressiveness
    \end{itemize}
  \end{block}
\end{frame}

\begin{frame}
  \begin{center}
    \textbf{\huge{Thank you!!}}
    \end{center}
\end{frame}

\end{document}