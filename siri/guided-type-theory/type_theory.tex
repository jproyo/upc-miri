\documentclass[12pt, a4paper]{article}
\usepackage[utf8]{inputenc}
\usepackage{amsmath}
\usepackage{amsfonts}
\usepackage{amsthm}
\usepackage{array}
\usepackage{graphicx}
\usepackage{parskip}
\usepackage[pdfencoding=auto]{hyperref}
\usepackage{fancyhdr}
\usepackage{lastpage}
\usepackage{tikz}
\usepackage{float}
\usepackage{listings}
\usepackage{color}
\usepackage{caption}
\usepackage{authblk}
\usepackage{longtable}
\usepackage[acronym]{glossaries}
\usepackage[nottoc]{tocbibind}
\usepackage[cache=false]{minted}
\usemintedstyle{default}
\newminted{haskell}{frame=lines,framerule=2pt}
\newminted{R}{frame=lines,framerule=2pt}
\graphicspath{{./images/}}

\tikzstyle{bag} = [align=center]

\title{%
      A Report of \textit{Type Theory and Formal Proof}
}
\author{Juan Pablo Royo Sales}
\affil{Universitat Politècnica de Catalunya}
\date\today

\pagestyle{fancy}
\fancyhf{}
\fancyhead[C]{}
\fancyhead[R]{UPC MIRI}
\fancyhead[L]{SIRI - Guided Work}
\fancyfoot[L,C]{}
\fancyfoot[R]{Page \thepage{} of \pageref{LastPage}}
\setlength{\headheight}{15pt}
\renewcommand{\headrulewidth}{0.4pt}
\renewcommand{\footrulewidth}{0.4pt}

\newacronym{lc}{$\lambda$-\textit{calculus}}{Lambda Calculus}
\newacronym{br}{$\beta$-\textit{reduction}}{Beta Reduction}

\newtheorem{hyp}{Hypothesis}

\begin{document}

\maketitle

\tableofcontents

\section{Introduction}
This report is going to provide a summary over the book~\cite{type_theory}.
Alongside the different chapters of the book I am going to describe briefly the most important parts of each chapter and, at the same time,
I am going to solve 1 or 2 of the exercises proposed by the authors.

The organization of the report is going to be the same as the chapters of the book.

\section{Untyped lambda calculus}
In this first chapter the authors define and describe \acrfull{lc} system which encapsulates the formalization of basic aspects
of mathematical functions, in particular construction and use. In \acrshort{lc} formalization system there are \textit{typed} and \textit{untyped} 
formalization of the same system. In this first case authors introduced the first basic and simple formalization which is \textit{untyped}.

\subsection{Definition}
There are \textit{two constructions principles} and \textit{one evaluation rule}

\textbf{Construction principles:}

\begin{itemize}
    \item \textit{Abstraction:} Given an expression $M$ and a variable $x$ we can construct the expression: $\lambda x.M$. This is abstraction of $x$ over $M$
    Example: $\lambda y.(\lambda x. x - y)$ Abstraction of $y$ over $\lambda x. x - y$
    \item \textit{Application:} Given 2 expressions $M$ and $N$ we can construct the expression: $M\ N$. This is the application of $M$ to $N$.
    Example: $(\lambda x.x^2 + 1)(3)$ Application of $3$ over $\lambda x.x^2 + 1$
\end{itemize}

\textbf{Evaluation Rule:} Formalization of this process is called \acrfull{br}. Makes use of substitution, expressed by $[\ ]$.
\acrshort{br}: An expression $(\lambda x. M)N$ can be rewritten to $M[x := N]$, which means every $x$ should be replaced by $N$ in $M$. This process 
is called \acrshort{br} of $(\lambda x. M)N$ to $M[x := N]$.

Example: $(\lambda x. x^2 + 1)(3)$ reduces to $(x^2 + 1)[x := 3]$, which is $3^2 + 1$.

\bibliographystyle{alpha}
\bibliography{type_theory}

\end{document}