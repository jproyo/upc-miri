\documentclass[12pt, a4paper]{article}
\usepackage[utf8]{inputenc}
\usepackage{amsmath}
\usepackage{amsthm}
\usepackage{graphicx}
\usepackage{parskip}
\usepackage{hyperref}
\usepackage{fancyhdr}
\usepackage{lastpage}
\title{Home work 5 - Solutions}
\author{Juan Pablo Royo Sales}
\date\today

\pagestyle{fancy}
\fancyhf{}
\fancyhead[C]{}
\fancyhead[R]{Juan Pablo Royo Sales - UPC MIRI}
\fancyhead[L]{RA - Homework 5}
\fancyfoot[L,C]{}
\fancyfoot[R]{Page \thepage{} of \pageref{LastPage}}
\renewcommand{\headrulewidth}{0.4pt}
\renewcommand{\footrulewidth}{0.4pt}

\begin{document}

\maketitle

\section{Exercise 24}
\subsection{Part a}

\begin{align*}
  P[X = 1 \land Y = 2] &= P[X = 1 \cap Y = 2]\\
                       &= P[X = 1] P[Y = 2]\\
                       &= e^{-1} e^{-2} 2 \\
                       &= 0.099
\end{align*}

\subsection{Part b}

\begin{align*}
  P[\frac{X+Y}{2} \geq 1] &= 1 - P[X+Y < 2]\\
                          &= 1 - P[X+Y = 0] + P[X+Y = 1]\\
                          &= 1 - \frac{e^{-3} 3^0}{0!} + \frac{e^{-3} 3^1}{1!}\\
                          &= 0.80
\end{align*}

\subsection{Part c}

\begin{align*}
  P[X = 1 \mid \frac{X+Y}{2} \geq 2] &= \frac{P[Y \geq 3]}{P[X+Y \geq 4]}\\
                                     &= \frac{1 - P[Y = 0] + P[Y = 1] + P[Y = 2]}{1 - P[X+Y=3] + P[X+Y=2] + P[X+Y=1] + P[X+Y=0]}\\
                                     &= \frac{1 - 0.13 + 0.27 + 0.27}{1 - 0.22 + 0.22 + 0.149 + 0.049}\\
                                     &= 0.911
\end{align*}

\section{Exercise 25}
\subsection{Part a}

\begin{align*}
  E[3X + 5] &= 3E[X] + 5\\
            &= 3\lambda + 5
\end{align*}

\subsection{Part b}

\begin{align*}
  Var[3X + 5] &= 9 Var[X]\\
              &= 9 \lambda (1 - p)
\end{align*}

\subsection{Part c}

\begin{align*}
  E[\frac{1}{1+X}] &= \sum_{i=0}^{n} \frac{1}{1+x_i} \frac{e^{-\lambda} \lambda^i}{i!}\\
                   &= \frac{e^{-\lambda}}{\lambda} \sum_{i=0}^{n} \frac{\lambda^{i+1}}{(i+1)!}\\
                   &= \frac{e^{-\lambda}}{\lambda} \sum_{i=0}^{n} \frac{\lambda^i}{i!} - 1\\
                   &= \frac{1 - e^{-\lambda}}{\lambda}
\end{align*}

\section{Exercise 26}
\subsection{Part a}

Since we have the following conditional probability $P[X_1 = 1 \mid X_1 = 1 \lor
X_2 = 1 \lor X_3 = 1]$, we know that we have 1 ball at least in $X_1$ or $X_2$
or $X_3$ bean, given by the condition probability. Therefore the $P[X_1 = 1 \mid
\text{It is in 1, 2 o 3}] = \frac{1}{3}$ 

\subsection{Part b}
Since $X_2 = 0$, we still have $m-1$ balls and $n-1$ bins. Therefore $E[X_1=1] = P[X_1 = 1] = \frac{m-1}{n-1}$

\subsection{Part c}

\begin{itemize}
  
\item The probability that bin 2 gets exactly $i-1$ balls is
  $\binom{n}{i-1}\frac{1}{n^{i-1}}(\frac{n-1}{n})^{n-i+1}$. \\

  The probability that bin 1 gets at least $i$ balls, given that bin 2 gets
  exactly $i-1$ balls is: $\binom{n-i+1}{i}\frac{1}{n^{i}}$. \\

  Therefore,

  \begin{align*}
    \sum_{i=1}^{n/2}\binom{n}{i-1}\frac{1}{n^{i-1}} (\frac{n-1}{n})^{n-i+1}\binom{n-i+1}{i}\frac{1}{n^{i}}
  \end{align*}

\item The probability that bin 2 gets exactly $i$ balls is
  $\binom{n}{i}\frac{1}{n^{i}}(\frac{n-1}{n})^{n-i}$.\\

  The probability that bin 1 gets exactly $i$ balls, given that bin 2 gets
  exactly $i$ balls is: $\binom{n-i}{i}\frac{1}{n^{i}}(\frac{n-1}{n})^{n-i}$.

  Therefore,

  \begin{align*}
    0.5(1-\sum_{i=1}^{n/2}\binom{n}{i}\frac{1}{n^{i}}(\frac{n-1}{n})^{n-i}\binom{n-i}{i}\frac{1}{n^{i}}(\frac{n-1}{n})^{n-i})
  \end{align*}


\end{itemize}

\section{Exercise 27}
\subsection{Part a}

\begin{subequations}
  \begin{align}
    E[X] &= \sum_{i=0}^m 1 \binom{n}{2} \frac{1}{m} \\
         &= \binom{n}{2}
  \end{align}
\end{subequations}

\subsection{Part b}
\begin{subequations}
  \begin{align}
    P[|Z - \mu| \geq c\sqrt{\mu}] &\leq \frac{Var[Z]}{(c\sqrt{\mu})^2}\\
                                  &\leq \frac{\mu}{c^2\mu}\\\label{eq:3}
                                  &\leq \frac{1}{c^2}
  \end{align}
\end{subequations}

\ref{eq:3} because $E[Z] = Var[Z]$ when p is small 

\subsection{Part c}

Let $Y$ be a r.v. counting the number of balls in bin $i$

Therefore applying Chernoff and union bound we have:

\begin{subequations}
  \begin{align}
    P[Y \geq 2] &= P[Y \geq (1 + (\frac{2m}{n} - 1))E[Y]]\\
                &\leq e^{-\frac{(\frac{2m}{n} - 1)^2(\frac{n}{m})}{2 + (\frac{2m}{n} - 1)}}\\
                &\leq e^{-\frac{(2m - n)^2}{m(n + 2m)}} \sim \theta(1)
  \end{align}
\end{subequations}



\section{Exercise 28}
\subsection{Part a}
We know by Chebyshev's that $P[|Y - \mu| \geq b] \leq \frac{Var[Y]}{b^2}$

Lets calculate then $Var[Y]$

\begin{subequations}
  \begin{align}
  Var[Y] &= \sum_{i=0}^{m-1} Var[Y_i]\\
         &= \sum_{i=0}^{m-1} \frac{1 - p_i}{p_i^2}\\
         &= \sum_{i=0}^{m-1} \frac{mi}{(m - i)^2}\\
         &= \sum_{j=1}^{m} \frac{m(m - j)}{j^2}\label{eq:1} \\
         &= m^2 \sum_{j=1}^{m} \frac{1}{j^2} - m \sum_{j=1}^{m} \frac{1}{j}\label{eq:2}\\
         &= \frac{m^2 \pi^2}{6} - m\ln{m}
  \end{align}
\end{subequations}

Replacing index $i$ by $j$ starting in 1 \ref{eq:1}\\
According to statement of the problem \ref{eq:2}\\

Applying Chebyshev's

\begin{align*}
  P[|Y - \mu| \geq c \mu] &\leq \frac{\frac{m^2 \pi^2}{6} - m\ln{m}}{c^3m^2{\ln{m}}^2}\\
                          &\leq \frac{m^2 \pi^2}{6c^2Hm}
\end{align*}

\subsection{Part b}
Let $Y$ be the number of balls in bin $i$. We know by definition that $E[Y] =
\ln{n} + c$ for some constant $c$

Therefore by Chernoff bound we have:

\begin{align*}
  P[Y \leq 0] &= P[Y \leq (1 - 1)E[Y]]\\
              &\leq e^{\frac{-E[Y]}{2}}\\
              &\leq e^\frac{-\ln{n} + c}{2}
\end{align*}

Using union bound we have that $P[\text{Some bin is empty}] \leq nP[Y \leq 0] =
e^{\ln{n} - \frac{\ln{n} + c}{2}}$, therefore we can set $c = \ln{n} + 2\ln{(1/\delta)}$

\subsection{Part c}
We can assure that this 2 bounds are much more tight that the one that we show
in class, because for any value of $m$

\begin{align*}
  e^\frac{-\ln{n} + c}{2} \ll n^{1-c}
\end{align*}



\end{document}
