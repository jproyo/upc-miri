\documentclass[12pt, a4paper]{article}
\usepackage[utf8]{inputenc}
\usepackage{amsmath}
\usepackage{amsthm}
\usepackage{amssymb}
\usepackage{graphicx}
\usepackage{parskip}
\usepackage{hyperref}
\usepackage{fancyhdr}
\usepackage{lastpage}
\usepackage[vlined,ruled]{algorithm2e}
\usepackage[acronym]{glossaries}
\usepackage{caption}

\title{%
  Computational Complexity \\
  Homework 4 - Solutions
}
\author{%
  Juan Pablo Royo Sales\\
  \small{Universitat Politècnica de Catalunya}
}
\date\today

\pagestyle{fancy}
\fancyhf{}
\fancyhead[C]{}
\fancyhead[R]{Juan Pablo Royo Sales - UPC MIRI}
\fancyhead[L]{CC - Homework 4}
\fancyfoot[L,C]{}
\fancyfoot[R]{Page \thepage{} of \pageref{LastPage}}
\setlength{\headheight}{15pt}
\renewcommand{\headrulewidth}{0.4pt}
\renewcommand{\footrulewidth}{0.4pt}

\renewcommand{\qedsymbol}{$\blacksquare$}
\newacronym{bpp}{BPP}{BPP}

\begin{document}

\maketitle

\section{Exercise 1 - More on hash functions}
\newtheorem{3wise}{3-wise independent}
\begin{3wise}
  $Pr_{a,b,c \in \mathbb{Z}_p} \left[ P_{a,b,c}(x_i) = y_i, for\ i = 1,2,3 \right] = \frac{1}{p^3}$
\end{3wise}

\begin{proof}
  Let $x_1 \neq x_2 \neq x_3 \in \mathbb{Z}_p$ and $y_1 \neq y_2 \neq y_3 \in \mathbb{Z}_p$.
  \begin{subequations}
    \begin{align}
      Pr_{a,b,c \in \mathbb{Z}_p} &= Pr \left[ \begin{bmatrix}
        x_{1}^2 & x_1 & 1\\
        x_{2}^2 & x_2 & 1\\
        x_{3}^2 & x_3 & 1
        \end{bmatrix}
        \begin{bmatrix}
        a\\
        b\\
        c
        \end{bmatrix}
        =
        \begin{bmatrix}
        y_1\\
        y_2\\
        y_3
        \end{bmatrix}
      \right] \\
      &= Pr \left[ \frac{1}{det(A)}Adj(A) \times \begin{bmatrix}
        x_{1}^2 & x_1 & 1\\
        x_{2}^2 & x_2 & 1\\
        x_{3}^2 & x_3 & 1
        \end{bmatrix}
        \begin{bmatrix}
        a\\
        b\\
        c
        \end{bmatrix}
        = \frac{1}{det(A)}Adj(A) \times
        \begin{bmatrix}
        y_1\\
        y_2\\
        y_3
        \end{bmatrix}
      \right] \label{eq:1}\\
      &= Pr \left[\begin{bmatrix}
        a\\
        b\\
        c
        \end{bmatrix}
        = \frac{1}{det(A)}Adj(A) \times
        \begin{bmatrix}
        y_1\\
        y_2\\
        y_3
        \end{bmatrix}
      \right] \label{eq:2}\\
      &= \frac{1}{p^3}\label{eq:3}
    \end{align}
  \end{subequations}

  Since we know that the $det(A) \neq 0$~\ref{eq:1} because $x_1 \neq x_2 \neq x_3$, where

  \begin{equation}
    A = \begin{bmatrix}
        x_{1}^2 & x_1 & 1\\
        x_{2}^2 & x_2 & 1\\
        x_{3}^2 & x_3 & 1
      \end{bmatrix}
  \end{equation}

  , we can compute the inverse of $A$.
  Because $a, b, c$ are choose \textit{u.a.r.} and are in $\mathbb{Z}_p$, therefore each option has a probability $\frac{1}{p}$, so the total probability by independence is~\ref{eq:3}
\end{proof}

\section{Exercise 2 - More on estimating size of set}\label{sec:2}

\newtheorem{2k}{Problem}\label{th:1}
\begin{2k}
  If $|S| > 2^k \implies Pr_{h \in_r U} = \left[ I(S,h) = 1 \right] = 1$
\end{2k}

\begin{proof}
  Knowing that $2$-Universal hashing is $P_{h \in_r U} [h(x) = h(y)] \leq \frac{1}{2^k}, \forall x,y \in \{0,1\}^k$, we take all the possible pairs divided by the probability of collision

  \begin{subequations}
    \begin{align}
      P[I =  1] &\leq \frac{\binom{2^k}{2}}{2^k} \\
      &= 1\label{eq:4}
    \end{align}
  \end{subequations}

  This~\ref{eq:4} is because we know that the numerator is bigger than the denominator.
\end{proof}

\begin{2k}
  If $|S| \leq 2^{k/2} \implies Pr_{h \in_r U} = \left[ I(S,h) = 1 \right] \leq 1/2$
\end{2k}

\begin{proof}
  We know that if $M = N^2$ where $N = |S|$ and $M$ is the size of $\mathcal{H}$ then $P[\textit{ collision in } N] = \frac{\binom{N}{2}}{M} \leq 1/2$, because we are dividing all possibilities of pairs in the set by the total number of possible collisions.

 According to the state of our problem we know that $(2^{k/2})^2 = 2^k$, so in our case our $M$ which is $2^k$ (the amount of possible collisions), is the square of the $N = 2^{k/2}$ which is the size of $S$.

Therefore $P[I(S,h)=1] \leq 1/2$ when $|S| \leq 2^{k/2}$

\end{proof}

\section{Exercise 3 - Approximate counting up to square}
\newtheorem{a-counting}{Problem}
\begin{a-counting}
  For any $\#P$ function $f$ there exists a \textit{deterministic} polynomial time algorithm that given an $n$-bit input $x$ and access to an $NP$-oracle outputs a number of $t$ satisfying that $t \leq f(x) \leq t^2$
\end{a-counting}

\begin{proof}
  Lets do the proof for $\#3SAT$ with access to $NP$-oracle and this can be state for any $\#P$ by parsimonious reduction.

  Let $A$ be the approximation algorithm with a constant factor $c$ such that:

  \begin{equation}
    \frac{1}{c}\#3SAT(\varphi) \leq A(\varphi) \leq c\#3SAT(\varphi)
  \end{equation}

  $\varphi$ is constructed with doing a copy of each $\phi$, $\varphi^t = \phi_1 \land \phi_2 \land \dots \phi_t$, so if $\phi$ has a $k$ number of satisfying assignments $\varphi^t$ is going to have a $k^2$, then:

Moreover,

$$
\begin{cases}
  \textit{if} \quad \#3SAT(\varphi) \geq 2^{k+1} \implies A(\varphi,k) = 1 \quad \textit{w.h.p.}\\
  \textit{if} \quad \#3SAT(\varphi) < 2^k \implies A(\varphi,k) = 0 \quad \textit{w.h.p.}
\end{cases}
$$

Using this approximation algorithm $A$ we can construct a function $f \in \#3SAT$ that has access to an $NP$-oracle machine in the following way:

\begin{algorithm}[H]
  \SetKwInOut{Input}{Input}
  \SetKwInOut{Output}{Output}
  \Input{$\varphi$}
  \Output{$2^k$}
  Let $k \leftarrow 0$;\\
  Let $t \leftarrow 0$;\\
  Let $c \leftarrow 0$;\tcp*[f]{Indicate that evaluation on $A$ changed}\\
  \For{$i \leftarrow 0$ \KwTo $n+1$}
  {$r \leftarrow A(\varphi, i)$;\\
    \eIf{$r = 1 \land c = 0$}{$t \leftarrow t + 1$}{$c \leftarrow 1$}
  }
  \eIf{output is $0$ for each computation from $i = 0$ to the end}
  {Query $NP$-oracle for exact value to output;
  }{Output $2^t$}
 \caption{$\#3SAT$ with access to $NP$-oracle}
\end{algorithm}

With this we have that:

\begin{equation}
  \frac{1}{c}2^k \leq \#3SAT(\varphi) \leq c2^{k}
\end{equation}

Lets find a good $A$ approximation algorithm to square upper bound.

Given the previous exercise~\ref{sec:2}, we are going to use a big set $|S|$ to evaluate $\varphi \land h(x) = 0$ and applying \textit{Valiant-Vazirani Theorem} we are going to be sure that we have a good probability that $\varphi$ has a satisfying assignment.

The algorithm with proceeds as follows:

\begin{algorithm}[H]
  \SetKwInOut{Input}{Input}
  \SetKwInOut{Output}{Output}
  \Input{$\varphi, k$}
  \Output{$1$ if there is $k$ valid assignments, $0$ otherwise}
  \eIf{$k < 5$ is small}
  {Check $\#3SAT(\varphi) \geq 2^{k}$
  }
  {Pick $h$ from a set of pairwise independent hash function as~\ref{sec:2};\\
    Output $1$ if $\varphi(a) = 1 \land h(a) = 0$
  }
  \caption{$A(\varphi, k)$ algorithm}
\end{algorithm}

Taking $\varphi^2 = \phi \land \phi$ we can enlarge the Set $|S|$ having a probability $1$ of collision as it is referred~\ref{sec:2}.

Therefore,

\begin{equation}
  ct \leq A(\varphi^2,k) \leq ct^2
\end{equation}

\end{proof}

\end{document}

