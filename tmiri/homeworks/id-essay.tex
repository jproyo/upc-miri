\documentclass[12pt, a4paper]{article}
\usepackage[utf8]{inputenc}
\usepackage{amsmath}
\usepackage{amsthm}
\usepackage{graphicx}
\usepackage{blindtext}
\usepackage{parskip}
\usepackage{hyperref}
\usepackage{color}
\title{
  Essay on Ethics\\
  \large{\textit{Identity inference of genomic data using long-range familial searches}}\\
  \date\today\vspace{-2em}}
\date{\normalize\today}

  
\begin{document}

\maketitle

The analyzed article describe how the use of \textbf{genomic} identification
techniques through long-range familiar search are being used to trace suspects
on crime investigations.

These type of investigations have been conducted using DNA third party services,
such as \textit{DNA.Land and GEDmatch}, which allows any person to upload a raw
genetic profile in a digital plain format offering the option to find relatives
by locating identity-by-descendent segment that can indicate a shared ancestor.

Using this kind of services law enforcement agencies have been able to identify
different suspects of unsolvable past crimes. Although some of them
led to a successful result, mainly in cases that have been opened a for a long
time, this kind of analysis has been also used for resolution on recent cases.

Basically the paper analyze the accuracy of these method to find suspects. The
study concludes that although in some cases doing a deeper investigation without
resting on this techniques would lead to a successful results, there have been
some crimes in which without conducting a \textbf{genetic} test of this
characteristics there wouldn't have been possible to find the criminal.

In spite of the study and its results it is clear that there is an ethical
conflict that emerge.

On the one hand, we have a public database provided by these services which is
exposing a lot of people data to be linked with possible suspects. As it is shown in
the study, if we take the \textbf{US} population which is the one under
analysis, and we conduct a search for relatives on this databases, there is a
high probability that found in the database a 3-cousin relative or even higher. If we are looking
closer relatives the probability decreases but it is low enough to discard the
subject. This means that there is going to be always some relative to be found
in the database and perhaps link that person or people is too invasive in terms
of data privacy.

Secondly, and this is something that is taken into consideration at the end of
the study, genetic profiles uploaded to these platforms are open and plain. This
is a huge security issue in terms of data manipulation and source of truth.

Another controversial point is that all this data and techniques are conducted
by private companies that most of the time are not 100\% independent.

In spite of all this, there are quite positive sides on this approach to find
criminals; law enforcement agencies are using everything at their disposal to
solve crimes and bring the true to cases where couldn't have been possible to
solve with traditional methods.

Another positive aspect of using genomic public database is that the accuracy of
this techniques combined with crime investigation theories and data, are leading to
stronger conclusions and evidence.

Having exposing the paper and the different aspects involved, i think that the
use of these techniques resting on private companies with public data access is
controversial in different manners.

Private sector are always looking for profit because it is their ultimate goal,
and sometimes in that seek for better margin if they are not restricted by
strong \textbf{Data Privacy Law}, there is no protection for regular citizens. As
we have seen on the study, a lot of genetic profiles are being used as a way to
link and trace crime suspects and there is no regulation on \textbf{US} to
protect genetic consumers on this because it is not consider an identity asset by law.

It is even more controversial knowing the fact that the genetic profile uploaded
to be analyzed is in plain format without any encryption or signature
verification at all. Any malicious agent in the process of the analysis or event
before the submission can modify these profiles intentionally or not, leading to
uncertain or obscure results. If this is possible, some people can lose their
freedom or be condemned unfairness. 

The only positive side of this method and procedure is the relief of some
victims that are waiting for the true for a lot of years and their only hope is
trying all the possible options available to have solve their cases.

In conclusion and in spite of the positive side of using genomic techniques for
finding criminals, i would feel more comfortable if this procedures are
conducting in the frame of a strong legislation that protects everyone: victims,
suspects and genomic consumers.





\end{document}
