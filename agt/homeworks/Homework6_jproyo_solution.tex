\documentclass[12pt, a4paper]{article}
\usepackage[utf8]{inputenc}
\usepackage{amsmath}
\usepackage{amsthm}
\usepackage{amssymb}
\usepackage{graphicx}
\usepackage{parskip}
\usepackage{hyperref}
\usepackage{fancyhdr}
\usepackage{lastpage}
\usepackage[vlined,ruled]{algorithm2e}
\usepackage[acronym]{glossaries}
\usepackage{caption}
\usepackage{titlesec}
  

\title{%
  Algorithmic Game Theory \\
  Homework 6 - Cooperative Games
}
\author{%
  Juan Pablo Royo Sales\\
  \small{Universitat Politècnica de Catalunya}
}
\date\today

\pagestyle{fancy}
\fancyhf{}
\fancyhead[C]{}
\fancyhead[R]{Juan Pablo Royo Sales - UPC MIRI}
\fancyhead[L]{AGT - Homework 6}
\fancyfoot[L,C]{}
\fancyfoot[R]{Page \thepage{} of \pageref{LastPage}}
\setlength{\headheight}{15pt}
\renewcommand{\headrulewidth}{0.4pt}
\renewcommand{\footrulewidth}{0.4pt}

\renewcommand{\qedsymbol}{$\blacksquare$}
\newacronym{bpp}{BPP}{BPP}

\begin{document}

\maketitle

\section{Problem 10}
\subsection{Question a}
\subsubsection{Monotone}
In order to probe if this game is \textit{monotone} we need to probe that $v(C) \leq v(D)$ for any $C,D$ such that $C \subseteq D$.
Since we have a case function we are going to analyze the different parts. 

Let have a graph $G_x = (V, X)$. Since by definition the players are the Edges, we can add player to $X$, so we are adding edges $Y$ to the graph $G_x$, such that
$X' = X \cup Y$. Lets call this new graph $G_x' = (V', X')$.

\begin{itemize}
  \item If $G_x'$ is connected then $v(X') = 2|X'| - \text{diam}(G_x')$. Since we are adding more edges $|X'| > |X|$, and the diameter is at least greater or equal $diam(G_x') \geq diam(G_x)$ because it is connected and by definition of diameter if we are adding edges the greatest length of the shortest path cannot be smaller with more edges.
  \item If $G_x'$ is not connected also $\frac{|X'|}{2} > \frac{|X|}{2}$.
\end{itemize}

Therefore for any $G_x \subseteq G_x'$, $v(X) \leq v(X')$, so it is \textbf{Monotone}.

\subsubsection{Superadditive}
A game is Superadditive if $v(C \cup D) \geq v(C) + v(D)$ for any 2 disjoint coalitions $C$ and $D$.

Lets analyize each case. 
Lets take if $G_{X \cup Y}$ is connected, then 

\begin{subequations}
  \begin{align}
    v(X \cup Y) &= 2(|X| + |Y|) - diam(G_{X \cup Y})\\
                &= 2|X| + 2|Y| - diam(G_{X \cup Y})
  \end{align}
\end{subequations}

And $2|X| + 2|Y| - diam(G_{X \cup Y}) \geq 2|X| - diam(G_X) + 2|Y| - diam(G_Y)$ because $diam(G_{X \cup Y}) \leq diam(G_X) + diam(G_Y)$ since it is included in one of the 2.

Lets take if $G_{X \cup Y}$ is not connected, then 

\begin{subequations}
  \begin{align}
    v(X \cup Y) &= \frac{|X| + |Y|}{2}\\
                &= \frac{|X|}{2} + \frac{|Y|}{2}\\
                &= v(X) + v(Y)
  \end{align}
\end{subequations}

Therefore it is superadditive.

\subsubsection{Supermodular}
A game is Supermodular if $v(C \cup D) + v(C \cap D) \geq v(C) + v(D)$.

By the previous item we have probe that it is superadditive, so if we sum $v(C \cap D)$ to the union, we are still going to have something
greater or equal of the sum of the individual valuation functions, therefore it is also supermodular.

\subsection{Question b}
Yes, a graph of only 1 edge because. Let suppose that that graph is $A = (Z, Y)$ where $Z={z,v}$ and $Y = {z,v}$. 
Lets assume that we take $X = Y$, then

\begin{itemize}
  \item $v(X) = 2 |X| - diam(A_X) = 2 \times 1 - 1 = 1$ 
  \item Also we have that $x(X) = 1$ because we have 1 edge.
  \item Therefore the core is not empty.
\end{itemize}

\section{Problem 16}
In order to probe if it is $NP-\text{Complete}$ we need to first show that it is $NP-\text{hard}$.

Lets call $X$ the votes that are not in $M$.

Lets consider every \textit{pairwise} competition and we could see that every \textit{pairwise} is determined without $M$ except for the pairwise $a$ and $b$

\begin{itemize}
  \item $d$ wins to $a$ and $b$ in first row $2K + 1$
  \item $a$ and $b$ wins to $c$
  \item $c$ wins to $d$
\end{itemize}

If there is someone who wins between $a$ and $b$ pairwise, that one either $a$ or $b$ will tie with $d$.
So, taking into consideration the previous statement $d$ wins \textbf{Copeland} manipulation election $\iff a \land b$ tie in their \textit{pairwise}.
But, after the votes in $X$ $a$ and $b$ are tied. In order to maintain this tie we need that the \textbf{combined weights} $k_i \in M$ such that $\{b \mid b\ P\ a\}$ ($b$ is preferred over $a$) 
is the same as \textbf{combined weights} $k_i \in M$ such that $\{a \mid a\ P\ b\}$ ($a$ is preferred over $b$).

Since there is a $PARTITION$ instance $(k1,\dots, k_n)$ with $\sum_{i=1}^n = 2K$, so $M$ can be balance and preserved the tie.

Therefore since $PARTITION$ is $NP-\text{Complete}$, \textit{Copeland-CWM} is also $NP-\text{Complete}$.

\end{document}