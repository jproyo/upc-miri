\documentclass{beamer}
\usepackage[T1]{fontenc}
\usepackage[utf8]{inputenc}
\usepackage{amsmath}
\usepackage{amsfonts}
\usepackage{amsthm}
\usepackage{array}
\usepackage{graphicx}
\usepackage{listings}
\usepackage{color}
\usepackage{fancyhdr}
\usepackage{tikz}
\usepackage{float}
\usepackage{caption}
\usepackage[linguistics]{forest}
\usepackage[cache=false]{minted}
\usemintedstyle{default}
\newminted{haskell}{frame=lines,framerule=2pt}
\graphicspath{{./images/}}
\setbeamertemplate{frametitle}{%
  \usebeamerfont{frametitle}\insertframetitle%
  \vphantom{g}% To avoid fluctuations per frame
  % \hrule% Uncomment to see desired effect, without a full-width hrule
  \par\hspace*{-\dimexpr0.5\paperwidth-0.5\textwidth}\rule[0.5\baselineskip]{\paperwidth}{0.4pt}
  \par\vspace*{-\baselineskip}% <-- reduce vertical space after rule
}

\usetheme{Boadilla}
\title{Analysis on Random Access Zippers}

\author{Juan Pablo Royo Sales}
\institute{Universitat Politècnica de Catalunya}
\date{June $11^{th}$, 2020}
\begin{document}

\begin{frame}
\titlepage
\end{frame}

\begin{frame}{Agenda}
  \tableofcontents
\end{frame}

\begin{frame}{Agenda}
  \section{Preliminaries}
  \tableofcontents[currentsection]
\end{frame}

\begin{frame}[fragile]{Preliminaries}

  \begin{block}{Why I choose this structure?}
    \begin{itemize}
      \item Interested on Zippers as \textbf{FP} Data Structure
    \end{itemize}
  \end{block}

\end{frame}

\begin{frame}[fragile]{Preliminaries}

  \begin{block}{Why I choose this structure?}
    \begin{itemize}
      \item Interested on Zippers as \textbf{FP} Data Structure
      \item Type Level correspondence with Derivatives - \textbf{Math reasoning!!!}
      \end{itemize}
  \end{block}

\end{frame}

\begin{frame}[fragile]{Preliminaries}

  \begin{block}{Why I choose this structure?}
    \begin{itemize}
      \item Interested on Zippers as \textbf{FP} Data Structure
      \item Type Level correspondence with Derivatives - \textbf{Math reasoning!!!}
      \end{itemize}
  \end{block}

  \begin{block}{Hint}


   a \footnote{Mcbride, Conor. (2009). The Derivative of a Regular Type is its Type of One-Hole Contexts (Extended Abstract).}
  \end{block}

\end{frame}



\begin{frame}{Agenda}
  \section{RAZ - Structure and Features}
  \tableofcontents[currentsection]
\end{frame}

\begin{frame}
  \frametitle{Agenda}
  \section{Haskell Implementation}
  \tableofcontents[currentsection]
\end{frame}

\begin{frame}
  \frametitle{Agenda}
  \section{Experiments Reproduction}
  \tableofcontents[currentsection]
\end{frame}

\begin{frame}
  \frametitle{Agenda}
  \section{Overall Analysis}
  \tableofcontents[currentsection]
\end{frame}

\begin{frame}
  \frametitle{Agenda}
  \section{Conclusions}
  \tableofcontents[currentsection]
\end{frame}


\begin{frame}
  \begin{center}
    \textbf{\huge{Thank you!!}}
    \end{center}
\end{frame}

\end{document}

