\documentclass[12pt, a4paper]{article}
\usepackage[utf8]{inputenc}
\usepackage{amsmath}
\usepackage{amsthm}
\usepackage{amsfonts}
\usepackage{graphicx}
\usepackage{parskip}
\usepackage{hyperref}
\usepackage{fancyhdr}
\usepackage{lastpage}
\usepackage{tikz}
\usepackage{float}
\usepackage{listings}
\usepackage{color}
\usepackage{caption}
\usepackage[acronym]{glossaries}
\usepackage[nottoc]{tocbibind}
\usepackage[cache=true]{minted}
\usemintedstyle{default}
\graphicspath{{./images/}}
\newminted{cpp}{frame=lines,framerule=2pt}
\allowdisplaybreaks[2]

\tikzstyle{bag} = [align=center]

\lstset{frame=tb,
  language=Haskell,
  aboveskip=3mm,
  belowskip=3mm,
  showstringspaces=false,
  columns=flexible,
  basicstyle={\small\ttfamily},
  numbers=left,
  numberstyle=\tiny\color{gray},
  keywordstyle=\color{blue},
  commentstyle=\color{dkgreen},
  stringstyle=\color{mauve},
  breaklines=true,
  breakatwhitespace=true,
  tabsize=2,
  stepnumber=1,
  escapechar=!
}

\definecolor{dkgreen}{rgb}{0,0.6,0}
\definecolor{gray}{rgb}{0.5,0.5,0.5}
\definecolor{mauve}{rgb}{0.58,0,0.82}

\title{%
      Combinatorial Problem Solving \\
      Final Project - Box Wrapping \\
      Linear Programming
}
\author{%
  Juan Pablo Royo Sales \\
  \small{Universitat Politècnica de Catalunya}
}
\date\today

\pagestyle{fancy}
\fancyhf{}
\fancyhead[C]{}
\fancyhead[R]{Juan Pablo Royo Sales - UPC MIRI}
\fancyhead[L]{CPS - Final Project - LP}
\fancyfoot[L,C]{}
\fancyfoot[R]{Page \thepage{} of \pageref{LastPage}}
\setlength{\headheight}{15pt}
\renewcommand{\headrulewidth}{0.4pt}
\renewcommand{\footrulewidth}{0.4pt}

\DeclareMathOperator*{\argmax}{argmax}
\DeclareMathOperator*{\st}{st}

\newacronym{lp}{LP}{Linear Programming}
\newacronym{cp}{CP}{Constraint Programming}

\begin{document}

\maketitle

\section{Linear Programming}

\subsection{Predefinitions}
Values given by the state the of the program.

\begin{itemize}
  \item $1 \leq W \leq 11$ maximum width paper roll
  \item $L$ maximum length paper roll. By definition is infinite
  \item $B = \{1, \dots, n\}$ total number of Boxes
  \item $w_b$ Width of the Box $b, b \in B$
  \item $h_b$ Height of the Box $b, b \in B$
\end{itemize}

\subsection{Formal Definition}
Given the statement of the problem proposed, we are going to define the following variables for our \acrfull{lp} solution.

\begin{subequations}
\begin{align}
  \min \quad & \sum_{b=0}^B l_b \\\label{eq:0}
 \st  \quad & \\
 \quad & x_{b=0}^{tl} = 0 \\\label{c:1}
 \quad & y_{b=0}^{tl} = 0 \\\label{c:2}
 \quad & y_b^{br} + 1 \leq l_b \quad & \forall b, 0 \leq b \leq B \\\label{c:3}
 \quad & x_b^{tl} \leq x_b^{br} \quad & \forall b, 0 \leq b \leq B \\\label{c:4}
 \quad & y_b^{tl} \leq y_b^{br} \quad & \forall b, 0 \leq b \leq B \\\label{c:5}
 \quad & x_b^{tl} + (1-r_b) \times (w_b - 1) + r_b \times (h_b-1) = x_b^{br} \quad & \forall b, 0 \leq b \leq B \\\label{c:6}
 \quad & y_b^{tl} + (1-r_b) \times (h_b - 1) + r_b \times (w_b-1) = y_b^{br} \quad & \forall b, 0 \leq b \leq B \\\label{c:7}
 \quad & b_i^{x^{br}} + b_i^{y^{br}} + b_j^{x^{tl}} + b_j^{y^{tl}} \leq 1 \quad & \forall i,j | 0 \leq i \leq j \leq B \\\label{c:8}
 \quad & x_{b=i}^{br} + 1 <= x_{b=j}^{tl} + b_i^{x^{br}} \times w_b \quad & \forall i,j\ |\ 0 \leq i \leq j \leq B \\\label{c:9}
 \quad & y_{b=i}^{br} + 1 <= y_{b=j}^{tl} + b_i^{y^{br}} \times h_b \quad & \forall i,j\ |\ 0 \leq i \leq j \leq B \\\label{c:10}
 \quad & x_{b=j}^{br} + 1 <= x_{b=i}^{tl} + b_j^{x^{tl}} \times w_b \quad & \forall i,j\ |\ 0 \leq i \leq j \leq B \\\label{c:11}
 \quad & y_{b=j}^{br} + 1 <= y_{b=i}^{tl} + b_j^{y^{tl}} \times h_b \quad & \forall i,j\ | 0 \leq i \leq j \leq B \\\label{c:12}
 \quad & 0 \leq x_b^{tl} \leq W-1 \quad & \forall b, 0 \leq b \leq B \\\label{b:1}
 \quad & 0 \leq y_b^{tl} \leq L-1 \quad & \forall b, 0 \leq b \leq B \\\label{b:2}
 \quad & 0 \leq x_b^{br} \leq W-1 \quad & \forall b, 0 \leq b \leq B \\\label{b:3}
 \quad & 0 \leq y_b^{br} \leq L-1 \quad & \forall b, 0 \leq b \leq B \\\label{b:4}
 \quad & 0 \leq l_b \leq \quad & \forall b, 0 \leq b \leq B \\\label{b:5}
 \quad & 1 \leq w_b \leq W \quad & \forall b, 0 \leq b \leq B \\\label{b:6}
 \quad & 1 \leq h_b \leq L \quad & \forall b, 0 \leq b \leq B \\\label{b:7}
 \quad & r_b \in \{0,1\} \quad & \forall b, 0 \leq b \leq B \\\label{b:8}
 \quad & b^{x^{br}}, b^{y^{br}}, b^{x^{tl}}, b^{y^{tl}} \in \{0,1\} \\\label{b:9}
\end{align}
\end{subequations}

\subsubsection{Variables}
\begin{itemize}
  \item $x_b^{tl}$: Top \textbf{left} corner $x$. Means the box $b$ is on the top left $x$-coordinate.
  \item $y_b^{tl}$: Top \textbf{left} corner $y$. Means the box $b$ is on the top left $y$-coordinate.
  \item $x_b^{br}$: Bottom \textbf{right} corner $x$. Means the box $b$ is on the bottom right $x$-coordinate.
  \item $y_b^{br}$: Bottom \textbf{right} corner $y$. Means the box $b$ is on the bottom right $y$-coordinate.
  \item $r_b$: Boolean Array that indicates if the Box $b$ is rotated or not.
  \item $l_b$: Length in the roll occupied by box $b$.
  \item $w_b$ and $h_b$ are width and height of each box $b$ and it is given by the input of the program.
  \item $b^{x^{br}}, b^{y^{br}}, b^{x^{tl}}, b^{y^{tl}}$: Boolean variable for each pair of boxes to construct exclusive OR in constrains~\ref{c:9},~\ref{c:10},~\ref{c:11} and~\ref{c:12}.
\end{itemize}

As we can see I am reusing some variables that I have used in \acrfull{cp} deliverable. I am going to do the same for Bounding and Constraints. Some of them are going to be equal to previous solution in \acrshort{cp} because they are expressed linearly.


\subsubsection{Constraints}
\begin{enumerate}

  \item Constraint~\ref{c:1}: To break column symmetry. We can set it in $0$ because I have ordered the boxes according to its area in descendant order previous the optimization.
  \item Constraint~\ref{c:2}: To break row symmetry.
  \item Constraint~\ref{c:3}: Length of the box $b$ should be greater than the bottom right coordinate.
  \item Constraint~\ref{c:4}: Top left $x$ coordinate should be less or equal than the bottom right in each $b$.
  \item Constraint~\ref{c:5}: Same~\ref{c:4} but for $y$ coordinates.
  \item Constraint~\ref{c:6}: The amount of $x$ coordinates a box $b$ occupies is its top left $x$ coordinate plus its width or height depending if the box is rotated or not.
  \item Constraint~\ref{c:7}: Same~\ref{c:6} but for $y$ coordinates
  \item Constraint~\ref{c:8}: Linear boolean restriction to ensure exclusive OR constraints~\ref{c:9},~\ref{c:10},~\ref{c:11} and~\ref{c:12}.
  \item Constraint~\ref{c:9}: Not overlap $x$ top left coordinates with bottom rights for each pair of boxes.
  \item Constraint~\ref{c:10}: Idem~\ref{c:9} but for $y$ coordinates.
  \item Constraint~\ref{c:11}: Idem~\ref{c:9} but considering that the pair is in the opposite position.
  \item Constraint~\ref{c:12}: Idem~\ref{c:11} but for $y$ coordinates.
\end{enumerate}

\subsection{Implementation Detail}
All the code is a one to one mapping with the mathematical formulation exposed before. In the only case that I haven't done this was in the case of the exclusive OR in constrains~\ref{c:9},~\ref{c:10},~\ref{c:11} and~\ref{c:12}, because expressing an OR in CPLEX is much more easier and better optimized by the engine rather than doing with \textbf{big-M} technique.

\begin{listing}[H]
  \inputminted[firstline=68, lastline=75, breaklines]{cpp}{../src/main.cc}
  \caption{Extracted from source code src/main.cc}
  \label{src:main}
\end{listing}

As we can see here~\ref{src:main}, the traditional short circuit OR does the optimization in the code.

Order are guarantee beforehand running the optimization.

\end{document}

