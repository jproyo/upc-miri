\documentclass[12pt, a4paper]{article}
\usepackage[utf8]{inputenc}
\usepackage{amsmath}
\usepackage{amsthm}
\usepackage{graphicx}
\usepackage{blindtext}
\usepackage{parskip}
\usepackage{hyperref}
\usepackage{listings}
\usepackage{color}
\definecolor{dkgreen}{rgb}{0,0.6,0}
\definecolor{gray}{rgb}{0.5,0.5,0.5}
\definecolor{mauve}{rgb}{0.58,0,0.82}
\renewcommand\refname{Bibliography}
\title{
  Review of \textit{Selective Applicative Functors}\\
  \date\today\vspace{-2em}}
\date{\normalize\today}

\lstset{frame=tb,
  language=Haskell,
  aboveskip=3mm,
  belowskip=3mm,
  showstringspaces=false,
  columns=flexible,
  basicstyle={\small\ttfamily},
  numbers=left,
  numberstyle=\tiny\color{gray},
  keywordstyle=\color{blue},
  commentstyle=\color{dkgreen},
  stringstyle=\color{mauve},
  breaklines=true,
  breakatwhitespace=true,
  tabsize=2,
  stepnumber=1,
}

\begin{document}

\maketitle

\section{Introduction}

\begin{itemize}
\item \textbf{Title:} Selective Applicative Functors
\item \textbf{Submitted To:} Proceedings of the ACM on Programming Languages
\end{itemize}

\section{Summary}
In this paper the authors have proposed a new categorical abstraction for
functional programming languages called \textit{Selective Applicative Functors}.\\
This model describes conditional and branch effectful computations which requires that the effects
are declared statically and in addition they can be selected dynamically to be executed, taking
the best properties of two well-known categorical abstractions which are \textit{Monads} and \textit{Applicative Functors}.

The main underlying motivation is to build this new categorical abstraction
which benefit from strong properties of \textbf{Monads and Applicative Functors} such as dependent
effectful computations and static analysis respectively, providing a new strong
lawful model with several combinators useful for everyday software development
in Functional Programming.

Although all the examples in the paper are presented in \textbf{Haskell} for exemplification purpose, it can be
applied to any Strong Typed Functional Programming language.

Regarding the subject, apart from analyzing some weak points on \textbf{Monads}
and \textbf{Applicative Functors} when there is a need to represent conditional
effectful computations, they present the model of this new \textbf{Type Class},
alongside with a series of combinators. Also they present the possible
implementation and hierarchy dependency with preexisting abstractions, proposing
the need to form a \textit{Applicative $->$ Selective $->$ Monad} relationship.

Moreover, after describing all different combinators, as it is usual in the Category
Theory field, they describe and state all the \textit{Laws} this new abstraction
should fulfill such as \textit{Identity, Distributivity, Associativity} and some
other that can be derived from these principal laws and parametricity
\cite{wadler_theorems}.

In addition, as in any other rigorous work on modeling, this paper shows 2 real software
implementation of this model and its results:

\begin{itemize}
  \item \textbf{Dune Build System:} Implementation of \textbf{Selective
      Functors} in \textit{Dune OCaml} package build system for doing branch static analysis of
    dependencies.
  \item \textbf{Haxl:} Haskell Facebook Framework for efficient and parallel
    data fetching. In this case \textit{Selective Applicative Functors} has been
    applied to do conditional effectful computation.  
\end{itemize}

Concluding this paper, and as an extension of the potential use of this abstraction, the
authors also develop an implementation of \textbf{Free Selective Functors}. As
it is well known in Functional Programming theory, \textit{free} abstractions
are \textit{Functors} from a category to its \textbf{F-algebras}. This
kind of constructions, and in particular in the case of the \textit{Selective
  Functors}, allow us to focus on internal aspects of the effect without needing
to define custom instance.\\
This \textbf{Free Selective Functors} analysis has been stated throughout all
this work first with a tiny example of an Internal DSL for representing a
Ping-Pong Program, and lately with a more complex problem to represent an \textbf{ISA}
(Instruction Set Architecture) simulation.

\section{Originality}
Studies on abstractions for describing effectful computations in Functional
Programming have been conducted at least for the least for the last 25 years if
we take one of the first papers from Wadler \cite{wadler_monads} about
\textbf{Monads}, but in the literature there is no previous work about
conditional or branch effectful computations as it is presented here.\\
Although there is a recognition inside this work in \textbf{Related Work}
section, that there is some intersection with \textbf{Arrows and Profunctors}
abstractions which can also represent compositions on effectful computations, it
is shown that \textbf{Selective Applicative Functors} are complementary to
\textbf{Arrows}.\\
Having saying that, the originality of this work is unquestionable.

\section{Significance}
This work is relevant for the community of Functional Programming languages in
general and also for mathematicians who study the field of Abstract Algebra and
Category Theory in particular. On the other hand it is also extremely
significant for the Industry in general that is using this family of languages
which is growing year after year, and in fact most of other's language paradigm
are incorporating Functional Programming concepts and tools more than ever.\\
Because of that it is quite suitable that this paper was presented in the most
important Research Conference on that field.


\section{Relevance}
In the field of Functional Programming languages having different alternatives
to describe effectful computations, with strong formal laws, based on Category
Theory constructions and implementations on real industrial software, is quite
relevant for the growth and maturity of the field. The methodology follow in
this work also shows the standard methodology exposed in different works on this
field providing not only the formal description but also the implementation in
at least one of the strongest and most pure functional programming language at
this moment.

\section{Validity}\label{validity}
In this work, the authors provide several evidence to reproduce everything that
it is stated there:

\begin{itemize}
  \item All the implementation of this abstraction in \textbf{Haskell} code
  \item How to incorporate this abstraction inside the current \textbf{Haskell}
    Typeclass hierarchy
  \item A list of more than 10 combinators using this abstraction with its
    implementation
  \item Real case study that are currently implemented in libraries based on
    this work both for \textbf{OCaml and Haskell}
  \item Free Selective Functors implementations for ping-pong and ISA case studies.
\end{itemize}

Although all the exposed before i would like to make some recommendations that
in my opinion should be revised by the authors to improve the work even more:

\begin{itemize}
 \item The authors state that \textit{"Many monads directly use
     \lstinline{select = selectM} in their \textbf{Selective} instances
     definitions."}. I think this is wrong because this is a new
   abstraction and it has not been any previous work on this abstraction, there
   for there is no \textbf{Monads} implementations known with \textbf{Selective}
   at this moment.\\
   I would rephrase this as \textit{"In some cases, Monads can be implemented
     using the equality \lstinline{select = selectM}"}.
 \item In section \textbf{2.3 Laws} there is a claimed about some complexity
   that they called \textit{"obscure"} in laws definition. Basically the meaning
   on associativity law which state that:

   \begin{lstlisting}[language=Haskell,label={ass}]
     x <*? (y <*? z) = (f <$> x) <*? (g <$> y) <*? (h <$> z)
       where
         f x = Right <$> x
         g y = \a -> bimap (,a) (\$a)
         h z = uncurry z
   \end{lstlisting}

   I consider the explanation of this law is quite descriptive but i would add
   that it is Associative only to the left in all the paragraph where this is
   state because otherwise tends to confusion and one can assume that it is full
   associative which is not.

\item In section \textit{6.1 Multiway Selective Functors} the authors state
     that \lstinline{branch} combinator has some performance advantages in some
     implementations like \lstinline{bindS} for Monads, assuring that with this
     the asymptotically analysis is less than $O(N)$, but they don't provide any
     rigorous analysis on why and what should be the final cost for using
     \lstinline{branch}. I would strongly recommend to either remove that part,
     do the analysis or put it in future work.
     
\end{itemize}

\section{References}
This paper contains 47 references most of them are state the art in the field
and works that has no more than 10 years old.
This paper has not been cited yet because it is brand new and has been
presented in the Last ICFP (International Conference of Functional Programming)
by ACM this year.

\section{Clarity}
I can state that the paper is very clear for people which has certain level of
knowledge in the field. Although I know Haskell professionally, exposing this
model only with this language might be one of the weakest aspect of the work
because could be a little hard to follow for people who is not so familiar with \textbf{ML} family languages.

\section{Suitability for the conference/journal}
This is a very suitable work for ICFP (International Conference of Functional
Programming) because as i mentioned above this is strictly related to that
field.

\section{Recommendation}
After my review and suggestions in \textbf{Validity section \ref{validity}}
are taken under consideration by the authors, I would strongly recommend this
paper being accepted for such prestigious Conference, because it is a high
quality work, rigorously analyzed and explored in all its aspects.

\section{Level of Confidence}
In particular, i am very confident with the topic and the field because i work
as a Haskell developer and i know several things about the literature exposed on
this work. 

\begin{thebibliography}{9}
\bibitem{wadler_theorems} 
  Phillip Wadler. 1989.\textit{Theorems for free!}. In FPCA, Vol. 89. 347-359.
\bibitem{wadler_monads} 
  P. Wadler. 1995.\textit{Monads for functional programming}. In Int'l School on
  Advanced Functional Programming. Springer, 24-52.
\end{thebibliography} 
\end{document}
