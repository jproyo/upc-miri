\documentclass{beamer}
\usetheme{Madrid}
\usepackage[T1]{fontenc}
\usepackage[utf8]{inputenc}
\usepackage{amsmath}
\usepackage{amsfonts}
\usepackage{amsthm}
\usepackage{array}
\usepackage{graphicx}
\usepackage{listings}
\usepackage{color}
\usepackage{fancyhdr}
\usepackage{tikz}
\usetikzlibrary{calc,shapes.multipart,chains,arrows}
\usepackage{float}
\usepackage{caption}
\usepackage[linguistics]{forest}
\setbeamertemplate{frametitle}{%
  \usebeamerfont{frametitle}\insertframetitle%
  \vphantom{g}
  \par\hspace*{-\dimexpr0.5\paperwidth-0.5\textwidth}\rule[0.5\baselineskip]{\paperwidth}{0.4pt}
  \par\vspace*{-\baselineskip}
}
\setbeamertemplate{footline}
{
  \leavevmode%
  \hbox{%
  \begin{beamercolorbox}[wd=.333333\paperwidth,ht=2.25ex,dp=1ex,center]{author in head/foot}%
    \usebeamerfont{author in head/foot}\insertauthor
  \end{beamercolorbox}%
  \begin{beamercolorbox}[wd=.333333\paperwidth,ht=2.25ex,dp=1ex,center]{title in head/foot}%
    \usebeamerfont{title in head/foot}\insertshortinstitute
  \end{beamercolorbox}%
  \begin{beamercolorbox}[wd=.333333\paperwidth,ht=2.25ex,dp=1ex,right]{date in head/foot}%
    \usebeamerfont{date in head/foot}\insertshortdate{}\hspace*{2em}
    \insertframenumber{} / \inserttotalframenumber\hspace*{2ex} 
  \end{beamercolorbox}}%
  \vskip0pt%
}

\usetheme{Boadilla}

\title{Boolean Combinations of Weighted Voting Games}

\author{Juan Pablo Royo Sales}
\institute{Universitat Politècnica de Catalunya}
\date{January 2020}
\begin{document}

\begin{frame}
\titlepage
\end{frame}
  
\begin{frame}[fragile]{Agenda}
  \tableofcontents
\end{frame}

\begin{frame}[fragile]{Agenda}
  \section{Introduction}
  \tableofcontents[currentsection]
\end{frame}

  
\begin{frame}[fragile]{Introduction}
  \begin{block}{Basic Notions}
    \begin{itemize}
      \item Based on \textit{Boolean Combinations of Weighted Voting Games paper} \textit{\textbf{BWVG}}\footnotemark
      \item It is a natural Generalization over \textbf{Weighted Voting Games}
      \item Intuitively is a decision making process via multiple committees
      \item Each committee has the authority to decide the outcome \textit{"yes" or "no"} about an issue.
      \item And each committee is a WVG 
      \item Individuals can appear in multiple committees
      \item Different committees can have different Threshold values
    \end{itemize}
  \end{block}
  \footnotetext{\tiny{Piotr Faliszewski, Edith Elkind, and Michael Wooldridge. 2009. Boolean combinations of weighted voting games. In Proceedings of The 8th International Conference on Autonomous Agents and Multiagent Systems - Volume 1 (AAMAS '09). International Foundation for Autonomous Agents and Multiagent Systems, Richland, SC, 185–192.}}
\end{frame}

\begin{frame}[fragile]{Introduction}
  \begin{block}{Questions to be answered?}
    \begin{itemize}
      \item Which coalitions might be able to bring the goal about?
      \item How important is a particular individual with respect to the achievement of the goal?
    \end{itemize}
  \end{block}
\end{frame}

\begin{frame}[fragile]{Introduction}
  \begin{block}{Goals of the Paper}
    \begin{itemize}
      \item Formal Definition of \textbf{BWVG}
      \item Investigate Computational Properties of \textbf{BWVG}
    \end{itemize}
  \end{block}
\end{frame}

\begin{frame}[fragile]{Agenda}
  \section{Preliminary Definitions}
  \tableofcontents[currentsection]
\end{frame}

\begin{frame}[fragile]{Preliminary Definitions}
  \begin{block}{Propositional Logic}
    \begin{itemize}
      \item Let $\Phi = \{p,q,\dots\}$ be a fixed non-empty vocabulary of Boolean variables    
      \item Let $\mathcal{L}$ denote the set of formulas of propositional logic over $\Phi$
      \item If "$\lor$" and "$\land$" are the only operators appearing in formula $\varphi$, se say that $\varphi$ is \textbf{monotone}
      \item If $\xi \subseteq \Phi$, we write $\xi \models \varphi$ mean that $\varphi$ is true satisfied by valuation $\xi$
    \end{itemize}
  \end{block}
\end{frame}

\begin{frame}[fragile]{Agenda}
  \section{Boolean Weighted Voting Games}
  \tableofcontents[currentsection]
\end{frame}

\begin{frame}[fragile]{Boolean Weighted Voting Games}
  \begin{block}{}
    asdfasfsd
  \end{block}
\end{frame}


\begin{frame}[fragile]{Agenda}
  \section{Shapley Value}
  \tableofcontents[currentsection]
\end{frame}

\begin{frame}[fragile]{Shapley Value}
  \begin{block}{}
    asdfasfsd
  \end{block}
\end{frame}

\begin{frame}[fragile]{Agenda}
  \section{The Core}
  \tableofcontents[currentsection]
\end{frame}

\begin{frame}[fragile]{The Core}
  \begin{block}{}
    asdfasfsd
  \end{block}
\end{frame}

\begin{frame}
  \begin{center}
    \textbf{\huge{Thank you!!}}
    \end{center}
\end{frame}

\end{document}