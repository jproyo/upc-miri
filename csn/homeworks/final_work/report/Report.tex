\documentclass[12pt, a4paper]{article}
\usepackage[utf8]{inputenc}
\usepackage{amsmath}
\usepackage{amsfonts}
\usepackage{amsthm}
\usepackage{array}
\usepackage{graphicx}
\usepackage{parskip}
\usepackage[pdfencoding=auto]{hyperref}
\usepackage{fancyhdr}
\usepackage{lastpage}
\usepackage{tikz}
\usepackage{float}
\usepackage{listings}
\usepackage{color}
\usepackage{caption}
\usepackage{authblk}
\usepackage{longtable}
\usepackage[acronym]{glossaries}
\usepackage[nottoc]{tocbibind}
\usepackage[cache=false]{minted}
\usemintedstyle{default}
\newminted{haskell}{frame=lines,framerule=2pt}
\newminted{R}{frame=lines,framerule=2pt}
\graphicspath{{./images/}}

\tikzstyle{bag} = [align=center]

\title{%
      Analysis of Software Design Principles \\
      under Complex Network Theory\\
}
\author{Juan Pablo Royo Sales \& Francesc Roy Campderrós}
\affil{Universitat Politècnica de Catalunya}
\date\today

\pagestyle{fancy}
\fancyhf{}
\fancyhead[C]{}
\fancyhead[R]{UPC MIRI}
\fancyhead[L]{CSN - Final Work}
\fancyfoot[L,C]{}
\fancyfoot[R]{Page \thepage{} of \pageref{LastPage}}
\setlength{\headheight}{15pt}
\renewcommand{\headrulewidth}{0.4pt}
\renewcommand{\footrulewidth}{0.4pt}

\newacronym{netdyn}{Network Dynamics}{Network Dynamics}
\newacronym{degdist}{Degree Distribution}{Degree Distribution}
\newacronym{growdeg}{Time Growth Degree}{Growth of Vertex Degree over time}
\newacronym{prefatt}{G Pref Attachment}{Growth + Preferential Attachment}
\newacronym{randatt}{G Random Attachment}{Growth + Random Attachment}
\newacronym{nprefatt}{NG Pref Attachment}{No Growth + Preferential Attachment}
\newacronym{baral}{Barabasi-Albert model}{Barabasi-Albert model}
\newacronym{indegree}{Incomming Degree}{Incomming Degree Language Set}
\newacronym{zeta}{Zeta}{Zeta Model}
\newacronym{aic}{Akaike}{Akaike Information Criterion}
\newacronym{dispp}{Displaced Poisson}{Displaced Poisson Model}
\newacronym{dispg}{Displaced Geometric}{Displaced Geometric Model}
\newacronym{hc}{High Cohesion}{High Cohesion}
\newacronym{lc}{Low Coupling}{Low Coupling}
\newacronym{cnt}{Complex Network Theory}{Complex Network Theory}
\newacronym{fp}{Functional Programming}{Functional Programming}
\newacronym{oop}{Object Oriented Programming}{Object Oriented Programming}
\newacronym{loc}{LoC}{Lines of Code}
\newacronym{cdg}{CDG}{Call Dependency Graph}
\newacronym{fu}{FU}{Functional Units}
\newacronym{anova}{ANOVA}{ANOVA Variance Analysis}

\newtheorem{hyp}{Hypothesis}

\begin{document}

\maketitle

\tableofcontents

\section{Introduction}
One of the most well known Software Design principles in \textbf{Software Engineering} is \acrfull{hc} and \acrfull{lc}, which is well described here~\cite{cohesion_coupling}.

As this two principles states a \textit{robust Software} should be design with \acrlong{lc} between their modules and \acrlong{hc} inside it. 

In other words, a Software that fullfil this characteristics should be very connected in their minimum \acrfull{fu} (Functions inside same file, Methods inside a class, etc), and with few connections between their coarse grained \acrshort{fu} (a.k.a Modules or Packages).

In this work, we are going to formulate some hypothesis which we believe it can been empirically proved. On the other hand we are going to show the relationship between these principles and how to measure them using \textbf{\acrfull{cnt}}.

Moreover, we are going to conduct an analysis on different kinds of software of different sizes and build under different language paradigms to see if the tool set that \acrshort{cnt} provides, are suitable for the general case. 

\section{Preliminaries}
In this section we are going to describe how and why the different Language Paradigms are selected and what is the criterion for selection of different Software solutions to be evaluated.

On the other hand as well, we are going to formulate some hypothesis that are going to guide our work to see if our assumptions can empirically been proved using \acrlong{cnt}.

\subsection{Context}
We have selected the most important $2$ main Language Paradigm to conduct the analysis: \acrfull{fp} and \acrfull{oop}. 

The reasons behind this decision are basically the following:

\begin{itemize}
    \item \textbf{95\%} of Software in the Industry are built with one of these 2 Paradigms according to the last results of this well-known survey~\cite{so_survey}.
    \item Due to the intrinsic nature of each of those Paradigms we have some hypothesis that we are going to describe later that can lead to different conclusion and Metrics
    \item If we can deduce some Software Design properties analyzing these 2 Paradigms we can generalize for the rest because they are quite different in nature and covers almost the whole Industry.
    \item We also believe that this Software Principles, \acrshort{hc} and \acrshort{lc}, should be applied indistinguishably no matter the chosen Paradigm. 
\end{itemize}

On the other hand the selection of the programs to be analyzed are the following:

\begin{itemize}
    \item Most of the software are Open Source or Free software that can be download publicly either from~\cite{github} or from~\cite{maven_repo}.
    \item Software that are marked as \textbf{PRIVATE} are Big Projects from Privates Companies that don't want to reveal the Sources Code and Names for Commercial reasons.
    \item In the case of \textbf{PRIVATE} \acrshort{fp} Solution, it belongs to a Company one of the authors of the current work is working right now.
    \item In the case of \textbf{PRIVATE} \acrshort{oop} Solution, it belongs to a Company one of the authors of the current work worked in the past.
    \item In both cases, taken anonymous data for conducting this analysis has been agreed with legal representatives of those Companies.
\end{itemize}

On the last hand we are going to use \textit{Haskell} Programs for analyzing \acrlong{fp} and \textit{Java} Programs for \acrlong{oop}. We believe that right now both are the most representative
ones in their Paradigm fields.

\subsection{Hypothesis}
In this work we are trying to prove the following \textbf{Hypothesis} that we consider can be proved using \acrfull{cnt}.

\begin{hyp}\label{hip:1}
Given any Software Program Solution, its Modularity Metric should be between the  $1$st Quartile and $3$rd Quartile, according to the average of the Network Metrics that we have been identified in this work,
to be considered as a well Software Designed Solution.
\end{hyp}

\begin{hyp}\label{hip:2}
Any \acrlong{oop} Program have a better modularity in terms of \acrlong{cnt} Metric rather than \acrlong{fp} Programs.
\end{hyp}

\begin{hyp}\label{hip:3}
The more \acrfull{loc} a Program have, the better Modularity it presents.
\end{hyp}
    
\begin{hyp}\label{hip:4}
If the Software follows the principle design of \acrshort{hc} and \acrshort{lc}, the \acrfull{degdist} of the Generated Graph should follow a power-law like.
\end{hyp}

\section{Results}
In this section first we are going to describe the \textbf{Experiments} conducted and after that we have obtained after running the different experiments; this is what we call \textbf{Metrics} subsection.

\subsection{Experiments}
In this section we are going to described how the experiment have been set up in order to prepare the graphs for taking the desired metrics, that could allow us to explain and verify the proposed hypothesis.

\subsubsection{Preliminaries}
In order to determine if a Software fullfil the 2 Software Design Principles that we want to analyze, \acrshort{hc} and \acrshort{lc}, 
we need to extract the \acrfull{cdg} from the programs that we want to analyze.

In \acrshort{cdg}, \textbf{nodes} are \acrlong{fu}: \textit{Functions} in the case of \acrshort{fp} and \textit{Methods} in the case of \acrshort{oop}.
An \textbf{edge} is when from inside a \acrshort{fu} another \acrshort{fu} is called or invoked.

Therefore, we need to build this \acrlong{cdg} in order to establish how the different Modules are interconnected in order to measure \acrshort{lc} and \acrshort{hc}.

\subsubsection{Building Graphs from Programs}
In order to achieve the desired \acrlong{cdg} we are going to use some specific tooling for each Paradigm.
The following has been used on each case:

\begin{itemize}
    \item \textbf{\acrlong{fp}}: \textit{function-call-graph}~\cite{fp_callgraph} is a Program that given a \textit{Haskell} Source Code it outputs a \textit{DOT} file with the \acrlong{cdg}
    \item \textbf{\acrlong{oop}}: \textit{java-callgraph}~\cite{java_callgraph} is a Program that given a \textit{Java} Compiled Jar it outputs a in \textit{stdout} the \acrlong{cdg}. In order to use only \textit{DOT} files we have converted this output into \textit{DOT} using a script that is under \mintinline{bash}{code/script_java.sh}
\end{itemize}

All the resulting graphs are under their respective folders as you can see here~\ref{apx:sec:org:1}.

\subsubsection{R Scripts}
We have only one \textit{R Script} where we concentrate all the logic to obtain metrics and plots. 

Basically, the most important function is the following one:

\begin{listing}[H]
    \inputminted[firstline=18, lastline=46, breaklines, highlightlines={19-23}]{R}{../code/Solution.R}
    \caption{Extracted from source code code/Script.R}
    \label{src:script:1}
\end{listing}
  
As it can be appreciated on highlighted lines, we are building the graph from the \textit{DOT} files, 
and the rest of the \mintinline{R}{run_metrics} function is the well-known metrics that we have explored before.   

On the other hand we've also built an \textit{R Script} that can be found in \mintinline{shell}{code/modularity_vs_linesCode.R} 
that plots the Modularity vs. \acrlong{loc}, which is going to be useful for the analysis of hypothesis~\ref{hip:3}.

Lately we have also built \mintinline{shell}{code/degree_distribution_fit.R} that is going to be useful for analyzing hypothesis~\ref{hip:4}. This last script is based on the Analysis of Degree Distribution that
we have conducted on Lab $2$.

\subsubsection{Testing Programs}
We have selected randomly $2$ programs $1$ \acrfull{fp} program and other \acrfull{oop} in order to test against
the metrics obtained and check if certain \textbf{hypothesis} holds or not.

\subsection{Metrics}
Regarding the metrics we are gathering the following ones:

\begin{itemize}
    \item \textbf{N}: Number of nodes. In our particular set up is the number of Functions or Methods and give us an idea of the big of the program.
    This can be also check by~\ref{apx:sec:loc} where we have how many \acrlong{loc} each program has.
    \item \textbf{E}: Number of vertices or the amount of calls that are taking place in the program
    \item \textbf{K}: Mean Degree Distribution. This will allow us to see how is the connectivity of the Graphs
    \item \textbf{MGD}: Average Path Length. 
    \item \textbf{Diameter}
    \item \textbf{Modularity}: This is a quite important metric. Although there is a work~\cite{paper_cohesion} that is proposing a better metric for modularity
     to measure \acrlong{hc} in a program, we believe that already explored Modularity metric is enough and it covers \acrlong{lc} as well.
    \item \textbf{Communities}: We believe that the number of communities should be similar of the number of modules, although it is out of the scope of this work exploring that
    and we think it could be an interesting topic to discuss in a future work.
    \item \textbf{Mean Clustering Coefficient (Mean CC Coef.)} and \textbf{Transitivity (Mean LC Coef.)}: We also believe that this 2 metrics are useful for measuring \acrlong{hc} and \acrlong{lc}
    alongside with \textbf{Modularity} because it is going to help us to understand if there is some concentration of calls in one particular \acrlong{fu}.
\end{itemize}

\subsubsection{FP Programs}
\begin{longtable}[H]{l r r r r r r}
    \caption{FP Programs Metrics 1}\label{table:fp_metrics_1}\\
        Program &  N & E & K & Delta & MGD & Diameter\\
        \hline            
        \endhead
        aeson     &  373 &  1167  & 6.26 & 0.0168 & 3.2450 &  9 \\      
        amazonka  &  739 &  2366  & 6.40 & 0.0087 & 3.3202 &  8 \\      
        async     &   60 &   120  & 4.00 & 0.0678 & 2.2458 &  6 \\      
        attoparsec&   61 &   180  & 5.90 & 0.0984 & 2.3978 &  5 \\      
        beam      &  852 &  2215  & 5.20 & 0.0061 & 4.4898 & 11 \\      
        cabal     & 2294 &  9115  & 7.95 & 0.0035 & 3.6017 & 10 \\      
        co-log    &   97 &   159  & 3.28 & 0.0341 & 3.8937 &  9 \\      
        conduit   &  457 &   875  & 3.83 & 0.0084 & 3.4084 &  9 \\      
        containers&   61 &   125  & 4.10 & 0.0683 & 3.3486 &  8 \\      
        criterion &   71 &   143  & 4.03 & 0.0575 & 3.3851 &  6 \\      
        cryptol   & 1803 &  6540  & 7.25 & 0.0040 & 3.4149 &  9 \\      
        cryptonite&  292 &   652  & 4.47 & 0.0153 & 3.9468 &  9 \\      
        dhall     &  707 &  2100  & 5.94 & 0.0084 & 3.6749 & 10 \\      
        free      &  148 &   328  & 4.43 & 0.0302 & 5.0257 & 11 \\      
        haskoin   &  569 &  1252  & 4.40 & 0.0077 & 5.0333 & 12 \\      
        hedgehog  &  567 &  1383  & 4.88 & 0.0086 & 4.0136 & 12 \\      
        helm      &   66 &    97  & 2.94 & 0.0452 & 3.4014 &  6 \\      
        hlint     &  266 &   624  & 4.69 & 0.0177 & 3.1839 & 10 \\      
        lens      & 1118 &  3908  & 6.99 & 0.0063 & 2.7743 &  7 \\      
        liquid    & 2568 &  7742  & 6.03 & 0.0023 & 3.3709 & 12 \\      
        megaparsec&  107 &   191  & 3.57 & 0.0337 & 4.4812 &  9 \\      
        mios      &  169 &   397  & 4.70 & 0.0280 & 3.8088 & 10 \\      
        optparse  &  174 &   464  & 5.33 & 0.0308 & 3.0912 &  7 \\      
        pandoc    & 3640 & 15951  & 8.76 & 0.0024 & 3.3057 &  9 \\      
        pipes     &  100 &   250  & 5.00 & 0.0505 & 2.2760 &  3 \\      
        postgresql&  501 &  1198  & 4.78 & 0.0096 & 3.8283 &  8 \\      
        protolude &  106 &   193  & 3.64 & 0.0347 & 4.0385 &  9 \\      
        QuickCheck&  264 &   699  & 5.30 & 0.0201 & 3.1726 &  8 \\      
        reflex    &  222 &   425  & 3.83 & 0.0173 & 3.4672 & 12 \\      
        relude    &  209 &   267  & 2.56 & 0.0123 & 6.0423 & 13 \\      
        servant   &  237 &   445  & 3.76 & 0.0159 & 4.5035 & 11 \\      
        snap      &  220 &   481  & 4.37 & 0.0200 & 3.8888 & 10 \\      
        stm       &   70 &   171  & 4.89 & 0.0708 & 2.4576 &  5 \\      
        summoner  &  194 &   458  & 4.72 & 0.0245 & 4.1549 & 10 \\      
        text      &  105 &   170  & 3.24 & 0.0311 & 2.1471 &  4 \\      
        vector    &  399 &  2953 1& 4.80 & 0.0372 & 2.2036 &  6 \\      
        yesod     &  367 &   748  & 4.08 & 0.0111 & 3.9486 & 12 \\     
        PRIVATE   & 1088 &   2239 &  4.12&  0.0038&  4.7322&  13 
\end{longtable}

\begin{longtable}[H]{l r r r r}
    \caption{FP Programs Metrics 2}\label{table:fp_metrics_2}\\
        Program & Modularity & Communities & Mean CC Coef & Mean LC Coef \\
        \hline            
        \endhead
        aeson      &  0.4826  &  14  &  0.3521 &  0.3367 \\      
        amazonka   &  0.4875  &  36  &  0.3273 &  0.2297 \\      
        async      &  0.3403  &   9  &  0.4868 &  0.4141 \\      
        attoparsec &  0.2458  &  16  &  0.4757 &  0.2906 \\      
        beam       &  0.6363  &  37  &  0.2488 &  0.2455 \\      
        cabal      &  0.4939  &  96  &  0.2981 &  0.1915 \\      
        co-log     &  0.6203  &  11  &  0.3144 &  0.4594 \\      
        conduit    &  0.4621  &  35  &  0.3211 &  0.3221 \\      
        containers &  0.4138  &   5  &  0.3692 &  0.3520 \\      
        criterion  &  0.6054  &   6  &  0.3516 &  0.3147 \\      
        cryptol    &  0.5320  &  79  &  0.3169 &  0.2060 \\      
        cryptonite &  0.5825  &  25  &  0.2886 &  0.2790 \\      
        dhall      &  0.5832  &  29  &  0.2975 &  0.2485 \\      
        free       &  0.5749  &  15  &  0.2570 &  0.4042 \\      
        haskoin    &  0.7069  &  24  &  0.2270 &  0.2604 \\      
        hedgehog   &  0.5792  &  44  &  0.2500 &  0.2757 \\      
        helm       &  0.5655  &   4  &  0.3585 &  0.4360 \\      
        hlint      &  0.4633  &  40  &  0.3589 &  0.3124 \\      
        lens       &  0.4207  &  24  &  0.3884 &  0.4447 \\      
        liquid     &  0.3388  &  19  &  0.3229 &  0.2057 \\      
        megaparsec &  0.6245  &   6  &  0.2847 &  0.2359 \\      
        mios       &  0.5310  &  16  &  0.3081 &  0.2211 \\      
        optparse   &  0.4466  &   8  &  0.3665 &  0.2248 \\      
        pandoc     &  0.4614  &  22  &  0.3248 &  0.2379 \\      
        pipes      &  0.3251  &   6  &  0.4704 &  0.6419 \\      
        postgresql &  0.6063  &  16  &  0.3060 &  0.5070 \\      
        protolude  &  0.5651  &   8  &  0.3174 &  0.3897 \\      
        QuickCheck &  0.4640  &  19  &  0.3605 &  0.3677 \\      
        reflex     &  0.5238  &  22  &  0.3469 &  0.4267 \\      
        relude     &  0.7759  &  14  &  0.2051 &  0.3334 \\      
        servant    &  0.6871  &  20  &  0.2463 &  0.3390 \\      
        snap       &  0.5939  &  19  &  0.2894 &  0.2925 \\      
        stm        &  0.3033  &   9  &  0.4538 &  0.3540 \\      
        summoner   &  0.5757  &  17  &  0.2866 &  0.1518 \\      
        text       &  0.3191  &  17  &  0.4863 &  0.5178 \\      
        vector     &  0.1458  &  17  &  0.4825 &  0.4041 \\      
        yesod      &  0.5858  &  13  &  0.2932 &  0.2731 \\    
        PRIVATE    &   0.6461 &   63 &   0.2323 &  0.1480
\end{longtable}

\subsubsection{FP Programs - Summary}
\begin{longtable}[H]{l r r r r r r}
    \caption{FP Programs - Summary Metrics 1}\label{table:fp_sum_metrics_1}\\
        Type & N & E & K & Delta & MGD & Diameter\\
        \hline            
        \endhead
        Min.    &   18.0 &    18 &   2.000 &  0.00230 & 2.147 &   3.000  \\
        1st Qu. &  105.5 &   192 &   3.915 &  0.00850 & 3.214 &   7.500  \\
        Median  &  237.0 &   481 &   4.690 &  0.01770 & 3.419 &   9.000  \\
        Mean    &  547.7 &  1764 &   5.036 &  0.02786 & 3.593 &   8.872  \\
        3rd Qu. &  568.0 &  1742 &   5.615 &  0.03440 & 3.981 &  10.500  \\
        Max.    & 3640.0 & 15951 &  14.800 &  0.11760 & 6.042 &  13.000 
\end{longtable}

\begin{longtable}[H]{l r r r r}
    \caption{FP Programs - Summary Metrics 2}\label{table:fp_sum_metrics_2}\\
        Type & Modularity & Communities & Mean CC Coef & Mean LC Coef \\
        \hline            
        \endhead
        Min.    & 0.1458 &   4.00 & 0.2051 & 0.1480  \\
        1st Qu. & 0.4540 &  10.00 & 0.2876 & 0.2369  \\
        Median  & 0.5320 &  17.00 & 0.3174 & 0.3124  \\
        Mean    & 0.5097 &  38.08 & 0.3315 & 0.3201  \\
        3rd Qu. & 0.5898 &  32.00 & 0.3597 & 0.3969  \\
        Max.    & 0.7759 & 419.00 & 0.4868 & 0.6419
\end{longtable}

\subsubsection{OOP Programs}
\begin{longtable}[H]{l r r r r r r}
    \caption{OOP Programs Metrics 1}\label{table:oop_metrics_1}\\
        Program &  N & E & K & Delta & MGD & Diameter\\
        \hline            
        \endhead
        akka           &   291 &   290 & 1.99 & 0.0069 & 5.0948 & 12 \\ 
        commons-cli-1. &   133 &   178 & 2.68 & 0.0203 & 4.9707 & 13 \\ 
        commons-codec- &   443 &   599 & 2.70 & 0.0061 & 6.0180 & 16 \\ 
        commons-csv-1. &   168 &   177 & 2.11 & 0.0126 & 2.8243 &  8 \\ 
        commons-email- &   182 &   202 & 2.22 & 0.0123 & 4.7634 & 14 \\ 
        disruptor-3.4. &   457 &   632 & 2.77 & 0.0061 & 6.4273 & 16 \\ 
        ftpserver-core &  1280 &  2319 & 3.62 & 0.0028 & 5.9379 & 17 \\ 
        grpc-core-1.34 &  3717 &  6544 & 3.52 & 0.0009 & 7.3362 & 24 \\ 
        guava          & 11786 & 18082 & 3.07 & 0.0003 & 7.4965 & 28 \\ 
        hbase-client-2 &  1724 &  1661 & 1.93 & 0.0011 & 8.4558 & 20 \\ 
        hsqldb-2.4.1   & 10394 & 22726 & 4.37 & 0.0004 & 6.6143 & 29 \\ 
        jackson-databi &  1637 &  2718 & 3.32 & 0.0020 & 6.9562 & 25 \\ 
        javax.servlet- &   440 &   387 & 1.76 & 0.0040 & 7.8849 & 19 \\ 
        jedis-3.4.1    &  4413 &  6625 & 3.00 & 0.0007 & 7.1696 & 25 \\ 
        jersey-core-1. &  1409 &  2046 & 2.90 & 0.0021 & 6.4969 & 19 \\ 
        jetty-7.0.0.pr &  1881 &  2947 & 3.13 & 0.0017 & 6.7668 & 24 \\ 
        joda-time-2.10 &  3901 &  8201 & 4.20 & 0.0011 & 6.3395 & 22 \\ 
        jsch-0.1.54    &  1314 &  2296 & 3.49 & 0.0027 & 5.5298 & 16 \\ 
        jsoup-1.13.1   &  1623 &  3558 & 4.38 & 0.0027 & 6.2448 & 17 \\ 
        junit-4.13.1   &  1602 &  2186 & 2.73 & 0.0017 & 7.0309 & 20 \\ 
        mail-1.4.7     &  1068 &  1474 & 2.76 & 0.0026 & 9.4969 & 29 \\ 
        mariadb-java-c &  2319 &  3678 & 3.17 & 0.0014 & 8.3089 & 24 \\ 
        mongo-java-dri &  9147 & 16481 & 3.60 & 0.0004 & 6.8356 & 27 \\ 
        mx4j-3.0.2     &   887 &  1192 & 2.69 & 0.0030 & 6.4340 & 18 \\ 
        org.eclipse.jg & 12059 & 23175 & 3.84 & 0.0003 & 6.3935 & 22 \\ 
        pdfbox-2.0.22  &  8372 & 16353 & 3.91 & 0.0005 & 6.0460 & 28 \\ 
        poi-4.1.2      & 15971 & 26910 & 3.37 & 0.0002 & 7.5505 & 28 \\ 
        postgresql-42. &  3164 &  4740 & 3.00 & 0.0009 & 6.8974 & 20 \\ 
        resteasy-jaxrs &  4289 &  6818 & 3.18 & 0.0007 & 6.2748 & 18 \\ 
        runtime-3.10.0 &   441 &   547 & 2.48 & 0.0056 & 5.2854 & 13 \\ 
        slf4j-api-1.7. &   180 &   182 & 2.02 & 0.0113 & 4.8468 & 11 \\ 
        spring-securit &  1501 &  2314 & 3.08 & 0.0021 & 6.5614 & 22 \\ 
        spring-web-5.3 &  7219 & 11261 & 3.12 & 0.0004 & 7.5364 & 34 \\ 
        tomcat-embed-c &  5386 &  8158 & 3.03 & 0.0006 & 7.3081 & 26 \\ 
        zookeeper-3.6. &  5284 &  9651 & 3.65 & 0.0007 & 6.3153 & 20 \\ 
        PRIVATE        &   339 &   529 & 3.12 & 0.0092 & 5.4404 & 13
\end{longtable}

\begin{longtable}[H]{l r r r r}
    \caption{OOP Programs Metrics 2}\label{table:oop_metrics_2}\\
        Program & Modularity & Communities & Mean CC Coef & Mean LC Coef \\
        \hline            
        \endhead
        akka         &  0.8347 &    39 &  0.0748 & 0 \\  
        commons-cli- &  0.6542 &    20 &  0.1658 & 0 \\  
        commons-code &  0.7835 &    67 &  0.0805 & 0 \\  
        commons-csv- &  0.8219 &    29 &  0.0425 & 0 \\  
        commons-emai &  0.7564 &    30 &  0.0893 & 0 \\  
        disruptor-3. &  0.6508 &    99 &  0.1152 & 0 \\  
        ftpserver-co &  0.6670 &   213 &  0.1232 & 0 \\  
        grpc-core-1. &  0.6385 &   779 &  0.1039 & 0 \\  
        guava        &  0.7583 &  2025 &  0.0754 & 0 \\  
        hbase-client &  0.9013 &   419 &  0.0264 & 0 \\  
        hsqldb-2.4.1 &  0.6779 &  1254 &  0.1260 & 0 \\  
        jackson-data &  0.7138 &   228 &  0.1261 & 0 \\  
        javax.servle &  0.8836 &   111 &  0.0348 & 0 \\  
        jedis-3.4.1  &  0.7270 &   586 &  0.1225 & 0 \\  
        jersey-core- &  0.8157 &   191 &  0.0776 & 0 \\  
        jetty-7.0.0. &  0.7034 &   286 &  0.1097 & 0 \\  
        joda-time-2. &  0.7110 &   461 &  0.1260 & 0 \\  
        jsch-0.1.54  &  0.6709 &   203 &  0.1223 & 0 \\  
        jsoup-1.13.1 &  0.7439 &   204 &  0.1290 & 0 \\  
        junit-4.13.1 &  0.7672 &   254 &  0.0888 & 0 \\  
        mail-1.4.7   &  0.7154 &   195 &  0.0872 & 0 \\  
        mariadb-java &  0.7519 &   244 &  0.0965 & 0 \\  
        mongo-java-d &  0.6193 &  1731 &  0.1066 & 0 \\  
        mx4j-3.0.2   &  0.7091 &   106 &  0.1061 & 0 \\  
        org.eclipse. &  0.6403 &  1404 &  0.1293 & 0 \\  
        pdfbox-2.0.2 &  0.6890 &   863 &  0.1311 & 0 \\  
        poi-4.1.2    &  0.7582 &  2181 &  0.0958 & 0 \\  
        postgresql-4 &  0.7695 &   506 &  0.0790 & 0 \\  
        resteasy-jax &  0.6456 &   888 &  0.0898 & 0 \\  
        runtime-3.10 &  0.7279 &    62 &  0.1229 & 0 \\  
        slf4j-api-1. &  0.8531 &    32 &  0.0670 & 0 \\  
        spring-secur &  0.6879 &   216 &  0.1068 & 0 \\  
        spring-web-5 &  0.6956 &  1315 &  0.0871 & 0 \\  
        tomcat-embed &  0.7786 &   684 &  0.0907 & 0 \\  
        zookeeper-3. &  0.6624 &   962 &  0.1052 & 0 \\  
        PRIVATE      &  0.6676 &    40 &  0.1739 & 0  
\end{longtable}

\subsubsection{OOP Programs - Summary}
\begin{longtable}[H]{l r r r r r r}
    \caption{OOP Programs - Summary Metrics 1}\label{table:oop_sum_metrics_1}\\
        Type & N & E & K & Delta & MGD & Diameter\\
        \hline            
        \endhead
        Min.    &   133 &   177 & 1.760 & 0.000200 & 2.824 &  8.00\\  
        1st Qu. &   672 &   912 & 2.715 & 0.000700 & 6.032 & 16.50\\  
        Median  &  1637 &  2718 & 3.120 & 0.001700 & 6.434 & 20.00\\  
        Mean    &  3456 &  5984 & 3.094 & 0.003403 & 6.489 & 20.46\\  
        3rd Qu. &  4351 &  8149 & 3.505 & 0.003500 & 7.100 & 24.50\\  
        Max.    & 15971 & 26910 & 4.380 & 0.020300 & 9.497 & 34.00  
\end{longtable}

\begin{longtable}[H]{l r r r r}
    \caption{OOP Programs - Summary Metrics 2}\label{table:oop_sum_metrics_2}\\
        Type & Modularity & Communities & Mean CC Coef & Mean LC Coef \\
        \hline            
        \endhead
        Min.    & 0.6193 &   20.0 & 0.02640 & 0 \\    
        1st Qu. & 0.6693 &  108.5 & 0.08715 & 0 \\    
        Median  & 0.7110 &  244.0 & 0.10610 & 0 \\    
        Mean    & 0.7256 &  512.0 & 0.10227 & 0 \\    
        3rd Qu. & 0.7628 &  731.5 & 0.12305 & 0 \\    
        Max.    & 0.9013 & 2181.0 & 0.17390 & 0 
\end{longtable}

\subsubsection{Testing Programs}
\begin{longtable}[H]{l r r r r r r}
    \caption{Testing Programs Metrics 1}\label{table:test_metrics_1}\\
        Program &  N & E & K & Delta & MGD & Diameter\\
        \hline            
        \endhead
        gogol         & 339 &  695 & 4.10 & 0.0121 & 4.9149 & 10\\    
        google-client & 942 & 1316 & 2.79 & 0.0030 & 5.5469 & 14\\ 
\end{longtable}

\begin{longtable}[H]{l r r r r}
    \caption{Testing Programs Metrics 2}\label{table:test_metrics_2}\\
        Program & Modularity & Communities & Mean CC Coef & Mean LC Coef \\
        \hline            
        \endhead
        gogol         & 0.6553 &  24 & 0.2359 & 0.1892 \\   
        google-client & 0.7413 & 178 & 0.0880 & 0.0000 \\
\end{longtable}

\subsubsection{Modularity}

\begin{minipage}[t]{\linewidth}
    \includegraphics[width=\textwidth]{modularity_paradigm.png}
    \captionsetup{type=figure}
    \captionof{figure}{Modularity Metric Comparing both Paradigms}
    \label{fig:modularity_paradigm}
  \end{minipage}


\subsubsection{Modularity VS logarithm of lines of code}


\begin{minipage}[t]{\linewidth}
    \includegraphics[width=\textwidth]{images/fp_lines_code_vs_modularity.jpeg}
    \captionsetup{type=figure}
    \captionof{figure}{Modularity VS log(Lines of Code) in FP}
    \label{fig:fp_log_lines_mod}
  \end{minipage}
  
  \begin{minipage}[t]{\linewidth}
    \includegraphics[width=\textwidth]{images/oop_lines_code_vs_modularity.jpeg}
    \captionsetup{type=figure}
    \captionof{figure}{Modularity VS log(Lines of Code) in OOP}
    \label{fig:oop_log_lines_mod}
  \end{minipage}
  
\begin{minipage}[t]{\linewidth}
    \includegraphics[width=\textwidth]{images/lines_code_vs_modularity.jpeg}
    \captionsetup{type=figure}
    \captionof{figure}{Modularity VS log(Lines of Code) in both paradigms together}
    \label{fig:log_lines_mod}
  \end{minipage}

\subsubsection{Degree Distribution - FP}

\begin{longtable}[H]{l r r r r r}
    \caption{FP Programs - Degree Distribution AIC Values}\label{table:fp_deg_dist_aic}\\
        Program & Disp Poisson & Disp Geom & Zeta 2 & Zeta & Zeta Truncated \\
        \hline            
        \endhead  
        aeson      &  1863.67550 &   0.00000 &   452.7808249 &  265.700378 &  265.584673 \\      
        amazonka   &  3616.77605 &   0.00000 &   837.9114064 &  471.041171 &  468.946567 \\      
        async      &   117.32660 &   0.00000 &    57.9771735 &   45.393422 &   47.431194 \\      
        attoparsec &   201.87476 &   0.00000 &    62.3974727 &   37.698004 &   39.563825 \\      
        beam       &  2991.16891 &   0.00000 &   643.2831784 &  378.697487 &  378.551372 \\      
        cabal      & 15395.53601 &   0.00000 &  3433.9869964 & 1812.260423 & 1790.841833 \\      
        co-log     &   151.95216 &   0.00000 &    36.8119509 &   31.385101 &   33.434477 \\      
        conduit    &  2569.95419 & 100.73831 &    22.7817231 &    0.000000 &    1.900681 \\      
        containers &    99.07037 &   0.00000 &    50.5854332 &   38.684562 &   40.727983 \\      
        criterion  &    21.21021 &   0.00000 &    83.0163027 &   64.800536 &   66.762371 \\      
        cryptol    & 11695.39923 &   0.00000 &  2346.8928801 & 1255.279725 & 1243.297796 \\      
        cryptonite &   629.52088 &   0.00000 &   221.8455212 &  151.598973 &  153.112739 \\      
        dhall      &  2852.00284 &   0.00000 &   799.2920305 &  476.202745 &  474.806324 \\      
        free       &   179.18579 &   0.00000 &   157.6723429 &  114.489806 &  116.181877 \\      
        haskoin    &   859.82342 &   0.00000 &   580.1156711 &  416.604701 &  417.333653 \\      
        hedgehog   &  1369.78069 &   0.00000 &   414.3882331 &  276.144396 &  276.983555 \\      
        helm       &   130.05235 &   0.00000 &     8.2776028 &    8.595222 &   10.714165 \\      
        hlint      &   925.12005 &   0.00000 &   226.2607732 &  151.972058 &  153.412732 \\      
        lens       & 15008.87852 &   0.00000 &   725.6466415 &  246.678077 &  243.899534 \\      
        liquid     & 20225.32515 &   0.00000 &  2341.8676728 & 1273.728286 & 1265.482522 \\      
        megaparsec &   175.20452 &   0.00000 &    58.6959541 &   47.188012 &   49.192447 \\      
        mios       &   183.79848 &   0.00000 &   194.2119753 &  137.095338 &  138.626423 \\      
        optparse   &   487.33560 &   0.00000 &   210.9939556 &  138.873027 &  140.179431 \\      
        pandoc     & 38849.09113 &   0.00000 &  5122.5021209 & 2432.889804 & 2395.127239 \\      
        pipes      &   443.08691 &   0.00000 &    53.4193641 &   29.916636 &   31.814254 \\      
        postgresql &  1993.14352 &   0.00000 &   498.3831077 &  337.969862 &  338.637945 \\      
        PRIVATE    &  2183.28655 &   0.00000 &   865.8009952 &  629.735886 &  630.170281 \\      
        protolude  &   282.15330 &   0.00000 &    17.5198441 &   10.981453 &   13.017689 \\      
        QuickCheck &   895.58710 &   0.00000 &   320.8080028 &  211.359256 &  212.284991 \\      
        reflex     &   640.20474 &   0.00000 &   147.4878994 &  112.081878 &  113.896163 \\      
        relude     &   403.01353 &  20.92190 &     0.0000000 &    1.975481 &    4.008396 \\      
        servant    &   477.06345 &   0.00000 &   109.9038212 &   81.117733 &   82.985295 \\      
        snap       &   503.40964 &   0.00000 &   179.8183309 &  126.887624 &  128.531432 \\      
        stm        &   185.14582 &   0.00000 &    66.8080777 &   45.693600 &   47.618021 \\      
        summoner   &   169.18845 &   0.00000 &   245.7948243 &  174.595515 &  175.964707 \\      
        text       &   485.77068 &  16.19161 &     0.3142174 &    0.000000 &    2.066109 \\      
        vector     &  8101.09215 &   0.00000 &   870.8989597 &  359.858763 &  349.192728 \\      
        yesod      &   900.84725 &   0.00000 &   271.9407233 &  198.458638 &  199.9\\92093 
    \end{longtable}

\subsubsection{Degree Distribution - OOP}
\begin{longtable}[H]{l r r r r r}
    \caption{OOP Programs - Degree Distribution AIC Values}\label{table:oop_deg_dist_aic}\\
        Program & Disp Poisson & Disp Geom & Zeta 2 & Zeta & Zeta Truncated \\
        \hline            
        \endhead  
        akka         &   849.6612 &   145.357181             &  20.536579 &   0.0000000 &   2.0277241 \\
        commons-cli  &   182.6214 &     0.000000             &  17.833938 &  18.3940841 &  20.4455174 \\
        commons-cod  &   645.1480 &     0.000000             &  60.343310 &  57.0577817 &  59.0402410 \\
        commons-csv  &   228.1798 &     3.582231             &   0.000000 &   0.2724416 &   2.3199915 \\
        commons-ema  &   351.5918 &    34.202642             &   0.174698 &   0.0000000 &   2.0437682 \\
        disruptor-3  &   783.4840 &     0.000000             &  31.659939 &  28.4734438 &  30.4552226 \\
        ftpserver-c  &  5580.8989 &   454.037696             &  33.448005 &   0.0000000 &   1.8283031 \\
        grpc-core-1  & 10835.3838 &    91.507957             & 199.586897 &   0.0000000 &   1.0972223 \\
        guava        & 33415.1211 &   783.739138             & 225.652920 &   0.0000000 &   0.9292341 \\
        hbase-clien  &  3936.7613 &   567.812973             & 111.768218 &   0.0000000 &   2.0037783 \\
        hsqldb-2.4.  & 60827.4576 &  3821.349841             & 769.117288 &   1.2167069 &   0.0000000 \\
        jackson-dat  &  6424.7809 &   477.549079             &  25.591008 &   0.0000000 &   1.8459512 \\
        javax.servl  &   648.7272 &    41.188374             &  28.799780 &   0.0000000 &   2.0181284 \\
        jedis-3.4.1  & 24909.2180 &  1073.532091             &  25.987669 &   0.0000000 &   1.7579590 \\
        jersey-core  &  2361.8926 &     0.000000             & 171.982172 & 143.2274926 & 145.0695584 \\
        jetty-7.0.0  &  4951.7062 &   213.122136             &  34.849186 &   0.0000000 &   1.8033043 \\
        joda-time-2  & 14344.2295 &     0.000000             & 626.523775 & 150.6451182 & 150.2312534 \\
        jsch-0.1.54  &  5288.0113 &   463.490503             &  25.704087 &   0.0000000 &   1.8569702 \\
        jsoup-1.13.  &  7396.7715 &   234.743571             & 173.210436 &   0.0000000 &   1.1219573 \\
        junit-4.13.  &  2547.7632 &     0.000000             & 164.405825 & 148.4537670 & 150.3365146 \\
        mail-1.4.7   &  2207.8988 &    43.976252             &   4.568415 &   0.0000000 &   1.9457354 \\
        mariadb-jav  &  5300.1980 &     0.000000             &  76.180868 &   5.4932517 &   7.1466141 \\
        mongo-java-  & 35497.2118 &  1675.279358             & 388.717083 &   0.0000000 &   0.3647674 \\
        mx4j-3.0.2   &  1979.2480 &   106.195916             &   0.000000 &   0.5395893 &   2.5146249 \\
        org.eclipse  & 55229.8134 &  2812.484850             & 646.961320 &   0.5943077 &   0.0000000 \\
        pdfbox-2.0.  & 48218.2095 &  3269.391476             & 338.721141 &   0.0000000 &   0.5533084 \\
        poi-4.1.2    & 64266.0959 &  4206.922778             & 338.744236 &   0.0000000 &   0.5289241 \\
        postgresql-  &  8332.5738 &   308.784490             &  40.648056 &   0.0000000 &   1.7415784 \\
        PRIVATE      &   971.4943 &    81.811881             &   1.688681 &   0.0000000 &   1.9977647 \\
        resteasy-ja  & 14213.8959 &   827.432359             &  68.805525 &   0.0000000 &   1.6086073 \\
        runtime-3.1  &   988.5741 &    59.818409             &   0.000000 &   1.8914142 &   3.9009734 \\
        slf4j-api-1  &   164.9488 &     0.000000             &  35.481064 &  36.5668486 &  38.6104244 \\
        spring-secu  &  3276.5518 &     0.000000             &  51.629722 &  14.4958942 &  16.3027900 \\
        spring-web-  & 20990.6186 &  1020.625763             & 120.106592 &   0.0000000 &   1.3590405 \\
        tomcat-embe  & 18388.7792 &  1354.552719             &  35.217930 &   0.0000000 &   1.6975627 \\
        zookeeper-3  & 24309.6424 &  1862.468736             & 153.995372 &   0.0000000 &   1.2761295 \\
    \end{longtable}

\begin{minipage}[t]{\linewidth}
    \includegraphics[width=\textwidth]{power_law_fp.png}
    \captionsetup{type=figure}
    \captionof{figure}{Example of FP Degree Distribution}
    \label{fig:power_law_fp}
  \end{minipage}

  \begin{minipage}[t]{\linewidth}
    \includegraphics[width=\textwidth]{power_law_oop.png}
    \captionsetup{type=figure}
    \captionof{figure}{Example of OOP Degree Distribution}
    \label{fig:power_law_oop}
  \end{minipage}


\section{Discussion and Analysis}

Here we are going to discuss the results obtained in our experiments.

\subsection{Hypothesis 1}

Regarding the first hypothesis~\ref{hip:1}, as we can see in~\ref{table:fp_sum_metrics_1} and~\ref{table:oop_sum_metrics_1}, and taking
the test programs~\ref{table:test_metrics_1} which the first one is \acrshort{fp} and the other is \acrshort{oop}, we can see that in the case
of \acrshort{fp} Modularity is slightly above the \textit{3rd Quartile} with $0.6553$, so we can consider that this Hypothesis hold for this program because
it is even better modularize. In the case of \acrshort{oop} also the modularity holds the hypothesis with $0.7413$.

We are aware that for having a more accurate test, we should have tested with a lot of programs, but due to the difficulty of generating the graphs and the time
involved it was not possible to achieve that goal which can be lead to future works.

\subsection{Hypothesis 2}

\acrlong{fp} average \textbf{modularity} is $0.5097$ whereas \acrlong{oop} average \textbf{modularity} is $0.7256$. This is an increase of $42\%$.
The lower value in \acrlong{oop} is $0.6193$ and the lower in \acrlong{fp} (maybe an outlier) is $0.1458$. 

We do an statiscal t-test to accept or reject our hypothesis. The null hypothesis is that both paradigms have almost equal average modularity. The alternative one is that one is bigger than the other:

\begin{minipage}[t]{\linewidth}
    \includegraphics[width=\textwidth]{images/ttest.png}
    \captionsetup{type=figure}
    \captionof{figure}{T test output (executed in R environment).}
    \label{fig:ttest}
  \end{minipage}

As we can see we accept our hypothesis as p-value is lower than 0.05.

\subsection{Hypothesis 3}
As we can see in this figure~\ref{fig:fp_log_lines_mod} and~\ref{fig:oop_log_lines_mod} we have to reject our Hypothesis as we can't see that there exist a clear relationship between this 2 variables. It is enough to check visually the images to make this statement.

But if we want to be more rigorous we will perform a statistical test to confirm it. Here the Null model is that this two variables are independent. So our alternative model that we propose is a linear model like this:

\begin{subequations}
    \begin{align}
        M = a + b \times LOC
    \end{align}
\end{subequations}

Where $M$ is \textbf{Modularity} and $LOC$ is \acrlong{loc}.

So we perform an \acrfull{anova} test between this $2$ models and the results are displayed in the following picture:

\begin{minipage}[t]{\linewidth}
    \includegraphics[width=\textwidth]{images/anova.png}
    \captionsetup{type=figure}
    \captionof{figure}{ANOVA test output (executed in R environment).}
    \label{fig:anova}
  \end{minipage}

As we can see we reject our hypothesis as p-value is greater than 0.05.

Our hypothesis was based on the intuition that when a program becomes bigger in terms of lines of code it has to be more modularized in order to be able to maintain it as developers.

\subsection{Hypothesis 4}

Regarding hypothesis~\ref{hip:4}, and after running the \acrlong{aic} analysis according to what have studied on Lab $2$, we can see that for \acrlong{fp} the hypothesis cannot be accepted,
because according to the results in~\ref{table:fp_deg_dist_aic}, \acrlong{dispg} is the most suitable distribution that fits the data.

On the other hand and regarding \acrlong{oop}, we have been able to prove using \acrlong{aic} technique, that almost all the programs fits into a \acrlong{zeta}, i.e. \textbf{power-law} like~\ref{table:oop_deg_dist_aic}.

Moreover, this is another metric that supports our assumption about the better modularization of \acrlong{oop} related that \acrlong{fp}.

\section{Conclusions}

We started this project inspired by~\cite{cohesion_coupling} where they used a refined modularity metric in order to asses the quality of Low Coupling and High cohesion.

We have focused our study not only in this mentioned aspect but also in:

\begin{itemize}
  \item Trying to see other typical properties that good-design-software has, as power-law-like degree distribution in its Call dependency graph.
  \item Capturing differences between different software paradigms in terms of modularity.
  \item Trying to find some relationship between the 2 metrics of a software program (lines of code and modularity).
\end{itemize}

In order to do this project, a dataset of programs was needed and the collection of it it was the most tedious aspect in the project. This collection has been made by hand in different online repositories mentioned in the References.

For instance in the case of Java programs, to generate the \acrlong{cdg}, a \textbf{compiled jar file} was needed and it is not common to find them online.

An alternative was downloading the source files and packaging them but this is not a scalable procedure if you want to have a considerable amount of observations in your dataset, because a lot of problems usually appear during compilation and packaging which should be solved meticulously. 

Possibly a good idea could be develop a tool that can generate the CDG from the source files.

This is the reason why other hypothesis that were proposed at the first moment as finding the relationship between modularity and number of code refactors (a.k.a. number of commits) or finding the relationship between modularity and number of contributors in a project were discarded (Github provides this 2 metrics for each project, i.e, number of commits and number of contributors).

\textbf{In conclusion,} we believe that \acrfull{cnt} is an accurate tool for \textbf{measuring Software Design Properties or Principles} such as \acrlong{hc} and \acrlong{lc}. 
Even more, we have empirically proven with this work that \acrlong{cnt} provide a tool set for establishing certain thresholds in order to conclude how good or bad a Program is designed and built. 

\bibliographystyle{alpha}
\bibliography{Report}

\appendix
\section{Organization}\label{apx:sec:org:1}

\begin{itemize}
    \item \textbf{code}: Under this folder you are going to find \textit{R Scripts} code for conducting this analysis.
    \item \textbf{fp\_graphs}: \textit{DOT} files that contains the Dependency Graph Representation of each \acrlong{fp} Program
    \item \textbf{oop\_graphs}: \textit{DOT} files that contains the Dependency Graph Representation of each \acrlong{oop} Program
    \item \textbf{report}: This report in Latex and PDF format.
\end{itemize}

\section{Programs Details - Line of Codes}\label{apx:sec:loc}
\subsection{FP Programs}
\begin{longtable}[H]{l r}
    \caption{FP Analyzed Programs}\label{table:ax_fp_prog_1}\\
        Program & LoC \\
        \hline            
        \endhead
        aeson & 6948 \\
        amazonka & 715531 \\
        async & 743 \\
        attoparsec & 4718 \\
        beam & 20151 \\
        cabal & 102525 \\
        co-log & 1436 \\
        conduit & 12963 \\
        containers & 19556 \\
        criterion & 2421 \\
        cryptol & 30740 \\
        cryptonite & 18763 \\
        dhall & 29058 \\
        free & 4472 \\
        fused-effects & 4145 \\
        ghcid & 1664 \\
        haskoin & 12066 \\
        hedgehog & 8277 \\
        helm & 2071 \\
        hlint & 6306 \\
        lens & 16691 \\
        liquid & 133740 \\
        megaparsec & 8144 \\
        mios & 6178 \\
        mtl & 932 \\
        optparse & 3220 \\
        pandoc & 69179 \\
        pipes & 1969 \\
        postgresql & 6596 \\
        protolude & 1901 \\
        quickcheck & 5077 \\
        reflex & 10062 \\
        relude & 2913 \\
        servant & 15725 \\
        snap & 5310 \\
        stm & 1550 \\
        summoner & 4025 \\
        text & 9783 \\
        vector & 12166 \\
        yesod & 19971 \\
        PRIVATE PROGRAM & 26975
\end{longtable}

\subsection{OOP Analytzed Programs}
\begin{longtable}[H]{l r}
    \caption{OOP Analyzed Programs}\label{table:ax_oop_prog_1}\\
        Program & LoC \\
        \hline            
        \endhead
        akka-actor\_2.10-2.3.9 & 35702 \\
        commons-cli-1.4 & 830 \\
        commons-codec-1.10 & 2231 \\
        commons-csv-1.8 & 607 \\
        commons-email-1.4 & 760 \\
        disruptor-3.4.2 & 1131 \\
        ftpserver-core-1.0.6 & 4052 \\
        grpc-core-1.34.1 & 9142 \\
        guava-28.1-jre & 37216 \\
        hbase-client-2.4.0 & 33209 \\
        hsqldb-2.4.1 & 30578 \\
        jackson-databind-2.12.0 & 22516 \\
        javax.servlet-api-4.0.1 & 901 \\
        jedis-3.4.1 & 12640 \\
        jersey-core-1.19.4 & 5656 \\
        jetty-7.0.0.pre5 & 6727 \\
        joda-time-2.10.6 & 8891 \\
        jsch-0.1.54 & 4833 \\
        jsoup-1.13.1 & 5955 \\
        junit-4.13.1 & 5803 \\
        mail-1.4.7 & 8817 \\
        mariadb-java-client-2.7.1 & 9229 \\
        mongo-java-driver-3.12.7 & 35676 \\
        mx4j-3.0.2 & 6902 \\
        org.eclipse.jgit & 42417 \\
        pdfbox-2.0.22 & 26413 \\
        poi-4.1.2 & 44691 \\
        postgresql-42.2.18 & 15199 \\
        resteasy-jaxrs-3.14.0.Final & 15267 \\
        runtime-3.10.0-v20140318-2214 & 921 \\
        slf4j-api-1.7.30 & 763 \\
        spring-security-core-5.4.2 & 4726 \\
        spring-web-5.3.2 & 21393 \\
        tomcat-embed-core-10.0.0 & 47185 \\
        zookeeper-3.6.2 & 19207 \\
        PRIVATE PROGRAM & 2143
\end{longtable}

\end{document}

