\documentclass[12pt, a4paper]{article}
\usepackage[utf8]{inputenc}
\usepackage{amsmath}
\usepackage{amsthm}
\usepackage{amssymb}
\usepackage{graphicx}
\usepackage{parskip}
\usepackage{hyperref}
\usepackage{fancyhdr}
\usepackage{lastpage}
\usepackage[vlined,ruled]{algorithm2e}
\usepackage[acronym]{glossaries}
\usepackage{caption}
\usepackage{titlesec}
  

\title{%
  Algorithmic Game Theory \\
  Final Exam
}
\author{%
  Juan Pablo Royo Sales\\
  \small{Universitat Politècnica de Catalunya}
}
\date\today

\pagestyle{fancy}
\fancyhf{}
\fancyhead[C]{}
\fancyhead[R]{Juan Pablo Royo Sales - UPC MIRI}
\fancyhead[L]{AGT - Final Exam}
\fancyfoot[L,C]{}
\fancyfoot[R]{Page \thepage{} of \pageref{LastPage}}
\setlength{\headheight}{15pt}
\renewcommand{\headrulewidth}{0.4pt}
\renewcommand{\footrulewidth}{0.4pt}

\renewcommand{\qedsymbol}{$\blacksquare$}
\newacronym{ncg}{NCG}{Network Congestion Game}
\newacronym{sncg}{SNCG}{Symmetric Network Congestion Game}
\newacronym{pne}{PNE}{Pure Nash Equilibrium}
\newacronym{ne}{NE}{Nash Equilibrium}
\newacronym{mincost}{MINCOST}{Min-Cost Flow Problem}

\begin{document}

\maketitle

\section{Exercise 1}
A \acrfull{ncg} is \textbf{symmetric} when $s_i = s$ and $t_i = t$, for $i \in N$ according to the definition that we have seen in class. Basically all the players, have the 
same source and destination endpoint $s$ and $t$ and all have the same path strategies.

The algorithm to find \acrfull{pne} in \acrshort{sncg} in \textit{poly-time} is doing a reduction to \acrfull{mincost}
in order to find the Optimum of $\varphi(s)$.

\begin{algorithm}[H]
  \SetKwInOut{Input}{Input}
  \SetKwInOut{Output}{Output}
  \Input{$G = (V, E, s, t)$, $d_e$ delay function}
  \Output{\acrshort{mincost} representation of the problem}
  \For{each $e \in E$}{
    \For{$1$ up to $d_e$}{
      $E \leftarrow E \cup \{e'_i\}$ such that $d_{e'_i} = 1$;\\
    }
    $E \leftarrow E \setminus \{e\}$;\\
  }
  return $N$;
  \caption{Compute \acrshort{mincost} reduction of \acrshort{sncg}}
\end{algorithm}

Basically the reduction is replacing each edge $e$ with the amount of the delay function for $e$ $d_e$ of new edges
but this time each of this edges with delay function equal to $1$.

The \acrshort{mincost} of the new network $N$ minimizes $\varphi(s)$.

Since the algorithm is giving also a Local Optimum, then $s$ is a \acrshort{pne}.

\section{Exercise 2}
As we have seen in class the \textbf{Social Cost} is the sum of all players 

\begin{equation}
  C(s) = \sum_{u \in V} c_u(s) = \alpha|E| + \sum_{u,v \in V} d_G(u,v)
\end{equation}

Every pair of vertices that is not \textit{connected} its distances is $\ge 2$, there is a lower bound for the 
social cost:

\begin{subequations}
  \begin{align}
    C(s) &\geq \alpha|E| + 2|E| + 2(n(n - 1) - 2|E|)\\
         &= 2n(n - 1) + (\alpha - 2)|E|\label{eq:1}
  \end{align}
\end{subequations}

\subsection{(a)}
In this case when $\alpha < 1$ the social Optimum is achieved when $|E|$ is maximum. So, every vertex is connected with each other. It is a complete graph.
Any \acrfull{ne} should be of diameter $1$ which implies that the complete graph is the only \acrshort{ne}.

\subsection{(b)}
In the case $1 \leq \alpha < 2$, the social Optimum is achieved by a complete graph, but not \acrshort{ne}. Any \acrshort{ne} is for diameter at most $2$.
So, the social cost is equal to~\ref{eq:1} and it is the worst cost when $|E|$ is minimum, which is $n-1$ for a connected graph. Therefore the worst \acrshort{ne} is the $S_n$ Star.

\subsection{(c)}
As we have seen in class when $\alpha > 2$ we need to minimize the number of edges $|E|$ in order to reduce the cost, but at the same time the graph should be connected.
Therefore only trees with diameter $2$ have optimal cost. One example that has \acrshort{ne}, with diameter $2$ and minimum number of edges is a \textbf{Cycle Graph} with 4 vertex.

\subsection{(d)}
Since when $\alpha > n^2$ no player has any incentive to buy an edge, there is going to be the minimum amount possible of edges at the same time with the graph connected, 
that by definition are \textbf{Spanning Trees} as we have seen in class. All are \acrshort{ne} because no one has incentive to change.
The $PoA$ is giving by:

\begin{subequations}
  \begin{align}
    PoA &= \frac{c(T_n)}{c(S_n)}\\
        &\leq \frac{\alpha(n-1)+(n-1)(n-1)}{\alpha(n-1)+1+2n(n-1)}\\ 
        & = O(1)
  \end{align}
\end{subequations}

\end{document}