\documentclass[12pt, a4paper]{article}
\usepackage[utf8]{inputenc}
\usepackage{amsmath}
\usepackage{amsthm}
\usepackage{graphicx}
\usepackage{parskip}
\usepackage{hyperref}
\usepackage{fancyhdr}
\usepackage{lastpage}
\usepackage[acronym]{glossaries}

\title{%
      Homework 1 \\
      Experimental Prove of $\theta(\log n)$ height on Random Binary Search Trees
}
\author{Juan Pablo Royo Sales}
\date\today

\pagestyle{fancy}
\fancyhf{}
\fancyhead[C]{}
\fancyhead[R]{Juan Pablo Royo Sales - UPC MIRI}
\fancyhead[L]{ADS - Homework 1}
\fancyfoot[L,C]{}
\fancyfoot[R]{Page \thepage{} of \pageref{LastPage}}
\setlength{\headheight}{15pt}
\renewcommand{\headrulewidth}{0.4pt}
\renewcommand{\footrulewidth}{0.4pt}

\makeglossaries

\newacronym{bst}{BST}{Binary Search Tree}
\newacronym{rbst}{RandomBST}{Randomly Built Binary Search Tree}
\newacronym{rtreap}{RTreap}{Randomized Treap}

\begin{document}

\maketitle

\section{Introduction}
In this Homework I have selected to explore and analyze \textbf{\acrfull{rbst}} in order to prove through experimentation that the expected height of \textit{\acrshort{rbst}} is $E[h] = \theta(\log n)$.

In order to do that we are going to first give a brief introduction to this type of Data Structure and the theoretical analysis of this feature.

\section{\acrlong{rbst}}
A \acrshort{rbst} is a \textbf{\acrfull{bst}} that is specifically built using random insertions, in order to achieve \textbf{balanced} \acrshort{bst} \textit{whp}.

Given an insertion in a \acrshort{rbst} of size $n - 1$ there exists the same probability for the key to be inserted in the same interval that the keys already present in the tree. Basically all $n{!}$ possible permutations of the keys are equally likely.

There exists a specific implementation, which is the one I used to conduct the analysis, which is \textbf{\acrfull{rtreap}}.

In the following sections I am going to describe this Data Structure that fulfill the specification of an \acrshort{rbst}.

\subsection{Formal Analysis}


\printglossary[type=\acronymtype]

\end{document}

