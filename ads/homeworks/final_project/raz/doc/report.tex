\documentclass[12pt, a4paper]{article}
\usepackage[utf8]{inputenc}
\usepackage{amsmath}
\usepackage{amsfonts}
\usepackage{amsthm}
\usepackage{graphicx}
\usepackage{parskip}
\usepackage{hyperref}
\usepackage{fancyhdr}
\usepackage{lastpage}
\usepackage{tikz}
\usepackage{float}
\usepackage{listings}
\usepackage{color}
\usepackage{caption}
\usepackage[acronym]{glossaries}
\usepackage[nottoc]{tocbibind}
\usepackage[cache=false]{minted}
\usemintedstyle{default}
\newminted{haskell}{frame=lines,framerule=2pt}
\graphicspath{{./images/}}

\tikzstyle{bag} = [align=center]

\title{%
      Final Homework\\
      Analysis of Random Access Zippers
}
\author{%
  Juan Pablo Royo Sales \\
  \small{Universitat Politècnica de Catalunya}
}
\date\today

\pagestyle{fancy}
\fancyhf{}
\fancyhead[C]{}
\fancyhead[R]{Juan Pablo Royo Sales - UPC MIRI}
\fancyhead[L]{ADS - Final Homework}
\fancyfoot[L,C]{}
\fancyfoot[R]{Page \thepage{} of \pageref{LastPage}}
\setlength{\headheight}{15pt}
\renewcommand{\headrulewidth}{0.4pt}
\renewcommand{\footrulewidth}{0.4pt}

\newacronym{haskell}{Haskell}{Haskell Programming Language}
\newacronym{kde}{KDE}{Kernel Density Estimate}
\newacronym{lrm}{LRM}{Linear Regression Model}
\newacronym{raz}{RAZ}{Random Access Zippers}
\newacronym{zip}{Zipper}{Zipper}
\newacronym{ftree}{F-Tree}{Finger Trees}
\newacronym{rrb-v}{RRB-Vector}{RRB-Vector}
\newacronym{bst}{BST}{Binary Search Trees}
\newacronym{fp}{FP}{Functional Programming}

\begin{document}

\maketitle

\section{Introduction}\label{sec:intro}
The propose of this work is to Analyzed and reproduced the work and experiments of \textbf{\acrfull{raz}} written for the first time here \cite{raz}.

\acrfull{raz} is a combination of a \textbf{\acrfull{zip}} Data Structure with a simple \textbf{\acrfull{bst}}, in fact with a probabilistic-balanced \acrshort{bst}. This powerful combination enables to have a \textit{Sequence} like structure with $O(1)$ edit time and $O(\log{n})$ amortized time for random access elements.

In the following sections I am going to describe:

\begin{itemize}
    \item Brief introduction of the Structure and its main features
    \item Implementation Details
    \item Experiments that have been reproduced from the paper plus other additional experiments
    \item Caveats and Remarks
    \item Conclusions
\end{itemize}

\section{Asset organization}
Please check the appendix~\ref{apx:org} to see how the different assets are organized, how to run the experiment, and so on.

\section{Analysis of RAZ's paper}
\subsection{Motivation}
The motivation of this data structure appears with the intention to have an \textbf{\textit{purely functional Sequence}} like data structure with better performance on edition and random access in worst case and amortized case, because typical sequence structures in \acrfull{fp} world, \textit{List}, has only efficient edit and access on the \textit{head} of the list, but requires \textbf{linear} time on worst case.

It is well known as it is stated on the paper that there were \textbf{2(two) other} data structures, \acrfull{ftree} and \acrfull{rrb-v} most recently, that overcome this issues in \acrshort{fp} sequence like. But the implementation of this 2 data structures are complex and not so extensible as \textit{List}. \acrshort{raz} main motivation is to give the same performance as \acrshort{ftree} and \acrshort{rrb-v} with the simplicity for extension as \textit{Lists}.

\section{Implementation}
%\acrshort{tst} implementation in \acrfull{haskell} is really short and easy to read as we can see here:
%
%
%\begin{listing}[H]
%  \inputminted[firstline=16, lastline=56, breaklines]{haskell}{../src/Data/Tree/TST.hs}
%  \caption{Extracted from source code src/Data/Tree/TST.hs}
%  \label{src:tst}
%\end{listing}
%
%
%\section{Experiments Details}
%
%\begin{itemize}
%  \item \textbf{Benchmark Running Time}
%  \item \textbf{Profile Space footprint}
%\end{itemize}
%
%The machine used to run the experiment is \textit{2,2 GHz 6-Core Intel Core i7} with \textit{32 Gb RAM}.
%
%
%\begin{itemize}
%  \item Running time of Searching $117943$ in \acrshort{tst} already loaded with $235886$ words
%  \item Running time of Searching $117943$ in \acrshort{map} already loaded with $235886$ words
%\end{itemize}


\section{Experiments}

\section{Caveats and Remarks}

\section{Conclusions}


\bibliographystyle{alpha}
\bibliography{report}

\printglossary[type=\acronymtype]

\appendix\label{apx:org}
\section{Haskell}
\subsection{Source Code}
In the source code there are 3 folders with code:

\begin{itemize}
  \item \textbf{app}: Which contains the program source code file. Here we can find:
  \item \textbf{src}: Contains the implementation source code of \acrshort{raz}.
    \begin{itemize}
      \item \textbf{\mintinline{haskell}{src/Experiments.hs}}: It contains the different experiment run.
      \item \textbf{\mintinline{haskell}{src/Data/Zipper/Random.hs}}: \acrshort{raz} Implementation.
    \end{itemize}
  \item \textbf{test}: Contains Property based testing that automatize the test of the implemented \acrshort{raz}
\end{itemize}

\subsection{Run the Code}
All the solution has been coded with \textbf{Stack} \cite{stack} version 2.1.3 or higher. It is a prerequisite to install \textit{stack} for running this code.

\subsubsection{Running Experiments}
In order to run the experiments just do the following in each case.

\begin{itemize}
  \item \textbf{Experiment 1} Building the structure from scratch with different sizes.

\begin{minted}{bash}
stack build
stack exec raz experiment-1
\end{minted}

\setlength{\rightskip}{0pt plus 1 fil}
This is going to left the results in CSV file format under \mintinline{bash}{output/result_exp1_UNIX_TIME_IN_MILLIS.csv}.


  \item \textbf{Experiment 2} Insert 1 million elements maintaining the same structure until reach 100 Millions elements.

\begin{minted}{bash}
stack build
stack exec raz experiment-2
\end{minted}

\setlength{\rightskip}{0pt plus 1 fil}
This is going to left the results in CSV file format under \mintinline{bash}{output/result_exp2_UNIX_TIME_IN_MILLIS.csv}.

  \item \textbf{Benchmark} Benchmark general execution time for each combinator \mintinline{haskell}{insert},  \mintinline{haskell}{alter}, \mintinline{haskell}{remove}, \mintinline{haskell}{move}, \mintinline{haskell}{focus}, \mintinline{haskell}{unfocus}.

\begin{minted}{bash}
stack build
stack exec raz bench
\end{minted}

\setlength{\rightskip}{0pt plus 1 fil}
This is going to left an HTML report with the benchmark results in \mintinline{bash}{output/report_benchmark_UNIX_TIME_IN_MILLIS.html}.


\end{itemize}

\section{Generated Outputs}\label{apx:reports}
All the Generated Outputs that has been used to build this document are under \mintinline{haskell}{output} folder.

\section{Report PDF Document}
This report document is under \mintinline{haskell}{doc} folder alongside images that are embedded in this report.

\end{document}

