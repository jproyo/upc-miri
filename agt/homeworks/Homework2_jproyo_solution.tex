\documentclass[12pt, a4paper]{article}
\usepackage[utf8]{inputenc}
\usepackage{amsmath}
\usepackage{amsthm}
\usepackage{amssymb}
\usepackage{graphicx}
\usepackage{parskip}
\usepackage{hyperref}
\usepackage{fancyhdr}
\usepackage{lastpage}
\usepackage[vlined,ruled]{algorithm2e}
\usepackage[acronym]{glossaries}
\usepackage{caption}
\usepackage{titlesec}
\usepackage{multirow}


\title{%
  Algorithmic Game Theory \\
  Homework 2 - Solutions
}
\author{%
  Juan Pablo Royo Sales\\
  \small{Universitat Politècnica de Catalunya}
}
\date\today

\pagestyle{fancy}
\fancyhf{}
\fancyhead[C]{}
\fancyhead[R]{Juan Pablo Royo Sales - UPC MIRI}
\fancyhead[L]{AGT - Homework 2}
\fancyfoot[L,C]{}
\fancyfoot[R]{Page \thepage{} of \pageref{LastPage}}
\setlength{\headheight}{15pt}
\renewcommand{\headrulewidth}{0.4pt}
\renewcommand{\footrulewidth}{0.4pt}

\renewcommand{\qedsymbol}{$\blacksquare$}

\begin{document}

\maketitle

\section{Problem 10}
\begin{center}
  \begin{tabular}{p{5pt}c|cc }
  \multicolumn{2}{}{} & \multicolumn{2}{c}{P2}\\
  & & A & B\\
  \multirow{2}{*}{\rotatebox[origin=c]{90}{P1}}
  & A & 6,6 & 2,7 \\
  & B & 7,2 & 0,0 \\
\end{tabular}
\end{center}

$PNE$ can be easily seen when $u_1(B) \land u_2(A)$ or $u_1(A) \land u_2(B)$ where $\{(7,2), (2,7)\}$.

In the case of \textit{mixed Nash Equilibrium} we need to divide the problem in two parts.

Lets fix the probability of Mixed Nash Equilibrium Game for $P2$ as $p$ if $P2$ chooses $A$ or $1-p$ if it chooses $B$. Then if $P1$ plays a mixed game we have that

\begin{subequations}
  \begin{align}
    u_1(A) & = u_1(B)\\
    6p + 2(1-p) &= 7p + 0(1-p)\\
    p &= \frac{2}{11}
  \end{align}
\end{subequations}

And therefore $1-p = \frac{9}{11}$.

Given the case of $P1$ chooses some $q$ probability for playing $A$ and $1-q$ for playing $B$, then we can analyze $P2$ playing a mixed game:


\begin{subequations}
  \begin{align}
    u_2(A) & = u_2(B)\\
    6p + 2(1-p) &= 7p + 0(1-p)\\
    p &= \frac{2}{11}
  \end{align}
\end{subequations}
And therefore $1-q = \frac{9}{11}$.

Therefore there is a \textit{mixed Nash Equilibrium} with probabilities $\{(\frac{2}{11},\frac{9}{11}),(\frac{2}{11},\frac{9}{11}) \}$

\section{Problem 12}
There is no $NE$ for the mixed game. Lets show why.
Given the following table:

\begin{center}
  \begin{tabular}{p{5pt}c|cc }
  \multicolumn{2}{}{} & \multicolumn{2}{c}{P2}\\
  & & A & B\\
  \multirow{3}{*}{\rotatebox[origin=c]{90}{P1}}
  & C & 1,1 & 4,2 \\
  & D & 3,3 & 1,1 \\
  & E & 2,2 & 2,3
\end{tabular}
\end{center}

We can set for example for $P2$ probabilities $p$ and $1-p$ for choosing $A$ or $B$ response.
If there was a $NE$ then $P1$ should have a distribution probability for choosing $C, D \text{or } E$ being indifferent one choose over the other as we shown in Problem 10. If that the case we should have the following equality.

\begin{subequations}
  \begin{align}
    u_1(C) &= u_1(D) &= u_1(E)\\
    p + 4(1-p) &= 3p + (1-p) &= 2p + 2(1-p)\\
    4 - 3p &= 2p + 1 &= 2\label{eq:8}
  \end{align}
\end{subequations}

But we can see that that is \textbf{false}. Therefore there is an incentive to choose one strategy over the other greater than 1 which indicates that in this game there is not a Pure Nash Equilibrium.

\section{Problem 13}
Lets analyze the utility after applying the distribution $(0.6,0.4), (0.2,0.4,0.4)$

\begin{center}
  \begin{tabular}{p{5pt}c|ccc }
  \multicolumn{2}{}{} & \multicolumn{3}{c}{P2}\\
  & & R & S & T\\
  \multirow{2}{*}{\rotatebox[origin=c]{90}{P1}}
  & A & 3.6,1.2 & 1.2,2.8 & 1.2,2.4 \\
  & B & 2.8,0.4 & 0.8,2.8 & 0.8,2.4 \\
\end{tabular}
\end{center}

As we can see there is a $NE$ when $P2$ chooses $S$ and $P1$ chooses $A$ since non of them has any incentive to change.

\end{document}

