\documentclass{beamer}
\usepackage[T1]{fontenc}
\usepackage{listings}
\usepackage{color}
\usepackage[linguistics]{forest}
\setbeamertemplate{frametitle}{%
  \usebeamerfont{frametitle}\insertframetitle%
  \vphantom{g}% To avoid fluctuations per frame
  % \hrule% Uncomment to see desired effect, without a full-width hrule
  \par\hspace*{-\dimexpr0.5\paperwidth-0.5\textwidth}\rule[0.5\baselineskip]{\paperwidth}{0.4pt}
  \par\vspace*{-\baselineskip}% <-- reduce vertical space after rule
}

\lstset{frame=tb,
  language=Haskell,
  aboveskip=3mm,
  belowskip=3mm,
  showstringspaces=false,
  columns=flexible,
  basicstyle={\small\ttfamily},
  numbers=left,
  numberstyle=\tiny\color{gray},
  keywordstyle=\color{blue},
  commentstyle=\color{dkgreen},
  stringstyle=\color{mauve},
  breaklines=true,
  breakatwhitespace=true,
  tabsize=2,
  stepnumber=1,
  escapechar=!
}

\definecolor{dkgreen}{rgb}{0,0.6,0}
\definecolor{gray}{rgb}{0.5,0.5,0.5}
\definecolor{mauve}{rgb}{0.58,0,0.82}

\usetheme{Boadilla}
\title{Selective Applicative Functors}
\author{Juan Pablo Royo Sales\\
  \ \\
  \tiny{Originally by:\\
  Andrew Mokhov, Georgy Lukyanov, Simon Marlow, Jeremie Dimino\\
  Presented in ICPF 2019}
}
\institute{UPC $\sim>$ FIB $\sim>$ TMIRI}
\date{December $12^{th}$, 2019}
\begin{document}

\begin{frame}
\titlepage
\end{frame}

\begin{frame}{Agenda}
  \tableofcontents
\end{frame}

\begin{frame}{Agenda}
  \section{Context - Monad and Applicative Functors}
  \tableofcontents[currentsection]
\end{frame}

\begin{frame}[fragile]{Context - Monad and Applicative Functors}
  \begin{block}{Applicative Functors}
    \textbf{\textit{Is a Functor to write effectful computations. It is weaker
        than Monad. Every Monad is an Applicative Functor}}.\\
  \end{block}

\begin{lstlisting}[language=Haskell]
  class Functor f => Applicative f where 
    (<*>) :: f (a -> b) -> f a -> f b
    
    pure :: a -> f a
  \end{lstlisting}

  \begin{block}{Hints}
  \begin{itemize}
    \item Non Dependent computations
    \item Commutative computations
    \end{itemize}
    \end{block}
 
\end{frame}

\begin{frame}[fragile]{Context - Monad and Applicative Functors}

  \begin{block}{Monad}
    \textbf{\textit{A Monad is a Monoid in the monoidal Category of Endofunctors}}.
  \end{block}

\begin{lstlisting}[language=Haskell]
  class Applicative f => Monad f where
    (>>=) :: f a -> (a -> f b) -> f b

    return :: a -> f a
  \end{lstlisting}

  \begin{block}{Hints}
  \begin{itemize}
    \item Dependent computations
    \item Non commutative
    \end{itemize}
  \end{block}
\end{frame}

\begin{frame}[fragile]{Context - Monad and Applicative Functors}

  \begin{block}{Monad}
    \begin{itemize}
      \item Effects are opaque
      \item Cannot do static analysis of effects
    \end{itemize}
  \end{block}

\begin{lstlisting}[language=Haskell]
    pingPongM :: IO ()
    pingPongM = getLine >>= \s -> if s == "ping" then putStrLn "pong" else return ()
\end{lstlisting}
\end{frame}

\begin{frame}[fragile]{Context - Monad and Applicative Functors}

  \begin{block}{Applicative Functors}
    \begin{itemize}
    \item Effects are no longer opaque
    \item Static analysis of effects can be done
    \item But we lose conditional effects
    \end{itemize}
  \end{block}

\begin{lstlisting}[language=Haskell]
    pingPongA :: IO ()
    pingPongA = fmap (\s -> id) getLine <*> putStrLn "pong"
\end{lstlisting}

\end{frame}

\begin{frame}[fragile]{Context - Monad and Applicative Functors}

  \begin{block}{Question}
    \textbf{Can we combine the advantages of Applicative Functors and Monad?}
  \end{block}

  \begin{block}{What do we want?}
    \begin{itemize}
     \item Conditional Execution of Effects
     \item Know Statically all effects embedded in a computation
    \end{itemize}
  \end{block}

\end{frame}

\begin{frame}{Agenda}
  \section{Selective Applicative Functors}
  \tableofcontents[currentsection]
\end{frame}

\begin{frame}[fragile]{Selective Applicative Functors}

  \begin{block}{Definition}
    It is an \textbf{abstraction} that provides a way to \textbf{embed pure}
    values into an \textbf{effectful context} $f$ and
    composing \textit{two independent} effects in which the \textbf{second} computation
    depends on the \textbf{first}.
  \end{block}

\begin{lstlisting}[language=Haskell]
  class Applicative f => Selective f where
    select :: f (Either a b) -> f (a -> b) -> f b

    -- An infix left-associative synonym for select
    (<*?) :: f (Either a b) -> f (a -> b) -> f b
    (<*?) = select 
\end{lstlisting}
  
  \begin{block}{Contributions}
    \begin{itemize}
    \item Introduce this Model
    \item See two industrial case studies 
    \end{itemize}
  \end{block}

\end{frame}

\begin{frame}[fragile]{Selective Applicative Functors}

\begin{lstlisting}[language=Haskell]
 selectM :: Monad f => f (Either a b) -> f (a -> b) -> f b
 selectM x y = x >>= \e -> case e of 
                              Left  a -> ($a) <$> y -- Execute y
                              Right b -> prue b     -- Skip y
\end{lstlisting}

\begin{lstlisting}[language=Haskell]
 selectA :: Applicative f => f (Either a b) -> f (a -> b) -> f b
 selectA x y = x >>= (\e f -> either f id e) <$> x <*> y -- Execute x and y 
\end{lstlisting}

  \begin{block}{Hints}
    \begin{itemize}
    \item \lstinline{selectM} is useful for conditional execution
    \item \lstinline{selectA} is useful for static analysis
    \end{itemize}
  \end{block}
  
\end{frame}

\begin{frame}[fragile]{Selective Applicative Functors}

\textbf{Combinators}

\begin{lstlisting}[language=Haskell]
  whenS :: Selective f => f Bool -> f () -> f () 
 
  branch :: Selective f => f (Either a b) -> f (a -> c) -> f (b -> c) -> f c
  
  ifS :: Selective f => f Bool -> f a -> f a -> f a

  ....

  (<||>) :: Selective f => f Bool -> f Bool -> f Bool
  (<&&>) :: Selective f => f Bool -> f Bool -> f Bool
  fromMaybeS :: Selective f => f a -> f (Maybe a) -> f a
  anyS :: Selective f => (a -> f Bool) -> [a] -> f Bool 
  allS :: Selective f => (a -> f Bool) -> [a] -> f Bool 
  whileS :: Selective f => f Bool -> f ()
\end{lstlisting}
\end{frame}

\begin{frame}
  \frametitle{Agenda}
  \section{Selective Applicative Functor Laws}
  \tableofcontents[currentsection]
\end{frame}
\begin{frame}[fragile]{Selective Applicative Functor Laws}

  \begin{block}{}
   \textcolor{red}{And we have LAWS!!!!}    
 \end{block}

\begin{lstlisting}[language=Haskell]
  -- Identity
  x <*? pure id = either id id <$> x
 
  -- Distributivity
  pure x <*? (y *> z) = (pure x <*? y) *> (pure x <*? z)

  -- Associativity 
  x <*? (y <*? z) = (f <$> x) <*? (g <$> y) <*? (h <$> z)
    where
      f x = Right <$> x
      g y = \a -> bimap (,a) ($a) y
      h z = uncurry z
\end{lstlisting}
\end{frame}


\begin{frame}
  \frametitle{Agenda}
  \section{Real Case Implementations}
  \tableofcontents[currentsection]
\end{frame}

\begin{frame}[fragile]{Real Case Implementations}

  \begin{block}{Dune Build System}
    \begin{itemize}
      \item Over-approximation for solving dependencies without building 
      \item Under-approximation for maximizing parallelism during build
      \item Original Implementation with Arrows Abstraction
    \end{itemize}
  \end{block}

\begin{lstlisting}[language=Haskell]
  newtype Task k v = Task { run :: forall f. Selective f => (k -> f v) -> f v }

  dependenciesOver :: Task k v -> [k]
  dependenciesOver task = getOver $ run task (\k -> Over [k])

  dependenciesUnder :: Task k v -> [k]
  dependenciesUnder task = getUnder $ run task (\k -> Under [k])

\end{lstlisting}
\end{frame}

\begin{frame}[fragile]{Real Case Implementations}
  \begin{forest}
    [release.tar
    [LICENSE]
    [exe
      [src.ml] [config] [\textbf{lib.c}] [\textbf{lib.ml}]
    ]
    ]
 \end{forest}
  
\begin{lstlisting}[language=Haskell]
  
  !$\lambda$!> dependenciesOver (fromJust $ script "release.tar")
  ["LICENSE","exe"]

  !$\lambda$!> dependenciesUnder (fromJust $ script "release.tar")
  ["LICENSE","exe"]

  !$\lambda$!> dependenciesOver (fromJust $ script "exe")
  ["src.ml","config", "lib.c","lib.ml"]

  !$\lambda$!> dependenciesUnder (fromJust $ script "exe")
  ["src.ml","config"]

\end{lstlisting}
\end{frame}

\begin{frame}[fragile]{Real Case Implementations}

  \begin{block}{Haxl Facebook}
    \begin{itemize}
    \item Speculative or Conditional Execution 
    \item Effective Parallelization on Speculative for large requests 
    \end{itemize}
  \end{block}

\begin{lstlisting}[language=Haskell]
  instance Selective Haxl where
    select (Haxl iox) (Haxl iof) = Haxl $ do
      rx <- iox
      rf <- iof
      return $ case (rx, rf) of
                (Done (Right b), _ ) -> Done b -- Abandon Second Computation
                (Done (Left a), _ ) -> ($a) <$> rf
                (_ , Done f) -> either f id <$> rx
                (Blocked bx x , Blocked bf f) -> Blocked (bx <> bf) (select x f)-- Selective execution
\end{lstlisting}
\end{frame}

\begin{frame}
  \frametitle{Agenda}
  \section{Related Work}
  \tableofcontents[currentsection]
\end{frame}

\begin{frame}[fragile]{Related Work}
  \begin{block}{Arrows}
    \begin{itemize}
    \item Generalize computation expliciting input and output. Instead of
      \textbf{f a} effect we have \textbf{a i o} with an isomorphism \textbf{a () (i -> o)}
    \item Define a similar abstraction to \textbf{Selective} called \textbf{ArrowChoice}
    \item \textbf{Dune} could have been done with Arrows for static analysis.
    \item \textbf{Haxl} should have been more difficult because it is a big
      codebase base on Applicative and Monads, therefore a complete rewriting
      should have been needed. 
    \end{itemize}
  \end{block}
\end{frame}
  
 \begin{frame}[fragile]{Related Work}
  \begin{block}{Profunctors}
    \begin{itemize}
    \item \textbf{Profunctors} are \textbf{bifunctors} that are \textbf{contravariant} in their first type
      argument and \textbf{covariant} in their second one.
    \item Define also an abstraction similar to \textbf{Selective} called \textbf{Cocartesian profunctor}
    \item It is beyond the scope of this work to analyze this relationship and
      their implications
    \end{itemize}
  \end{block}
\end{frame}

\begin{frame}[fragile]{Related Work} 
  \begin{block}{Parse Combinators}
    \begin{itemize}
    \item \textbf{Alternative} is another abstraction which was motivated by
      \textit{Monadic Parsers}
    \item As \textbf{Selective} allows to do choices computations on parsers
    \item There is an interesting research opportunity for analyzing the
      relationship between these two abstraction
    \end{itemize}
  \end{block}


\end{frame}

\begin{frame}
  \frametitle{Agenda}
  \section{Conclusions}
  \tableofcontents[currentsection]
\end{frame}
\begin{frame}{Conclusions}
  \begin{itemize}
    \item New \textbf{Abstraction} linked to \textbf{Applicative Functors and
        Monads}
    \item \textbf{Selective Applicative Functors} requires all effects to be
      known statically, before execution starts.
    \item Also allows \textbf{dependent and conditional} computation
    \item We have also the \textbf{usefulness} with 2 industrial real case implementation 
  \end{itemize}
\end{frame}

\begin{frame}
  \begin{center}
    \textbf{\huge{Thank you!!}}
    \end{center}
\end{frame}
 
\end{document}

