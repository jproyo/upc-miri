\documentclass{beamer}
\usepackage[T1]{fontenc}
\usepackage[utf8]{inputenc}
\usepackage{amsmath}
\usepackage{amsfonts}
\usepackage{amsthm}
\usepackage{array}
\usepackage{graphicx}
\usepackage{listings}
\usepackage{color}
\usepackage{fancyhdr}
\usepackage{tikz}
\usetikzlibrary{calc,shapes.multipart,chains,arrows}
\usepackage{float}
\usepackage{caption}
\usepackage[linguistics]{forest}
\usepackage[cache=false]{minted}
\usemintedstyle{default}
\newminted{haskell}{}
\graphicspath{{./images/}}
\setbeamertemplate{frametitle}{%
  \usebeamerfont{frametitle}\insertframetitle%
  \vphantom{g}% To avoid fluctuations per frame
  % \hrule% Uncomment to see desired effect, without a full-width hrule
  \par\hspace*{-\dimexpr0.5\paperwidth-0.5\textwidth}\rule[0.5\baselineskip]{\paperwidth}{0.4pt}
  \par\vspace*{-\baselineskip}% <-- reduce vertical space after rule
}

\usetheme{Boadilla}
\title{Boolean Combinations of WVG}

\author{Juan Pablo Royo Sales}
\institute{Universitat Politècnica de Catalunya}
\date{January 2020}
\begin{document}

\begin{frame}
\titlepage
\end{frame}

\begin{frame}[fragile]{Definition of the Problem}

  \begin{block}{}
    \begin{itemize}
      \item Set of $n$ players 
      \item Partitioned into two groups. 
      \item \textbf{Bad pairings}: \textit{two players in such a pair do not want to be in the same group}.
      \item Players free to choose any group.
      \item Modeled as a Graph $G = (V,E)$, such that there is an edge $(i, j) \in E$ if $i$ and $j$ form
      a bad pair.
      \item The private objective of player $i$ is to maximize the number of its neighbors that
      are in the \textbf{other group}.
    \end{itemize}
  \end{block}

\end{frame}


\begin{frame}[fragile]{Formalization}
  \begin{block}{Given the problem definition}
    \begin{itemize}
      \item Let $G=(V,E)$ be the graph representing with $V$ the number of players as vertices and $E$ the set of edges such as $\forall_{i,j \in V}\ (i,j) \in E$ if it is a bad pairing.
      \item $N = V$.
      \item $\forall_{i \in N}, A_i = \{1,0\}$ $1$ if it is in Group 1, $0$ if it is in Group 2.
    \end{itemize}
  \end{block}
\end{frame}

\begin{frame}[fragile]{Formalization}
  \begin{block}{Pay-off function}
    \[
u_i(v_2,\dots,v_n) = 
    \begin{cases}
      1 & \exists\ j \neq i, (i,j) \in E \text{ and $j$ is in the other group of $i$} \\
      0 & otherwise 
    \end{cases} 
\]
  \end{block}
\end{frame}

\begin{frame}[fragile]{Formalization}
  \begin{block}{Best Response Function}
    \[
BR_i(v_{-i}) = 
    \begin{cases}
      \{1\} & \forall_{j \neq i}, (i,j) \in E \implies \sum_j 1 - v_j > \sum_j v_j\\ 
      \{0\} & \forall_{j \neq i}, (i,j) \in E \implies \sum_j v_j > \sum_j 1 - v_j\\
      \{0,1\} & otherwise
    \end{cases}
\]
  \end{block}
\end{frame}

\begin{frame}[fragile]{NPE Analysis}
\newtheorem{npe}{NPE Analysis}
\newtheorem{proofpart}{Part}
\begin{npe}
  $v=(v_1,\dots,v_n)$ is a $NPE \iff \forall_{i,j}, (i,j) \in E\ \sum_{i=1}^n 1 - v_i = \sum_{i=1}^n v_j$
\end{npe}
\end{frame}

\begin{frame}[fragile]{NPE Analysis}
\begin{proof}

\begin{proofpart}
$\impliedby$
$\forall_{i,j}, (i,j) \in E\ \sum_{i=1}^n 1 - v_i = \sum_{i=1}^n v_j \implies$
\begin{subequations}
  \begin{align}
     &\implies \forall_{i,j}, (i,j) \in E\ \sum_{j \neq i} 1 - v_i > \sum_{j \neq i} v_i \implies \{1\} \in BR_i\\
     &\implies \forall_{i,j}, (i,j) \in E\ \sum_{j \neq i} 1 - v_i < \sum_{j \neq i} v_i \implies \{0\} \in BR_i
  \end{align}
\end{subequations}
\end{proofpart}
\end{proof}
\end{frame}

\begin{frame}[fragile]{NPE Analysis}
\begin{proof}
\begin{proofpart}
  $\implies$\\
  $\forall_{i,j}, (i,j) \in E\ \sum_{i=1}^n 1 - v_i \neq \sum_{i=1}^n v_j$ if this holds, then \newline
  \begin{subequations}
    \begin{align}
       \exists j &, v_j = 1 \iff (i,j) \in E\ \land \sum_{j \neq i} 1 - v_i < \sum_{j \neq i} v_i + v_j\\
                 &, v_j = 0 \iff (i,j) \in E\ \land \sum_{j \neq i} (1 - v_i)+v_j > \sum_{j \neq i} v_i\\
    \end{align}
  \end{subequations}
  
  But if this is true $i$ changes to other group.\newline
  Therefore $\forall_{i,j}, (i,j) \in E\ \sum_{i=1}^n 1 - v_i = \sum_{i=1}^n v_j$
  \end{proofpart}
  
  \end{proof}
\end{frame}

\begin{frame}[fragile]{Complexity Analysis}
  \begin{block}{}
    \begin{itemize}
      \item Polinomial in the size of $V$ by the size of $E$
      \item Problem of type \textbf{\textit{Strategy General Form}}
      \item Therefore, $O(|V||E|)$
    \end{itemize}
  \end{block}
\end{frame}

\begin{frame}
  \begin{center}
    \textbf{\huge{Thank you!!}}
    \end{center}
\end{frame}

\end{document}

