\documentclass{beamer}
\usetheme{Madrid}
\usepackage{bookmark}
\usepackage[T1]{fontenc}
\usepackage[utf8]{inputenc}
\usepackage{amsmath}
\usepackage{amsfonts}
\usepackage{amsthm}
\usepackage{array}
\usepackage{graphicx}
\usepackage{listings}
\usepackage{color}
\usepackage{fancyhdr}
\usepackage{tikz}
\usetikzlibrary{calc,shapes.multipart,chains,arrows}
\usepackage{float}
\usepackage{caption}
\usepackage[linguistics]{forest}
\setbeamertemplate{frametitle}{%
  \usebeamerfont{frametitle}\insertframetitle%
  \vphantom{g}
  \par\hspace*{-\dimexpr0.5\paperwidth-0.5\textwidth}\rule[0.5\baselineskip]{\paperwidth}{0.4pt}
  \par\vspace*{-\baselineskip}
}
\setbeamertemplate{footline}
{
  \leavevmode%
  \hbox{%
  \begin{beamercolorbox}[wd=.333333\paperwidth,ht=2.25ex,dp=1ex,center]{author in head/foot}%
    \usebeamerfont{author in head/foot}\insertauthor
  \end{beamercolorbox}%
  \begin{beamercolorbox}[wd=.333333\paperwidth,ht=2.25ex,dp=1ex,center]{title in head/foot}%
    \usebeamerfont{title in head/foot}\insertshortinstitute
  \end{beamercolorbox}%
  \begin{beamercolorbox}[wd=.333333\paperwidth,ht=2.25ex,dp=1ex,right]{date in head/foot}%
    \usebeamerfont{date in head/foot}\insertshortdate{}\hspace*{2em}
    \insertframenumber{} / \inserttotalframenumber\hspace*{2ex} 
  \end{beamercolorbox}}%
  \vskip0pt%
}

\usetheme{Boadilla}

\title{Boolean Combinations of Weighted Voting Games}

\author{Juan Pablo Royo Sales}
\institute{Universitat Politècnica de Catalunya}
\date{January 2020}
\begin{document}

\begin{frame}
\titlepage
\end{frame}
  
\begin{frame}[fragile]{Agenda}
  \tableofcontents
\end{frame}

\begin{frame}[fragile]{Agenda}
  \section{Introduction}
  \tableofcontents[currentsection]
\end{frame}

  
\begin{frame}[fragile]{Introduction}
  \begin{block}{Basic Notions}
    \begin{itemize}
      \item Based on \textit{Boolean Combinations of Weighted Voting Games paper} \textit{\textbf{BWVG}}\footnotemark
      \item It is a natural Generalization over \textbf{Weighted Voting Games}
      \item Intuitively is a decision making process via multiple committees
      \item Each committee has the authority to decide the outcome \textit{"yes" or "no"} about an issue.
      \item And each committee is a WVG 
      \item Individuals can appear in multiple committees
      \item Different committees can have different Threshold values
    \end{itemize}
  \end{block}
  \footnotetext{\tiny{Piotr Faliszewski, Edith Elkind, and Michael Wooldridge. 2009. Boolean combinations of weighted voting games. In Proceedings of The 8th International Conference on Autonomous Agents and Multiagent Systems - Volume 1 (AAMAS '09). International Foundation for Autonomous Agents and Multiagent Systems, Richland, SC, 185–192.}}
\end{frame}

\begin{frame}[fragile]{Introduction}
  \begin{block}{Questions to be answered?}
    \begin{itemize}
      \item Which coalitions might be able to bring the goal about?
      \item How important is a particular individual with respect to the achievement of the goal?
    \end{itemize}
  \end{block}
\end{frame}

\begin{frame}[fragile]{Introduction}
  \begin{block}{Goals of the Paper}
    \begin{itemize}
      \item Formal Definition of \textbf{BWVG}
      \item Investigate Computational Properties of \textbf{BWVG}
    \end{itemize}
  \end{block}
\end{frame}

\begin{frame}[fragile]{Agenda}
  \section{Preliminary Definitions}
  \tableofcontents[currentsection]
\end{frame}

\begin{frame}[fragile]{Preliminary Definitions}
  \begin{block}{Propositional Logic}
    \begin{itemize}
      \item Let $\Phi = \{p,q,\dots\}$ be a fixed non-empty vocabulary of Boolean variables    
      \item Let $\mathcal{L}$ denote the set of formulas of propositional logic over $\Phi$
      \item If "$\lor$" and "$\land$" are the only operators appearing in formula $\varphi$, se say that $\varphi$ is \textbf{monotone}
      \item If $\xi \subseteq \Phi$, we write $\xi \models \varphi$ mean that $\varphi$ is true satisfied by valuation $\xi$
    \end{itemize}
  \end{block}
\end{frame}

\begin{frame}[fragile]{Preliminary Definitions}
  \begin{block}{Simple Games}
    \begin{itemize}
      \item A coalitional game is Simple if $v(C) \in \{0,1\} \forall C \subseteq N$
      \item $C$ wins if $v(C) = 1$ and $C$ losses otherwise.  
      \item A Simple Game is \textbf{monotone} if $v(C) = 1 \implies v(C') = 1$ for any $C \subseteq C'$.
      \item In this paper authors consider both \textit{monotone} and \textit{non-monotone} Simple Games.
      \item They assume games with finite numbers of players $|N| = n$, $N = \{1,\dots,n\}$
    \end{itemize}
  \end{block}
\end{frame}

\begin{frame}[fragile]{Preliminary Definitions}
  \begin{block}{Weighted Voting Games}
    \begin{itemize}
      \item Given $N = \{1,\dots,n\}$ players
      \item A list of $n$ weights $w = (w_1, \dots, w_n) \in \mathbb{R}^n$
      \item A threshold $T \in \mathbb{R}$
      \item When $N$ is clear from the context $q = (T; w_1, \dots, w_n)$ to denote a $WVG$ $g$
      \item $w(C)$ total weight of coalition $C$, $w(C) = \sum_{i \in C} w_i$ 
      \item Characteristic function given by $v(C) = 1$ if $w(C) \geq T$ and $v(C) = 0$ otherwise.
      \item If all Weights are non-negative the game is monotone.
    \end{itemize}
  \end{block}
\end{frame}

\begin{frame}[fragile]{Preliminary Definitions}
  \begin{block}{Computational Complexity}
    \begin{itemize}
      \item $P$, $NP$, $coNP$, $\Sigma_2^p$, $\Pi_2^p$
      \item $D^p$: A Language $L \in D^p$ if $L = L_1 \cap L_2$, for some language $L_1 \in NP$ and $L_2 \in coNP$
      \item $D_2^p$: A Language $L \in D_2^p$ if $L = L_1 \cap L_2$, for some language $L_1 \in \Sigma_2^p$ and $L_2 \in \Pi_2^p$
      \item A Language $L \in UP$ if its Characteristic Function is in $\# P$
    \end{itemize}
  \end{block}
\end{frame}

\begin{frame}[fragile]{Agenda}
  \section{Formal Definition BWVG}
  \tableofcontents[currentsection]
\end{frame}

\begin{frame}[fragile]{Boolean Weighted Voting Games}
  \begin{block}{Definition}
    A \textbf{BWVG} is a tuple $G = \langle N, \mathcal{G}, \Phi, \varphi \rangle$, where:
    \begin{itemize}
      \item $N = \{1,\dots,n\}$ is a set of players;
      \item $\mathcal{G} = \{g^1, \dots, g^n\}$ is a Set of \textbf{WVG} over $N$, where $j$th game, $g^j$, is
      given by a vector of weights $w^j = (w_1^j, \dots, w_n^j)$ and a Threshold $T^j$. $\mathcal{G}$ is called the \textbf{component games} of $G$;
      \item $\Phi = \{p^1, \dots, p^n\}$ Set of Propositional Variables, in which each $p^j$ correspond with the \textbf{component} $g^j$;
      \item $\varphi$ is a propositional formula over $\Phi$. 
    \end{itemize}
  \end{block}
  \begin{block}{Shorthand Definition}
    Example:
    \begin{itemize}
      \item $g^1 \land g^2 \equiv \langle N, \{g^1,g^2\}, \{p^1, p^2\}, p^1 \land p^2 \rangle$
    \end{itemize}
  \end{block}
\end{frame}

\begin{frame}[fragile]{Boolean Weighted Voting Games}
  \begin{block}{Winning Coalition}
    We say that $C$ is a \textit{wins} $G$ if:
    \begin{equation}
      \exists \xi_1 \subseteq \Phi_C\ :\ \forall \xi_2 \subseteq (\Phi \setminus \Phi_C)\ :\ \xi_1 \cup \xi_2 \models \varphi \label{eq:1}
    \end{equation}
  \end{block}
  \begin{block}{Intuitively~\ref{eq:1}}
    A coalition $C$ wins if it is able to fix variables under its control in such a way that the goal formula $\varphi$ is guaranteed to be \textbf{True}.
  \end{block}
  \begin{block}{Notes}
    It is allowed \textbf{\textit{BWVG}} to contain \textit{negative} weights
  \end{block}
\end{frame}

\begin{frame}[fragile]{Agenda}
  \section{Representational Complexity}
  \tableofcontents[currentsection]
\end{frame}

\begin{frame}[fragile]{Representational Complexity}
  \begin{block}{Preliminaries}
    \begin{itemize}
      \item Any Simple Game with $n$ players can be represented as a \textit{K-Vector Weighted Voting Game}
      for $k = O(2^n)$, and therefore as a \textbf{BWVG} with $O(2^n)$ \textbf{component games} $\mathcal{G}$.
      \item That worst-case can be improved in \textbf{BWVG}
    \end{itemize}
  \end{block}
\end{frame}

\begin{frame}[fragile]{Representational Complexity}
  \newtheorem{prop1}{Proposition 1}
  \begin{prop1}
    The total number of Boolean weighted voting games with $|N| = n$ and $|\varphi| = s$ is
    most $2^{O(sn^2 \log(sn))}$
  \end{prop1}
  \begin{proof}
    \begin{itemize}
      \item Any weighted voting game\footnotemark\ can be represented using Integer weights whose absolute values do not exceed $2^{O(n \log{n})}$
      \item \textit{w.l.g.} we assumed that $|\mathcal{G}| = |\Phi|$ and $|\Phi| \leq |\varphi| = s$
      \item Given a \textbf{BWVG} $G$ with $n$ players and $|\varphi| = s$, we can find a equivalent representation using $O(sn^2 \log{n})$
      bits to represent all weights in ALL components, plus another $O(s \log{s})$ bits to represent $\mathcal{G}, \Phi$ and $\varphi$.
      \item Therefore, the total number of \textbf{distinct games} can be represented as \textbf{BWVG} with $|N| = n$ and $|\varphi| = s$ is
      $2^{O(sn^2 \log(sn))}$
    \end{itemize}
  \end{proof}
  \footnotetext{\tiny{S. Muroga. Threshold Logic and its Applications. Wiley, 1971.}}
\end{frame}


\begin{frame}[fragile]{Agenda}
  \section{Shapley Value}
  \tableofcontents[currentsection]
\end{frame}

\begin{frame}[fragile]{Shapley Value}
  \begin{block}{}
    asdfasfsd
  \end{block}
\end{frame}

\begin{frame}[fragile]{Agenda}
  \section{The Core}
  \tableofcontents[currentsection]
\end{frame}

\begin{frame}[fragile]{The Core}
  \begin{block}{}
    asdfasfsd
  \end{block}
\end{frame}

\begin{frame}
  \begin{center}
    \textbf{\huge{Thank you!!}}
    \end{center}
\end{frame}

\end{document}