\documentclass[12pt, a4paper]{article}
\usepackage[utf8]{inputenc}
\usepackage{amsmath}
\usepackage{amsthm}
\usepackage{graphicx}
\usepackage{parskip}
\usepackage{titling} 
\usepackage{hyperref}
\usepackage{fancyhdr}
\usepackage{lastpage}
\graphicspath{ {./images/} }
\title{
  Lab Session 2 - Answers \\
  AMMM
}
\author{Juan Pablo Royo Sales}
\date\today

\pagestyle{fancy}
\fancyhf{}
\fancyhead[C]{}
\fancyhead[R]{Juan Pablo Royo Sales - UPC MIRI}
\fancyhead[L]{AMMM - Lab Session 2}
\fancyfoot[L,C]{}
\fancyfoot[R]{Page \thepage{} of \pageref{LastPage}}
\renewcommand{\headrulewidth}{0.4pt}
\renewcommand{\footrulewidth}{0.4pt}

\begin{document}

\begin{titlingpage}
  \maketitle
\end{titlingpage}

\section{Answers}
\subsection{Answer 1}\label{answer_1}

On the P1 solution, we have the following results:

\begin{itemize}
  \item Constraints 10
  \item Variables 13
  \item Optimal solutions: we have the following matrix

\begin{subequations}
  \begin{align}
    z = 1
    x_tc = [[1 0 0], [0 0.59151 0.40828] [0.44636 0.55364 0] [1 0 0]]
  \end{align}
\end{subequations}

 \item Regarding solving time is taking $0.1090$ 
\end{itemize}

On the P2 IP solution we have the following:

\begin{itemize}
\item Constraints is the same 10, because we are working with the constraints model
\item Variables in this case we change to 12 Binaries vars and 1 of other. In
  this case it is because we have 3 CPUS and 4 Tasks which is the 12 binaries we
  have defined. The other is a free variable.
  \item Optimal solutions: we have the following matrix

\begin{subequations}
  \begin{align}
    z = 1
    x_tc = [[1 0 0], [0 0 0] [0 1 0] [1 0 0]]
  \end{align}
\end{subequations}

We can check here since we are with binaries solution that compare the others
all the values that are bellow 0.5 are round down to 0, which is far away from
optimal in this particular case.

\item Regarding solving time is taking $0.1360$. It is greater because we are
  working with integer numbers and the resource allocation is greater for this
  type of problem 
\end{itemize}


\subsection{Answer 2}
The problem is now unsatisfiable because the capacity constraint in the IP
problem cannot be satisfied because of the following:

\begin{itemize}
\item Task $560.89$ only can go to CPU with $701.78$
  \item When that happen there is no more capacity to attend to the rest of the
    task as the following sum shows

    \begin{subequations}
      \begin{align}
        sum_rc = 505.67 + 503.68 = 1009.35 \\
        sum_rt = 261.27 + 310.51 + 105.80 + 344.7 = 1022.28 \\
        sum_rc < sum_rt 
      \end{align}
    \end{subequations}

    Therefore there is no capacity.
\end{itemize}

\subsection{Answer 3}
The value of the highest loaded computer for each value of $K | \forall k \in K
1 <= k <= 5$ is the following:

\begin{itemize}
\item $K = 1 \implies CPU_3 = 0.64\%$
\item $K = 2 \implies CPU_3 = 0.44\%$
\item $K = 3 \implies CPU_3 = 0.37\%$
\item $K = 4 \implies CPU_3 = 0.15\%$
\item $K = 5 \implies CPU_3 = 0\%$
\end{itemize}

\subsection{Answer 4}

On the P2 solution, we have the following results:

\begin{itemize}
  \item Constraints 11
  \item Variables 15 binaries and 1 of Other kind
  \item Regarding solving time is taking $0.2578$. This is the most time
    consuming because it is not feasible.
\end{itemize}

On the P2a solution with $K=1$ we have the following:

\begin{itemize}
\item Constraints is 12 because here there has been added another contratint.
\item Variables are the same as before 15 binaries and 1 of other kind.
\item Regarding solving time is taking $0.2001$.
\end{itemize}

On the P2b solution we have the following:

\begin{itemize}
\item Constraints is 8 because here we are eliminating one constraint and
  minimizing the amount of server loaded.
\item Variables are the same as before 15 binaries, because the constraint elimination.
\item Regarding solving time is taking $0.2419$. It is taking more that P2a
  because here there is not rebalancing of the tasks with $K$ parameter, so
  optimization is more time consuming.
\end{itemize}


\end{document}