\documentclass[12pt, a4paper]{article}
\usepackage[utf8]{inputenc}
\usepackage{amsmath}
\usepackage{amsthm}
\usepackage{amsfonts}
\usepackage{graphicx}
\usepackage{parskip}
\usepackage{hyperref}
\usepackage{fancyhdr}
\usepackage{lastpage}
\usepackage{tikz}
\usepackage{float}
\usepackage{listings}
\usepackage{color}
\usepackage{caption}
\usepackage[vlined,ruled]{algorithm2e}
\usepackage[acronym]{glossaries}
\usepackage[nottoc]{tocbibind}
\usepackage[cache=false]{minted}
\usepackage{mathtools}
\usemintedstyle{default}
\graphicspath{{./images/}}
\newminted{haskell}{frame=lines,framerule=2pt}
\allowdisplaybreaks[2]

\DeclareMathOperator*{\argmax}{argmax}
\DeclareMathOperator*{\st}{st}
\DeclarePairedDelimiter\floor{\lfloor}{\rfloor}


\begin{document}

\begin{enumerate}
  \item The number of possible configurations is $13$. Let $x,y$ be the number of items of size $6$ and $10$, respectively, in a configuration. The capacity of the bins is $30$, and then, the number of possible configurations corresponds to the non-negative integer solutions of $6x+10y \leq 30$, $x \geq 0$ and $y \geq 0$. To count the number of solutions of this system one can fix the value of yy and then find the possible values for xx.

  \begin{itemize}
    \item If $y=0$, then $x \in \{0,1,2,3,4,5\}$
    \item If $y=1$, then $x \in \{0,1,2,3\}$
    \item If $y=2$, then $x \in \{0,1\}$
    \item If $y=3$, then $x = 0$
  \end{itemize}

  \item $x_{C_1} = 2,8$, so we can store $10$ items of size $6$ in $2$ bins and the remaining $4$ in another bin which is going to occupy $24/30 = 0.8$ of the new bin. Then we have the $2.8$ of the LP solution for $C_1$.\\
  In the case of $x_{C_2} = 2.6666666667$ we store $6$ items of size $10$, and the remaning $2$ items in another bin which is going to occupy $20/30 = 0.6666666667$ of the new bin. Then we have the $2.6666666667$ of the LP solution for $C_2$.\\
  Therefore it is feasiable.
  \item Yes, because $t = 2$ and all the bins ends ups with $\text{\#entries} \geq t$.
  \item $2$ in each configuration
  \item \begin{itemize}
    \item $4$ of size $6$
    \item $2$ of size $10$
  \end{itemize}
  \item $1$ for each configuration
  \item $2$ no matter the order of the remaining entries of different sizes.
  \item \begin{itemize}
    \item If we sum up all the sizes we have $164$ in total.
    \item $164/30 = 5,4666666667$, so the Optimal fractional should be $5.46$ bins.
    \item Since we are working with integer number of bins the optimal is $6$. 
    \item In the algorithm we are using as i state in the answer 2, $2$ bins for $C_1$ and $2$ bins for $C_2$ plus $2$ bins more as I stated in answer $7$. There fore we are using $6$ bins according to the algorithm and it is Optimal.
  \end{itemize}
  
\end{enumerate}


\end{document}

