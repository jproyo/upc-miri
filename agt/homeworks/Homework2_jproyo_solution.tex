\documentclass[12pt, a4paper]{article}
\usepackage[utf8]{inputenc}
\usepackage{amsmath}
\usepackage{amsthm}
\usepackage{amssymb}
\usepackage{graphicx}
\usepackage{parskip}
\usepackage{hyperref}
\usepackage{fancyhdr}
\usepackage{lastpage}
\usepackage[vlined,ruled]{algorithm2e}
\usepackage[acronym]{glossaries}
\usepackage{caption}
\usepackage{titlesec}
\usepackage{multirow}


\title{%
  Algorithmic Game Theory \\
  Homework 2 - Solutions
}
\author{%
  Juan Pablo Royo Sales\\
  \small{Universitat Politècnica de Catalunya}
}
\date\today

\pagestyle{fancy}
\fancyhf{}
\fancyhead[C]{}
\fancyhead[R]{Juan Pablo Royo Sales - UPC MIRI}
\fancyhead[L]{AGT - Homework 2}
\fancyfoot[L,C]{}
\fancyfoot[R]{Page \thepage{} of \pageref{LastPage}}
\setlength{\headheight}{15pt}
\renewcommand{\headrulewidth}{0.4pt}
\renewcommand{\footrulewidth}{0.4pt}

\renewcommand{\qedsymbol}{$\blacksquare$}

\begin{document}

\maketitle

\section{Problem 10}
\begin{center}
  \begin{tabular}{p{5pt}c|cc }
  \multicolumn{2}{}{} & \multicolumn{2}{c}{P2}\\
  & & A & B\\
  \multirow{2}{*}{\rotatebox[origin=c]{90}{P1}}
  & A & 6,6 & 2,7 \\
  & B & 7,2 & 0,0 \\
\end{tabular}
\end{center}

Given $\sigma^{*} = \{(x, 1-x), (y, 1-y)\}$

For $P1$ we know that
\begin{itemize}
  \item If $x \neq 0 \land x \neq 1 \implies \text{ supp}(\sigma_{p1})=\{A,B\}$.\\
       Therefore $\mu_1((1,0),(y,1-y)) = \mu_1((0,1), (y, 1-y))$
  \item If $x \neq 0 \land x \neq 1 \implies \text{ supp}(\sigma_{p1})=\{A,B\}$.\\
       Therefore $\mu_1((1,0),(y,1-y)) = \mu_1((0,1), (y, 1-y))$
  \item If $x = 1 \implies \text{ supp}(\sigma_{p1})=\{A\}$.\\
       Therefore $\mu_1((1,0),(y,1-y)) > \mu_1((0,1), (y, 1-y))$
  \item If $x = 0 \implies \text{ supp}(\sigma_{p1})=\{B\}$.\\
       Therefore $\mu_1((1,0),(y,1-y)) < \mu_1((0,1), (y, 1-y))$
\end{itemize}


And for $P2$ we know that
\begin{itemize}
  \item If $y \neq 0 \land y \neq 1 \implies \text{ supp}(\sigma_{p2})=\{A,B\}$.\\
       Therefore $\mu_2((x,1-x),(1,0)) = \mu_2((x, 1-x),(0,1))$
  \item If $y \neq 0 \land y \neq 1 \implies \text{ supp}(\sigma_{p2})=\{A,B\}$.\\
       Therefore $\mu_2((x,1-x),(1,0)) = \mu_2((x, 1-x),(0,1))$
  \item If $y = 1 \implies \text{ supp}(\sigma_{p2})=\{A\}$.\\
       Therefore $\mu_2((x,1-x),(1,0)) > \mu_2((x, 1-x),(0,1))$
  \item If $y = 0 \implies \text{ supp}(\sigma_{p2})=\{B\}$.\\
       Therefore $\mu_2((x,1-x),(1,0)) < \mu_2((x, 1-x),(0,1))$
\end{itemize}

Lets analyze case by case:

\begin{enumerate}
  \item $P1 = A, P2 = A$\\
 \begin{subequations}
  \begin{align}
    \mu_1((1,0),(1,0)) &> \mu_1((0,1),(1,0))\\
    6 &> 7
  \end{align}
\end{subequations}
Therefore, this \textbf{IS NOT $NE$}

 \item $P1 = A, P2 = B$\\
 \begin{subequations}
  \begin{align}
    \mu_1((1,0),(0,1)) &> \mu_1((0,1),(0,1))\\
    2 &> 0
  \end{align}
\end{subequations}
Therefore, this \textbf{IS $NE$}

\item $P1 = A, P2 = \{A,B\}$\\
 \begin{subequations}
  \begin{align}
    6y &= 2(1-y)\\
    8y &= 2\\
     y &= \frac{1}{4}\\
    1-y &= \frac{3}{4}
  \end{align}
\end{subequations}
Therefore, this \textbf{IS MIXED $NE$}


 \item $P1 = B, P2 = A$\\
 \begin{subequations}
  \begin{align}
    \mu_1((0,1),(1,0)) &> \mu_1((1,0),(1,0))\\
    7 &> 6
  \end{align}
\end{subequations}
Therefore, this \textbf{IS $NE$}

 \item $P1 = B, P2 = B$\\
 \begin{subequations}
  \begin{align}
    \mu_1((0,1),(0,1)) &> \mu_1((1,0),(0,1))\\
    0 &> 2
  \end{align}
\end{subequations}
Therefore, this \textbf{IS NOT $NE$}

\item $P1 = B, P2 = \{A,B\}$\\
 \begin{subequations}
  \begin{align}
    7y &= 0(1-y)\\
     y &= 0\\
  \end{align}
\end{subequations}
Therefore, this \textbf{IS NOT MIXED $NE$}

\item $P1 = \{A,B\}, P2 = A$\\
 \begin{subequations}
  \begin{align}
    6x &= 2(1-x)\\
    x &= \frac{1}{4}\\
    1 - x &= \frac{3}{4}
  \end{align}
\end{subequations}
Therefore, this \textbf{IS MIXED $NE$}


\item $P1 = \{A,B\}, P2 = B$\\
 \begin{subequations}
  \begin{align}
    7x &= 0(1-x)\\
     x &= 0\\
  \end{align}
\end{subequations}
Therefore, this \textbf{IS NOT MIXED $NE$}


\item $P1 = \{A,B\}, P2 = \{A,B\}$\\

When $\mu_1((1,0),(y, 1-y)) = \mu_1((0,1),(y,1-y))$
 \begin{subequations}
  \begin{align}
    6y + 7y &= 2(1-y) + 0(1-y)\\
    13y &= 2 - 2-y\\
    y &= \frac{2}{11}\\
    1-y &= \frac{9}{11}
  \end{align}
\end{subequations}
Therefore, this \textbf{IS MIXED $NE$}\\

When $\mu_2((x,1-x),(1,0)) = \mu_2((x,1-x),(0,1))$
 \begin{subequations}
  \begin{align}
    6x + 7x &= 2(1-x) + 0(1-x)\\
    13x &= 2 - 2-x\\
    x &= \frac{2}{11}\\
    1-x &= \frac{9}{11}
  \end{align}
\end{subequations}
Therefore, this \textbf{IS MIXED $NE$}
\end{enumerate}

Therefore the following are the $NE$ and MIXED $NE$ found:

$NE = \{((0,1),(1,0)), ((1,0),(0,1))\}$

$\text{MIXED NE} = \{((0,1),(\frac{1}{4},\frac{3}{4})), ((\frac{1}{4},\frac{3}{4}),(1,0)), ((\frac{2}{11},\frac{9}{11}),(\frac{2}{11},\frac{9}{11}))\}$

\section{Problem 12}

\begin{center}
  \begin{tabular}{p{5pt}c|cc }
  \multicolumn{2}{}{} & \multicolumn{2}{c}{P2}\\
  & & A & B\\
  \multirow{3}{*}{\rotatebox[origin=c]{90}{P1}}
  & C & 1,1 & 4,2 \\
  & D & 3,3 & 1,1 \\
  & E & 2,2 & 2,3
\end{tabular}
\end{center}

Given $\sigma^{*} = \{(x_1, x_2, x_3), (y, 1-y)\}$

For $P1$ we know that
\begin{itemize}
  \item If $x_1,x_2,x_3 \notin \{0,1\} \implies \text{ supp}(\sigma_{p1})=\{C,D,E\}$.\\
    Therefore $\mu_1((1,0,0),(y,1-y)) = \mu_1((0,1,0), (y, 1-y)) = \mu_1((0,0,1), (y, 1-y))$
  \item If $x_1 = 1 \implies \text{ supp}(\sigma_{p1})=\{C\}$.\\
    Therefore $\mu_1((1,0,0),(y,1-y)) > \mu_1((0,1,0), (y, 1-y))\ \land\ \mu_1((1,0,0),(y,1-y)) > \mu_1((0,0,1), (y, 1-y))$
  \item If $x_2 = 1 \implies \text{ supp}(\sigma_{p1})=\{D\}$.\\
    Therefore $\mu_1((0,1,0),(y,1-y)) > \mu_1((1,0,0), (y, 1-y))\ \land\ \mu_1((0,1,0),(y,1-y)) > \mu_1((0,0,1), (y, 1-y))$
  \item If $x_3 = 1 \implies \text{ supp}(\sigma_{p1})=\{E\}$.\\
    Therefore $\mu_1((0,0,1),(y,1-y)) > \mu_1((1,0,0), (y, 1-y))\ \land\ \mu_1((0,0,1),(y,1-y)) > \mu_1((0,1,0), (y, 1-y))$
\end{itemize}

Lets analyze case by case:

\begin{enumerate}
  \item $P1 = C, P2 = A$\\
 \begin{subequations}
  \begin{align}
    \mu_1((1,0,0),(1,0)) &> \mu_1((0,1,0),(1,0))\\
    1 &> 3
  \end{align}
\end{subequations}
Therefore, this \textbf{IS NOT $NE$}

  \item $P1 = C, P2 = B$\\
 \begin{subequations}
  \begin{align}
    \mu_1((1,0,0),(0,1)) &> \mu_1((0,1,0),(0,1))\\
    4 &> 1
  \end{align}
\end{subequations}
 \begin{subequations}
  \begin{align}
    \mu_1((1,0,0),(0,1)) &> \mu_1((0,0,1),(0,1))\\
    4 &> 2
  \end{align}
\end{subequations}
Therefore, this \textbf{IS $NE$}


\item $P1 = C, P2 = \{A,B\}$\\
 \begin{subequations}
  \begin{align}
    y &= 4(1-y)\\
     y &= \frac{4}{5}\\
    1-y &= \frac{1}{5}
  \end{align}
\end{subequations}
Therefore, this \textbf{IS MIXED $NE$}

\item $P1 = D, P2 = A$\\
 \begin{subequations}
  \begin{align}
    \mu_1((0,1,0),(1,0)) &> \mu_1((1,0,0),(1,0))\\
    3 &> 1
  \end{align}
\end{subequations}
 \begin{subequations}
  \begin{align}
    \mu_1((0,1,0),(1,0)) &> \mu_1((0,0,1),(1,0))\\
    3 &> 2
  \end{align}
\end{subequations}
Therefore, this \textbf{IS $NE$}

\item $P1 = D, P2 = B$\\
 \begin{subequations}
  \begin{align}
    \mu_1((0,1,0),(0,1)) &> \mu_1((1,0,0),(0,1))\\
    1 &> 4
  \end{align}
\end{subequations}
Therefore, this \textbf{IS NOT $NE$}

\item $P1 = D, P2 = \{A,B\}$\\
 \begin{subequations}
  \begin{align}
    3y &= 1-y\\
     y &= \frac{1}{4}\\
    1-y &= \frac{3}{4}
  \end{align}
\end{subequations}
Therefore, this \textbf{IS MIXED $NE$}

\item $P1 = E, P2 = A$\\
 \begin{subequations}
  \begin{align}
    \mu_1((0,0,1),(1,0)) &> \mu_1((0,1,0),(1,0))\\
    2 &> 3
  \end{align}
\end{subequations}
Therefore, this \textbf{IS NOT $NE$}


\item $P1 = E, P2 = B$\\
 \begin{subequations}
  \begin{align}
    \mu_1((0,0,1),(0,1)) &> \mu_1((1,0,0),(0,1))\\
    2 &> 4
  \end{align}
\end{subequations}
Therefore, this \textbf{IS NOT $NE$}

\item $P1 = E, P2 = \{A,B\}$\\
 \begin{subequations}
  \begin{align}
    2y &= 2-2y\\
     y &= 0\\
  \end{align}
\end{subequations}
Therefore, this \textbf{IS NOT MIXED $NE$}

\item $P1 = \{C,D,E\}, P2 = \{A,B\}$\\
 \begin{subequations}
  \begin{align}
    \mu_1((1,0,0),(y,1-y)) &= \mu_1((0,1,0), (y, 1-y)) &= \mu_1((0,0,1), (y, 1-y))\\
    y+4-4y &= 3y+1-y &= 2y + 2-2y\label{eq:1}\\
  \end{align}
\end{subequations}
Therefore, this \textbf{IS NOT MIXED $NE$} because~\ref{eq:1} leads to inequality.

\end{enumerate}

Therefore the following are the $NE$ and MIXED $NE$ found:

$NE = \{((1,0,0),(0,1)), ((0,1,0),(1,0))\}$

$\text{MIXED NE} = \{((1,0,0),(\frac{4}{5},\frac{1}{5})), ((0,1,0),(\frac{1}{4},\frac{3}{4}))\}$



\section{Problem 13}
Lets analyze the utility after applying the distribution $(0.6,0.4), (0.2,0.4,0.4)$

\begin{center}
  \begin{tabular}{p{5pt}c|ccc }
  \multicolumn{2}{}{} & \multicolumn{3}{c}{P2}\\
  & & R & S & T\\
  \multirow{2}{*}{\rotatebox[origin=c]{90}{P1}}
  & A & 3.6,1.2 & 1.2,2.8 & 1.2,2.4 \\
  & B & 2.8,0.4 & 0.8,2.8 & 0.8,2.4 \\
\end{tabular}
\end{center}

As we can see there is a $NE$ when $P2$ chooses $S$ and $P1$ chooses $A$ since non of them has any incentive to change.

\end{document}

